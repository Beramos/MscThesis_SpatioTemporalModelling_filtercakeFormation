\chapter[Samenvatting]%
{Samenvatting}
\hyphenation{
be-schrijft
wij-ten
ge-ne-ra-tie
si-mu-la-tie-re-sul-ta-ten
ont-wikkeling
be-schrij-ving
}
\vspace{-2cm}
Momenteel hebben membraanfiltratieprocessen een grote operationele kost die vooral te wijten is aan de genomen maatregelen tegen de vervuiling van het membraan. De beschrijving van deze vervuilingsmechanismen is nog steeds zeer empirisch en biedt geen goed kader voor de ontwikkeling van beslissingsondersteunende systemen. Deze systemen kunnen onder \mbox{andere} de kosteneffectiviteit van deze processen verbeteren. Het doel van deze thesis is de ontwikkeling en evaluatie van een spatio-temporeel model dat het gedrag van gesuspendeerde partikels in een filtratiesysteem op een realistische en accurate manier kan beschrijven. Dit model kan waardevolle inzichten leveren in de belangrijkste onderliggende mechanismen van filterkoekvorming. Deze mechanismen kunnen dan opgenomen worden in de volgende generatie aan beslissingsondersteunende systemen. \par
In het kader van dit proefschrift werd een literatuuronderzoek uitgevoerd dat een overzicht biedt van de huidige membraanvervuilingsmodellen. Tevens wordt een overzicht gegeven van de beschikbare profilometrische technieken voor de kalibratie en validatie van het model in ontwikkeling. Het spatio-temporeel model van \cite{Ghijs2014}, dat de filterkoekvorming beschrijft in membraanbioreactoren, werd verder uitgebouwd naar een driedimensionaal model. Een polydisperse voorstelling van de disperse fase werd bewerkstelligd en een nieuw en \mbox{effici\"ent} collisiedetectie-algoritme werd ge\"implementeerd. Voor de analyse van de simulatieresultaten werd een gebruiksvriendelijke grafische gebruiksomgeving ontwikkeld. \par
Een microfiltratie pilootopstelling  werd ontwikkeld via numerieke stromingsleer voor de kalibratie/validatie van het model en een kwalitatieve validatie werd uitgevoerd op basis van het Segr\'e-Silberberg effect. Dit laatste toonde een aantal onvolmaaktheden aan in de krachtenbalans over de gesuspendeerde partikels. Vervolgens werd een scenarioanalyse uitgevoerd om het gedrag van het model onder verschillende operationele condities te evalueren. \par
De simulatieresultaten en kwalitatieve validatie hebben geleid tot een duidelijk inzicht in enkele tekortkomingen van het model zoals de vorming van smalle filterkoektorens en de afwezigheid van de gepaste wandrepulsie-effecten. Met het oog op de toekomstige ontwikkeling van het model werden een aantal richtlijnen verschaft om deze kwesties aan te pakken. Er kan besloten worden dat er veel vooruitgang geboekt is naar een realistische voorstelling van filterkoekvormende processen, waarbij veel inzicht verworven is in de onderliggende mechanismen; het uiteindelijke hoofddoel van deze dissertatie.




