\documentclass[a4paper,11pt,twoside,openright]{book}

%---------------------------------------------------------------------------
%List of packages with theid respective use, you can uncomment if not used
%---------------------------------------------------------------------------

%--------------layout of the document--------------
% this package is used to get a wider A4 paper
\usepackage{a4wide}
% these packages are used to produce nicer captions
\usepackage[font={small},labelfont={bf,sf},width=0.8\textwidth, singlelinecheck=false]{caption}
%\usepackage[font={small},labelfont={bf,sf},hang,singlelinecheck=false]{caption}
% use the titlesec package to change the section 
\usepackage[calcwidth,newparttoc]{titlesec}
% use the titletoc package to change the toc
\usepackage{titletoc}
% hyperrefs used in the produced pdf file
\usepackage[a4paper,plainpages=false,hidelinks]{hyperref}
% \usepackage[a4paper,plainpages=false,colorlinks=false,allbordercolors=grey,pdfborderstyle={/S/U/W 1}]{hyperref} %underlined hyperref
%\hypersetup{
%}
% this package is used for colored text
\usepackage{xcolor}
%\usepackage[cmyk]{xcolor}
% this package is used for intelligent spaces
\usepackage{xspace}
% this package is a collection of fonts of symbols and characters that aren't always included in standard distributions of TeX
\usepackage{amsfonts}
% this package defines the full set of symbol names for two fonts of extra symbols included in the amsfonts collection
\usepackage{amssymb}
% this package is used to create custom headers and footers
\usepackage{fancyhdr}
\pagestyle{fancy}
% this package used to produce a different style of page numbering
% \usepackage[auto]{chappg}
% this package allows you to have toc's for each chapter
% \usepackage{minitoc}
% package used to make an index
% \usepackage{makeidx} 
% use this package to use times new roman fonts
% \usepackage{times}
% use this package for striketrough fonts
\usepackage[normalem]{ulem}

%--------------figure and table adaptations--------------
% this package is used for figures with are spread over two or more pages
\usepackage{captcont}
% this package is used for creating table with multirows
\usepackage{multirow}
% this package enables one to put several figures next to each other
\usepackage{subfigure}
%Show floats (images, tables, �) only AFTER they have been referenced for the first time:
\usepackage{captcont}
%floats over multiple pages
\usepackage{subcaption}
% better control of subfigure captions
\usepackage{flafter} 
% enables footnotes in a table itself
\usepackage{threeparttable}
% Publication quality tables 
\usepackage{booktabs}  % http://archive.cs.uu.nl/mirror/CTAN/macros/latex/contrib/booktabs/booktabs.pdf
% needed to change rowheight in tables
\usepackage{array}
%figuren opnemen en eps toelaten
\usepackage{graphicx}
%figuren opnemen en eps toelaten
\usepackage{graphicx}
\usepackage{epstopdf}
% this package is the natural frontend for pgf (a package for creating graphics in an inline manner).
%\usepackage{tikz}
%\usetikzlibrary{arrows,shapes,calendar,matrix,backgrounds,folding}
% this package is used to wrap text around figures
% \usepackage{wrapfig}
\usepackage{float}

%--------------mathematical stuff--------------
% this package enhances the typeset appearance of mathematical formulas
\usepackage{amsmath}
% this package provides access to bold math symbols
\usepackage{bm}
% extra symbols available: http://home.online.no/~pjacklam/latex/textcomp.pdf
\usepackage{textcomp}
% larger integrals to adapt to specific use
\usepackage{bigints}
% use this package to use the garamond font
% \usepackage[garamond]{mathdesign}
% this package is used to create partitioned matrices
%\usepackage{pmat}

%--------------Interesting environments style--------------
% these packages provides an environment for algorithms (cfr. Pseudocode)
\usepackage[ruled]{algorithm2e} % info: http://archive.cs.uu.nl/mirror/CTAN/macros/latex/contrib/algorithm2e/doc/algorithm2e.pdf
%\usepackage{algorithm}  
%\usepackage{algorithmic}
% The thmtools bundle is a collection of packages that is designed to provide an easier interface to theorems
%\usepackage{thmtools}
% use this package for url's
\usepackage{url}
% this package is used to get nicer chemical notations
%\usepackage[version=3]{mhchem}
% create todo notes to show where what needs adaptation
\usepackage[colorinlistoftodos, textwidth=4cm, shadow]{todonotes}

%--------------bibliograhy stuff--------------
% this package is used to produce both author-date and standard numerical citations for BibTeX bibliographies
\usepackage[round]{natbib}

%--------------Others--------------
% this package allows you to resume a previous enumerate environment
\usepackage{enumitem}
% this package is used to generate lipsum code
\usepackage{lipsum}
% this package is used for if then environments
\usepackage{ifthen}
%All text, within the landscape environment is rotated through 90 degrees:
\usepackage{lscape}
\usepackage{pdflscape} % for more output functionality: rotation etc.
% this package is used to rotate text
\usepackage[figuresright]{rotating}
%\def\changemargin#1{\list{}{\voffset#1}\item[]}
%\let\endchangemargin=\endlist% change page margins on certain pages
\usepackage[squaren,thinqspace,thinspace]{SIunits}
% better unit typesetting

%---------------------------------------------------------------------------
% Some adaptations and settings of the distances in document
%---------------------------------------------------------------------------
% set interline set to ...
\renewcommand{\baselinestretch}{1.2}
% space between two paragraphs should be larger than between two lines of text
\newcommand{\npar} {\par \vspace{2.3ex plus 0.3ex minus 0.3ex}} 
% no indentations
\setlength{\parindent}{0cm}
% makes all pages the height of the text on that page, and no extra vertical space is added
\raggedbottom

%---------------------------------------------------------------------------
% making glossary of abbrevitations and symbols
% ATTENTION: To make the glossaries working, you need to adapt your output
% profiles of the editor you're working in (KIle, Texniccenter)
% See http://biomath.ugent.be/wiki/index.php/Software#Glossaries for
% Explanation.
% If you're not able to make it work, consider using the nomenclature package, also given above
%---------------------------------------------------------------------------

%\makeindex  %index, not done in thesis

% this package is used to create a nomenclature list (alternative to the glossaries package; easier, but less powerfull)
%\usepackage[noprefix]{nomencl}
% this package is used to create a list of symbols (alternative to the glossaries package; easier, but less powerfull)
% \usepackage[final,pageno]{listofsymbols}

%---------------------------------------------------------------------------
%code visualization of code snippets + characteristics; done with listings
%can be used for multiple code languages, but is different from pseudocode
% info http://en.wikibooks.org/wiki/LaTeX/Source_Code_Listings
%---------------------------------------------------------------------------
\definecolor{mygreen}{rgb}{0,0.6,0}

\usepackage{listings}
\lstset{ %
	language=Python,                % choose the language of the code
	basicstyle=\footnotesize\ttfamily,       % the size of the fonts that are used for the code
	numbers=left,                   % where to put the line-numbers
	numberstyle=\scriptsize,      % the size of the fonts that are used for the line-numbers
	stepnumber=1,                   % the step between two line-numbers. If it is 1 each line will be numbered
	numbersep=5pt,                  % how far the line-numbers are from the code
	backgroundcolor=\color{white},  % choose the background color. You must add \usepackage{color}
	showspaces=false,               % show spaces adding particular underscores
	showstringspaces=false,         % underline spaces within strings
	showtabs=false,                 % show tabs within strings adding particular underscores
	frame=none,           % adds a frame around the code
	tabsize=2,          % sets default tabsize to 2 spaces
	captionpos=b,           % sets the caption-position to bottom
	breaklines=true,        % sets automatic line breaking
	breakatwhitespace=false,    % sets if automatic breaks should only happen at whitespace
	xleftmargin=15pt,
	xrightmargin=5pt,
	commentstyle=\color{mygreen}, %teal
	keywordstyle=\color{blue},
	stringstyle=\color{orange},
	escapeinside={\%*}{*}            % if you want to add LaTeX within your code
}

%---------------------------------------------------------------------------
% new command to make superscripts in the text/tables mode
%---------------------------------------------------------------------------
%http://anthony.liekens.net/index.php/LaTeX/SubscriptAndSuperscriptInTextMode
\newcommand{\superscript}[1]{\ensuremath{^{\textrm{#1}}}} 
\newcommand{\subscript}[1]{\ensuremath{_{\textrm{#1}}}}
% new commands to create nicer sub and superscripts in math mode
\newcommand{\sms}[1]{\mbox{\tiny{#1}}}
\newcommand{\lss}[1]{_{\mbox{\tiny{#1}}}}
\newcommand{\uss}[1]{^{\mbox{\tiny{#1}}}}

%---------------------------------------------------------------------------
% Short version commando to introduce figures. 
%---------------------------------------------------------------------------
%\mijnfiguur[H]{width=5cm}{bestandsnaam}{Het bijschrift bij deze figuur}{label}
\newcommand{\mijnfiguur}[5][ht]{            % Het eerste argument is standaar `ht'.
    \begin{figure}[#1]                      % Beginnen van de figure omgeving
        \begin{center}                      % Beginnen van de center omgeving
            \includegraphics[#2]{#3}        % Het eigenlijk invoegen van de figuur (2: opties, 3: bestandsnaam)
        \end{center}
        \caption{#4\label{#5}}          % Het bijschrift (argument 4) en het label (argument 3)
    \end{figure}
    }

    
% --------------------------------------------------------------------------
%  bolt vector notation   
% --------------------------------------------------------------------------
\newcommand{\vecb}[1]{\vec{\mathbf{#1}}}

%---------------------------------------------------------------------------
%Chapter and Parts of the document style changes
%online comment this part out if you will use parts as a structure above chapters
%---------------------------------------------------------------------------
% define color of chapter page
%\definecolor{chaptergray}{rgb}{0.85,0.85,0.85}
% \definecolor{chaptergray}{rgb}{1,1,1}
%
%% define document part label (not always needed to use parts)
%\titleformat{\part}[display]{\flushright\bfseries\vspace{-10cm}} 
%{\usefont{OT1}{pag}{b}{n}\fontsize{50}{54}\selectfont{PART~\thepart}} {0pt} {\vspace{0.5cm}\huge\usefont{OT1}{pag}{m}{n}\fontsize{25}{30}\selectfont\uppercase} [\thispagestyle{empty}\pagecolor{chaptergray}]

% define chapter label
\titleformat{\chapter}[display]{\flushright\vspace{-3cm}\pagecolor{white}}
{\usefont{OT1}{pag}{b}{n}\fontsize{30}{34}\selectfont {CHAPTER~\thechapter}}{10pt}{\usefont{OT1}{pag}{m}{n}\fontsize{22}{24}\selectfont}[{\thispagestyle{empty}\vspace{3cm}}]
% define section label
\titleformat{\section}[hang]{}
{\usefont{OT1}{pag}{b}{n}\fontsize{14}{16}\selectfont\thesection}{16pt}{\usefont{OT1}{pag}{b}{n}\fontsize{14}{16}\selectfont}[{}\vspace{0.25cm}]
%\titleformat{\section}[hang]{}
%{\fontsize{12}{14}\bfseries\selectfont\thesection}{14pt}{\fontsize{12}{14}\bfseries\selectfont}[{}\vspace{0.25cm}]
% define subsection label
\titleformat{\subsection}[hang]{}
{\usefont{OT1}{pag}{b}{n}\selectfont\thesubsection}{13pt}{\usefont{OT1}{pag}{b}{n}\selectfont}[{}\vspace{0.25cm}]
% define subsubsection label
\titleformat{\subsubsection}[hang]{\sffamily}
{\usefont{OT1}{pag}{m}{n}\fontsize{11}{12}\selectfont\thesubsubsection}{13pt}{\usefont{OT1}{pag}{m}{n}\fontsize{11}{12}\selectfont}[{}\vspace{0.05cm}]

% define paragraph label
% \titleformat{\paragraph}[runin]{\sffamily}
% {\usefont{OT1}{pag}{b}{n}\fontsize{12}{13}\selectfont\theparagraph}{}{}[{}]

%---------------------------------------------------------------------------
% define the headers and the footers
%---------------------------------------------------------------------------
\renewcommand{\chaptermark}[1]%
 {\markboth{\MakeUppercase{CHAPTER~\thechapter \mdseries{~~~#1}}}{}}
\renewcommand{\sectionmark}[1]%
 {\markright{\MakeUppercase{\thesection \mdseries{~~~#1}}}}

\renewcommand{\footrulewidth}{0pt}
\newcommand{\helv}{%
\fontfamily{phv}\fontseries{b}\fontsize{8}{10}\selectfont}
%\fancyhf{}
%\fancyhead[LE,RO]{\helv \thepage}
%\fancyhead[LO]{\helv \leftmark}
%\fancyhead[RE]{\helv \rightmark}
%\renewcommand{\headrulewidth}{0.15pt}

%plain-pagestyle is the chapter pages iteself
\fancypagestyle{plain}{%
\fancyhf{} % clear all header and footer fields
\fancyfoot[C]{\helv \thepage} % except the center
\renewcommand{\headrulewidth}{0pt}
\renewcommand{\footrulewidth}{0pt}}

%---------------------------------------------------------------------------
% define new commands, that can later be used in the text
%---------------------------------------------------------------------------
\newcommand{\ismatrix}[1]{\bm{#1}}
\newcommand{\isvector}[1]{\bm{#1}}
\newcommand{\isstate}[1]{\mathrm{#1}}
\newcommand{\isunit}[1]{\,\mathrm{#1}}

\DeclareMathOperator*{\E}{E}

%---------------------------------------------------------------------------
% Some extra ticks and tricks
%---------------------------------------------------------------------------
% define tocdepth
\setcounter{tocdepth}{2}
% set lenght of headheight
\setlength{\headheight}{21pt}
% to suppress widows and orphans
\widowpenalty=10000
\clubpenalty=10000

%---------------------------------------------------------------------------
% Figure placement
%---------------------------------------------------------------------------
\graphicspath{{./figs/}}

%---------------------------------------------------------------------------
% PDF properties (you only see this as document properties (metadata) of your pdf-document
%---------------------------------------------------------------------------
\hypersetup{
    pdfauthor = {Your name},
    pdftitle = {Your title},
    pdfkeywords = {some relevant keywords}}

% Personal commands
\newcommand{\rotlabel}[3]{\raisebox{#1}{\rotatebox[]{#2}{#3}}}


% ------------ glossaries (symbols and abbreviations) ------------
% ftp://ftp.dante.de/tex-archive/macros/latex/exptl/siunitx/siunitx.pdf voor de SIunitx package

\usepackage[toc,acronym,shortcuts,nonumberlist,nopostdot]{glossaries}

\setlength{\glsdescwidth}{\textwidth}
\newglossary[slg]{symbolslist}{syi}{syg}{Symbolslist}

%\glssetnoexpandfield{unit}
\glsaddkey{unit}{\glsentrytext{\glslabel}}{\glsentryunit}{\GLsentryunit}{\glsunit}{\Glsunit}{\GLSunit}

\makeglossaries
\makeindex
\newglossarystyle{symbunitlong}{%
\setglossarystyle{long3col}% base this style on the list style
\renewenvironment{theglossary}{% Change the table type --> 3 columns
\begin{longtable}{lp{0.6\glsdescwidth}>{\centering\arraybackslash}p{2cm}}}%
  {\end{longtable}}%
%
\renewcommand*{\glossaryheader}{%  Change the table header
  \bfseries Symbol & \bfseries Description & \bfseries Unit \\
  \hline
  \endhead}
\renewcommand*{\glossentry}[2]{%  Change the displayed items
\glstarget{##1}{\glossentryname{##1}} %
& \glossentrydesc{##1}% Description
& \glsunit{##1}  \tabularnewline
}
} 


%---------------------------------------------------------------------------
%---------------------------------------------------------------------------
% The actual document starts here
%---------------------------------------------------------------------------
%---------------------------------------------------------------------------
% This can be done in the text itself, but easier to bring them all together here
% proper

\newglossaryentry{J}{name={\ensuremath{J}},description={flux},sort=J, unit={[\unit{\metre\cubic\,\metre\rpsquare\,\second\reciprocal}]}, type=symbolslist}
\newglossaryentry{fluidDyn}{name={\ensuremath{\mu_f}},description={fluid dynamic viscosity},sort=muf, unit={[\unit{\kilogram\,\metre\reciprocal\,\second\reciprocal}]}, type=symbolslist}
\newglossaryentry{fluidKin}{name={\ensuremath{\nu_f}},description={fluid kinematic viscosity},sort=nuf, unit={[\unit{\metre\squared\,\second\reciprocal}]}, type=symbolslist}
\newglossaryentry{R}{name={\ensuremath{R}},description={membrane resistance},sort=R, unit={[\unit{\metre\reciprocal}]}, type=symbolslist}
\newglossaryentry{Rc}{name={\ensuremath{R_\mathrm{c}}},description={filter cake resistance},sort=Rc, unit={[\unit{\metre\reciprocal}]}, type=symbolslist}
\newglossaryentry{Rm}{name={\ensuremath{R_\mathrm{m}}},description={clean membrane resistance},sort=Rcb, unit={[\unit{\metre\reciprocal}]}, type=symbolslist}
\newglossaryentry{Rt}{name={\ensuremath{R_\mathrm{t}}},description={total hydraulic resistance},sort=Rcc, unit=[\unit{\metre\reciprocal}], type=symbolslist}
\newglossaryentry{Rf}{name={\ensuremath{R_\mathrm{f}}},description={pore blocking and inner membrane fouling resistance},sort=Rca, unit={[\unit{\metre\reciprocal}]}, type=symbolslist}
\newglossaryentry{Rb}{name={\ensuremath{R_\mathrm{b}}},description={pore blocking resistance},sort=Rb, unit={[\unit{\metre\reciprocal}]}, type=symbolslist}
\newglossaryentry{dp}{name={\ensuremath{\mathrm{\Delta} p}},description={pressure drop},sort=dp, unit={[\unit{\pascal}]}, type=symbolslist}
\newglossaryentry{cM}{name={\ensuremath{c_\mathrm{m}}},description={concentration on membrane surface},sort=cm, unit={[\unit{\kilogram\,\metre\rpcubic}]}, type=symbolslist}
\newglossaryentry{cb}{name={\ensuremath{c_\mathrm{b}}},description={bulk concentration},sort=cb, unit={[\unit{\kilogram\,\metre\rpcubic}]}, type=symbolslist}
\newglossaryentry{kp}{name={\ensuremath{k_\mathrm{p}}},description={mass transfer coefficient},sort=kzp, unit={[\unit{\metre\cubic\,\metre\rpsquared\,\second\reciprocal}]}, type=symbolslist}
\newglossaryentry{kf}{name={\ensuremath{k_\mathrm{f}}},description={model parameter fouling}, sort=kzf, unit={[\,-\,]}, type=symbolslist}
\newglossaryentry{kc}{name={\ensuremath{k_\mathrm{c}}},description={model parameter cake layer},sort=kzc, unit={[\unit{\metre\squared\,\kilogram\reciprocal}]}, type=symbolslist}
\newglossaryentry{thetaC}{name={\ensuremath{\theta_c}},description={critical friction angle}, sort=tzhetaC, unit={[\,-\,]}, type=symbolslist}
\newglossaryentry{theta}{name={\ensuremath{\theta}},description={angle of friction}, sort=tzheta, unit={[\,-\,]}, type=symbolslist}
\newglossaryentry{Fi}{name={\ensuremath{F_\mathrm{i}}},description={interparticle force}, sort=Fi, unit={[\unit{\newton}]}, type=symbolslist}
\newglossaryentry{Fg}{name={\ensuremath{F_\mathrm{g}}},description={gravity force}, sort=Fg, unit={[\unit{\newton}]}, type=symbolslist}
\newglossaryentry{Fl}{name={\ensuremath{F_\mathrm{l}}},description={lift force}, sort=Fl, unit={[\unit{\newton}]}, type=symbolslist}
\newglossaryentry{Fd}{name={\ensuremath{F_\mathrm{d}}},description={drag force}, sort=Fd, unit={[\unit{\newton}]}, type=symbolslist}
\newglossaryentry{fc}{name={\ensuremath{f_\mathrm{c}}},description={friction factor}, sort=fZN, unit={ [\,-\,]}, type=symbolslist}
\newglossaryentry{f}{name={\ensuremath{f}},description={external body forces}, sort=fZ, unit={[\unit{\newton\,\metre\reciprocal}]}, type=symbolslist}
\newglossaryentry{phis}{name={\ensuremath{\phi_\mathrm{s}}},description={shape factor}, sort=phis, unit={[\,-\,]}, type=symbolslist}
\newglossaryentry{eps}{name={\ensuremath{\epsilon}}, description={actual filter cake porosity}, sort=eps, unit={[\,-\,]}, type=symbolslist}
\newglossaryentry{epssph}{name={\ensuremath{\epsilon_\mathrm{sph}}}, description={filter cake porosity for perfect spheres}, sort=epssph, unit={[\,-\,]}, type=symbolslist}
\newglossaryentry{dl}{name={\ensuremath{l}},description={cake thickness}, sort=l, unit={[\unit{\metre}]}, type=symbolslist}
\newglossaryentry{S}{name={\ensuremath{S}},description={specific surface area}, sort=S, unit={[\unit{\metre\squared}]}, type=symbolslist}
\newglossaryentry{Si}{name={\ensuremath{S_\mathrm{i}}},description={specific surface area of the particles}, sort=Si, unit={[\unit{\metre\squared}]}, type=symbolslist}
\newglossaryentry{Sf}{name={\ensuremath{S_\mathrm{f}}},description={model parameter fouling saturation},sort=sf, unit={[\,-\,]}, type=symbolslist}
\newglossaryentry{Am}{name={\ensuremath{A_\mathrm{m}}},description={membrane area},sort=Am, unit={[\unit{\metre\squared}]}, type=symbolslist}
\newglossaryentry{K}{name={\ensuremath{K}},description={Kozeny constant}, sort=K, unit={[\,-\,]}, type=symbolslist}
\newglossaryentry{k}{name={\ensuremath{k}},description={adhesion parameter}, sort=kz, unit={[ \unit{\second\,\metre\reciprocal}]}, type=symbolslist}
\newglossaryentry{kappa}{name={\ensuremath{\kappa}},description={fluid velocity gradient}, sort=kz, unit={[ \unit{\second\reciprocal}]}, type=symbolslist}
\newglossaryentry{L}{name={\ensuremath{L}},description={characteristic length}, sort=L, unit={[\unit{\metre}]}, type=symbolslist}
\newglossaryentry{alpha}{name={\ensuremath{\alpha}},description={specific filtration resistance}, sort=azlpha, unit={[\unit{\metre\,\kilogram\reciprocal}]}, type=symbolslist}
\newglossaryentry{rhoS}{name={\ensuremath{\rho_\mathrm{s}}},description={density of the bulk particles}, sort=rzhos, unit={[\unit{\kilogram\,\metre\rpcubic}]}, type=symbolslist}
\newglossaryentry{rhof}{name={\ensuremath{\rho_\mathrm{f}}},description={density of the continuous phase}, sort=rzhof, unit={[\unit{\kilogram\,\metre\rpcubic}]}, type=symbolslist}
\newglossaryentry{vl}{name={\ensuremath{v_\mathrm{l}}},description={inertia lifting velocity}, sort=vl, unit={[\unit{\metre\,\second\reciprocal}]}, type=symbolslist}
\newglossaryentry{vi}{name={\ensuremath{v_\mathrm{i}}},description={particle interaction velocity}, sort=vi, unit={[\unit{\metre\,\second\reciprocal}]}, type=symbolslist}
\newglossaryentry{v}{name={\ensuremath{v}},description={velocity}, sort=vb, unit={[\unit{\metre\,\second\reciprocal}]}, type=symbolslist}
\newglossaryentry{vg}{name={\ensuremath{v_\mathrm{g}}},description={gravitational sedimentation velocity}, sort=vg, unit={[\unit{\metre\,\second\reciprocal}]}, type=symbolslist}
\newglossaryentry{vd}{name={\ensuremath{v_\mathrm{d}}},description={diffusion velocity}, sort=vd, unit={[\unit{\metre\,\second\reciprocal}]}, type=symbolslist}
\newglossaryentry{vs}{name={\ensuremath{v_\mathrm{s}}},description={shear induced diffusion velocity}, sort=vs, unit={[\unit{\metre\,\second\reciprocal}]}, type=symbolslist}
\newglossaryentry{vtot}{name={\ensuremath{v_\mathrm{tot}}},description={total backtransport velocity}, sort=vtot, unit={[\unit{\metre\,\second\reciprocal}]}, type=symbolslist}
\newglossaryentry{partDia}{name={\ensuremath{d_\mathrm{p}}},description={particle diameter}, sort=dhap, unit={[\unit{\metre}]}, type=symbolslist}
\newglossaryentry{Fnum}{name={\ensuremath{F_{\mathrm{num}}}},description={particle size distribution}, sort=Fnum, unit={[\,-\,]}, type=symbolslist}
\newglossaryentry{N0}{name={\ensuremath{N_\mathrm{0}}},description={initial number of membrane pores}, sort=N0, unit={[\,-\,]}, type=symbolslist}
\newglossaryentry{N}{name={\ensuremath{N}},description={number of free pores}, sort=N, unit={[\,-\,]}, type=symbolslist}
\newglossaryentry{dpteff}{name={\ensuremath{\mathrm{\Delta} p_{\mathrm{t,eff}}}},description={the effective transmembrane pressure acting on the membrane surface},sort=dpteff, unit={[\unit{\pascal}]}, type=symbolslist}
\newglossaryentry{dpc}{name={\ensuremath{\mathrm{\Delta} p_{\mathrm{c}}}},description={pressure drop over cake layer},sort=dpc, unit={[\unit{\pascal}]}, type=symbolslist}
\newglossaryentry{di}{name={\ensuremath{\Delta i}},description={thickness of a filter cake slice}, sort=di, unit={[\unit{\metre}]}, type=symbolslist}
\newglossaryentry{E}{name={\ensuremath{E}},description={probability of deposition}, sort=E, unit={[\,-\,]}, type=symbolslist}
\newglossaryentry{Kd}{name={\ensuremath{K_\mathrm{d}}},description={rate coefficient of sludge detachment}, sort=Kd, unit={[\unit{\second\reciprocal}]}, type=symbolslist}
\newglossaryentry{Msf}{name={\ensuremath{M_\mathrm{sf}}},description={mass of sludge in the filter cake}, sort=Msf, unit={[\unit{\kilogram\,\metre\rpsquare}]}, type=symbolslist}
\newglossaryentry{Rpor}{name={\ensuremath{R_\mathrm{pore}}},description={retained weight fraction distribution of the membrane pore size distribution}, sort=Rpor, unit={[\,-\,]}, type=symbolslist}
\newglossaryentry{gbulk}{name={\ensuremath{g_\mathrm{b}}},description={particle size distribution of the bulk particles}, sort=gb, unit={[\,-\,]}, type=symbolslist}
\newglossaryentry{gcake}{name={\ensuremath{g_\mathrm{cake}}},description={particle size distribution of the particles retained by the membrane}, sort=gcake, unit={[\,-\,]}, type=symbolslist}
\newglossaryentry{gmem}{name={\ensuremath{g_\mathrm{m}}},description={particle size distribution of the particles able to enter the pores}, sort=gm, unit={[\,-\,]}, type=symbolslist}
\newglossaryentry{Vm}{name={\ensuremath{V_\mathrm{m}}}, description={total membrane volume}, sort=Vam, unit={[\unit{\metre\cubic}]}, type=symbolslist}

\newglossaryentry{Vp}{name={\ensuremath{V_\mathrm{p}}}, description={particle volume}, sort=Vaa, unit={[\unit{\metre\cubic}]}, type=symbolslist}

\newglossaryentry{rhopm}{name={\ensuremath{\rho_\mathrm{p,m}}},description={the density of particles in the membrane pores}, sort=rzhopm, unit={[kg\,m$^\mathrm{-3}$]}, type=symbolslist}
\newglossaryentry{cbm}{name={\ensuremath{c_\mathrm{m}^\mathrm{b}}}, description={the mass concentration of particles able to penetrate the membrane}, sort=cbm, unit={[kg\,m$^\mathrm{-3}$]}, type=symbolslist}
\newglossaryentry{Kp}{name={\ensuremath{K_\mathrm{p}}},description={membrane specific constant}, sort=Kp, unit={[m$^\mathrm{-1}$]}, type=symbolslist}
\newglossaryentry{Kc}{name={\ensuremath{K_\mathrm{c}}},description={specific cake resistance}, sort=Kc, unit={ [m$^\mathrm{-2}$]}, type=symbolslist}
\newglossaryentry{cc}{name={\ensuremath{c_\mathrm{c}}},description={mass concentration of the cake layer}, sort=cc, unit={[\unit{\kilogram\,\metre\rpcubic}]}, type=symbolslist}
% \newglossaryentry{C}{name={\ensuremath{C}},description={concentration of suspended solids}, sort=C, unit={ [\unit{\kilogram\,\metre\rpcubic}]}, type=symbolslist}
\newglossaryentry{Dh}{name={\ensuremath{D_\mathrm{h}}},description={hydraulic diameter}, sort=Dh, unit={[\unit{\metre}]}, type=symbolslist}
\newglossaryentry{dpHead}{name={\ensuremath{\bar{d}_\mathrm{p}}},description={mean diameter of bulk particles}, sort=dhapHead, unit={[\unit{\metre}]}, type=symbolslist}
\newglossaryentry{Ft}{name={\ensuremath{F_\mathrm{t}}},description={tangential shear stress}, sort=Ft, unit={[\unit{\newton}]}, type=symbolslist}
\newglossaryentry{Ftau}{name={\ensuremath{F_\mathrm{\tau}}},description={friction force}, sort=Ftau, unit={[\unit{\newton}]}, type=symbolslist}
\newglossaryentry{FN}{name={\ensuremath{F_\mathrm{N}}},description={normal force}, sort=FN, unit={[\unit{\newton}]}, type=symbolslist}
\newglossaryentry{FA}{name={\ensuremath{F_\mathrm{A}}},description={adhesion force}, sort=FabA, unit={[\unit{\newton}]}, type=symbolslist}
\newglossaryentry{omega}{name={\ensuremath{\omega}},description={fraction of retained particles that attach to the cake layer}, sort=omega, unit={[\,-\,]}, type=symbolslist}
\newglossaryentry{Omega}{name={\ensuremath{\Omega}},description={volume of fluid element}, sort=omega, unit={[\unit{\metre\cubic}]}, type=symbolslist}
\newglossaryentry{cbc}{name={\ensuremath{c^\mathrm{b}_\mathrm{c}}},description={bulk concentration of particles that are retained on the membrane surface}, sort=cbc, unit={[\unit{\kilogram\,\metre\rpcubic}]}, type=symbolslist}
\newglossaryentry{tf}{name={\ensuremath{t_\mathrm{f}}},description={filtration time/cycle}, sort=tf, unit={[\unit{\second}]}, type=symbolslist}
\newglossaryentry{pf}{name={\ensuremath{p_\mathrm{f}}},description={total fraction of depositable particles}, sort=pf, unit={[\,-\,]}, type=symbolslist}
\newglossaryentry{psi}{name={\ensuremath{\psi}},description={relative kinematic pressure}, sort=psi, unit={[\unit{\metre\squared\,\second\rpsquared}]}, type=symbolslist}
\newglossaryentry{Rep}{name={\ensuremath{Re_\mathrm{p}}},description={particle Reynolds number}, sort=Rep, unit={[\,-\,]}, type=symbolslist}
\newglossaryentry{Ur}{name={\ensuremath{\mathbf{U}_\mathrm{r}}},description={relative particle velocity}, sort=Ur, unit={[\unit{\metre\,\second\reciprocal}]}, type=symbolslist}
\newglossaryentry{Ueff}{name={\ensuremath{\mathbf{U}_\mathrm{r,eff}}},description={Fax\'en corrected relative velocity}, sort=Ureff, unit={[\unit{\metre\,\second\reciprocal}]}, type=symbolslist}
\newglossaryentry{Up}{name={\ensuremath{\mathbf{U}_\mathrm{p}}},description={particle velocity}, sort=Up, unit={[\unit{\metre\,\second\reciprocal}]}, type=symbolslist}
\newglossaryentry{Uc}{name={\ensuremath{\mathbf{U}_\mathrm{c}}},description={fluid velocity}, sort=Uc, unit={[\unit{\metre\,\second\reciprocal}]}, type=symbolslist}
\newglossaryentry{mp}{name={\ensuremath{m_\mathrm{p}}},description={particle mass}, sort=mzp, unit={[\unit{\kilogram}]}, type=symbolslist}
%\newglossaryentry{p}{name={\ensuremath{p}},description={pressure [\unit{\pascal]}}, sort=p}
\newglossaryentry{tau}{name={\ensuremath{\mathbf{\tau}}},description={viscous stress tensor}, sort=txaau,unit={[\unit{\newton}]} , type=symbolslist}
\newglossaryentry{Fbody}{name={\ensuremath{\vecb{F}_\mathrm{body}}},description={body forces}, sort=faabody, unit={[\unit{\newton}]}, type=symbolslist}
\newglossaryentry{Fsurf}{name={\ensuremath{\vecb{F}_\mathrm{surf}}},description={surface forces}, sort=faasurf, unit={[\unit{\newton}]}, type=symbolslist}
\newglossaryentry{Farch}{name={\ensuremath{\vecb{F}_\mathrm{arch}}},description={Archimedes force }, sort=faaarch, unit={[\unit{\newton}]}, type=symbolslist}
\newglossaryentry{Fp}{name={\ensuremath{\vecb{F}_\mathrm{p}}},description={pressure gradient induced force}, sort=Faap, unit={[\unit{\newton}]}, type=symbolslist}
\newglossaryentry{Fhydr}{name={\ensuremath{\vecb{F}_\mathrm{hydr}}},description={resulting hydrodynamic force}, sort=Faahydr, unit={[\unit{\newton}]}, type=symbolslist}
\newglossaryentry{Fdrag}{name={\ensuremath{\vecb{F}_\mathrm{drag}}},description={drag force}, sort=Faadrag, unit={[\unit{\newton}]}, type=symbolslist}
\newglossaryentry{Fam}{name={\ensuremath{\vecb{F}_\mathrm{am}}},description={added mass force}, sort=Faaam, unit={[\unit{\newton}]}, type=symbolslist}
\newglossaryentry{Fhist}{name={\ensuremath{\vecb{F}_\mathrm{hist}}},description={history force}, sort=Faahist, unit={[\unit{\newton}]}, type=symbolslist}
\newglossaryentry{Fgrav}{name={\ensuremath{\vecb{F}_\mathrm{g}}},description={gravitational force}, sort=Faag, unit={[\unit{\newton}]}, type=symbolslist}
\newglossaryentry{nabP}{name={\ensuremath{\nabla\, p}},description={pressure gradient}, sort=nzabla, unit={[\unit{\pascal\,\metre\reciprocal}]}, type=symbolslist}
\newglossaryentry{Flift}{name={\ensuremath{\vecb{F}_\mathrm{lift}}},description={lift force}, sort=Faalift, unit={[\unit{\newton}]}, type=symbolslist}
\newglossaryentry{PUp}{name={\ensuremath{P(\mathbf{U}_\mathrm{p})}},description={adhesion probability}, sort=Paup, unit={[\,-\,]}, type=symbolslist}
\newglossaryentry{Ucf}{name={\ensuremath{\mathbf{U}_\mathrm{cf}}},description={cross-flow velocity}, sort=Ucf, unit={ [\unit{\metre\,\second\reciprocal}]}, type=symbolslist}
\newglossaryentry{A}{name={\ensuremath{A}},description={inlet area},sort=A, unit={[\unit{\metre\squared}]},	 type=symbolslist}
\newglossaryentry{deltat}{name={\ensuremath{\Delta t}},description={time step},sort=dt, unit={[\unit{\second}]}, type=symbolslist}
\newglossaryentry{M}{name={\ensuremath{M}},description={mass per membrane surface area}, sort=M, unit={[\unit{\kilogram\,\metre\rpsquared}]}, type=symbolslist}
\newglossaryentry{m}{name={\ensuremath{m}},description={mass}, sort=mza, unit={[\unit{\kilogram}]}, type=symbolslist}
\newglossaryentry{Mtot}{name={\ensuremath{m_\mathrm{tot}}},description={total mass inflow of suspended solids during the entire simulation}, sort=mztot, unit={[\unit{\kilogram}]}, type=symbolslist}
\newglossaryentry{ttot}{name={\ensuremath{t_\mathrm{tot}}},description={total simulation time}, sort=ttot, unit={[\unit{\second}]}, type=symbolslist}
\newglossaryentry{tauW}{name={\ensuremath{\tau_\mathrm{W}}},description={shear stress}, sort=txauW, unit={[\unit{\pascal}]}, type=symbolslist}
\newglossaryentry{muMax}{name={\ensuremath{\mu_\mathrm{max}}},description={maximum friction}, sort=muMax, unit={[\,-\,]}, type=symbolslist}
\newglossaryentry{Re}{name={\ensuremath{Re}},description={Reynolds number}, sort=Re, unit={[\,-\,]}, type=symbolslist}
\newglossaryentry{t}{name={\ensuremath{t}},description={time}, sort=t, unit={[\unit{\second}]}, type=symbolslist}
%improper
\newglossaryentry{epsm}{name={\ensuremath{\epsilon_\mathrm{m}}}, description={membrane porosity}, sort=epsm, unit={[\,-\,]}, type=symbolslist}
\newglossaryentry{Vpm}{name={\ensuremath{V_\mathrm{p,m}}},description={volume of particles that sediment each filtration cycle}, sort=Vampm, unit={[\unit{\metre\cubic}]}, type=symbolslist}
\newglossaryentry{Um}{name={\ensuremath{\mathbf{U}_\mathrm{m}}},description={maximum channel velocity}, sort=Um, unit={[\unit{\metre\,\second\reciprocal}]}, type=symbolslist}
  	%made on the fly manually
% This can be done in the text itself, but easier to bring them all together
\newacronym{API}{API}{application programming interface}
\newacronym{CAD}{CAD}{computer-aided design}
\newacronym{DLVO}{DLVO}{Derjaguin, Landau, Verwey and Overbeek}
\newacronym{RIS}{RIS}{resistance-in-series}
\newacronym{TMP}{TMP}{transmembrane pressure}
\newacronym{MBR}{MBR}{membrane bioreactor}
\newacronym{PSD}{PSD}{particle size distribution}
\newacronym{SS}{SS}{suspended solids}
\newacronym{CFD}{CFD}{computational fluid dynamics}
\newacronym{AFM}{AFM}{atomic force microscopy}
% \newacronym{3D}{3D}{three dimensions}
% \newacronym{2D}{2D}{two dimensions}
\newacronym{PDE}{PDE}{partial differential equation}
\newacronym{FVM}{FVM}{finite volume method}
\newacronym{FEM}{FEM}{finite element method}
% \newacronym{EPS}{EPS}{extracellular polymeric substance}
\newacronym{ABM}{ABM}{agent-based model}
\newacronym{GUI}{GUI}{graphical user interface}

\newacronym{CDF}{CDF}{cumulative distribution function} %continuous
\newacronym{PMF}{PMF}{probability mass function} %discrete
\newacronym{SCR}{SCR}{simulation time to computational time ratio} %discrete

% profilometry
\newacronym{SP}{SP}{stylus profilometry}
\newacronym{WLI}{WLI}{white light interferometry}
\newacronym{DHM}{DHM}{digital holographic microscopy}
\newacronym{OCT}{OCT}{optical coherence tomography}
%\newacronym{FVMi}{FVMi}{focus variation microscopy}
%\newacronym{IDM}{IDM}{intensity detection microscopy}
%\newacronym{CDM}{CDM}{contrast detection microscopy}
\newacronym{FPP}{FPP}{fringe projection profilometry}
\newacronym{FP}{FP}{Fourier profilometry}
\newacronym{MP}{MP}{Moir\'e profilometry}
\newacronym{WLAC}{WLAC}{white light axial chromatism}
\newacronym{SEM}{SEM}{scanning electron microscopy}
\newacronym{STM}{STM}{scanning tunneling microscopy}
\newacronym{DICM}{DICM}{differential interference contrast microscopy}
\newacronym{CLSM}{CLSM}{confocal laser scanning microscopy}
   %made on the fly manually

\begin{document}
 %print all symbols even if they are not called in the text
%REMARK: The overload of empty pages in the beginning is since the book
%style is making sure you are always starting a new chapter/part/... on a right
% page of the book. This is automatically done.

% ------------ titlepage ------------
%\input{titlepage_phd.tex} 
%\begin{titlepage}
\begin{titlepage}

\noindent
\vspace{-3cm}

\selectfont
\sffamily


\begin{center}

\includegraphics[height=3.5cm]{UGentlogo}
\vspace{0.75cm}

\LARGE Faculteit Bio-ingenieurswetenschappen\\
\vspace{0.25cm}
Academiejaar 2015-2016\\
\vspace{4.5cm}
\huge \textbf{Spatio-temporal modelling of filter cake formation in filtration processes}
\vspace{3.5cm}
\end{center}
\bfseries
\large Bram De Jaegher\\
\mdseries
Promotor: Prof.~dr.~ir. Ingmar Nopens \& dr.~ir. Jan Baetens\\
Tutor: ir. Wouter Naessens\\ 
\vspace{2.5cm}
\begin{center}
Masterproef voorgedragen tot het behalen van de graad van\\
Master in de bio-ingenieurswetenschappen: Chemie en bioprocestechnologie
\vspace{0.75cm}
\end{center}
\thispagestyle{empty}

\end{titlepage}
%\end{titlepage}

% Typisch copyright voor een thesis.
% Te plaatsen juist na het titelblad.

\rule[-0.4\baselineskip]{0cm}{14\baselineskip}   
\par \vspace{2.9ex plus 0.3ex minus 0.3ex}
De auteur en promotor geven de toelating deze scriptie voor consultatie beschikbaar te stellen en delen ervan te kopi\"eren voor persoonlijk gebruik. Elk ander gebruik valt onder de beperkingen van het auteursrecht, in het bijzonder met betrekking tot de verplichting uitdrukkelijk de bron te vermelden bij het aanhalen van resultaten uit deze scriptie.
\par \vspace{2.3ex plus 0.3ex minus 0.3ex}
The author and promoter give the permission to use this thesis for consultation and to copy parts of it for personal use. Every other use is subject to the copyright laws, more specifically the source must be extensively specified when using results from this thesis.
\par \vspace{2.3ex plus 0.3ex minus 0.3ex}
Ghent, June 3, 2016 % Vul de juiste datum in!!!
\par \vspace{2.3ex plus 0.3ex minus 0.3ex}

The promoters, 
\npar
\vspace{2cm}
\npar
% Pas de volgende lijn aan!!!
Prof. dr. ir. Ingmar Nopens \hfill dr. ir. Jan Baetens 
\npar
\vspace{0cm}
\npar
The tutor, \hfill The author,
\npar
\vspace{2cm}
\npar
% Pas de volgende lijn aan!!!
ir. Wouter Naessens \hfill Bram De Jaegher

\thispagestyle{empty} 


\cleardoublepage

\frontmatter

%=======================================
%We want specific page numbering style here
\fancyhf{}
\renewcommand{\headrulewidth}{0pt}
\fancyfoot[C]{\helv \thepage}                  % Paginanummers onderaan gecentreerd.
%=======================================

%intermezzo about making titles conform
\titleformat{\chapter}[display]{\flushright\vspace{-3cm}\pagecolor{white}}
{\usefont{OT1}{pag}{b}{n}\fontsize{30}{34}\selectfont {CHAPTER~\thechapter}}{10pt}{\usefont{OT1}{pag}{b}{n}\fontsize{30}{34}\selectfont}[{\thispagestyle{plain}\vspace{3cm}}]

% ------------ thanks! ------------
\chapter{Acknowledgement}
It all started with a bright summer's day in August when I timidly entered the simulation lab of the notorious department of Mathematical Modelling, Statistics and Bioinformatics. Now, ten months later, I can proudly present this master thesis. Despite the vast amount of work and the various complaints of my computer, I can confidently say that I consider this year as a very positive and enriching period of my student career. However, the result of this master dissertation would not have been the same without a few people and therefore I would like to pay them my respects. \\ \\
First of all, thank you Ingmar Nopens and Jan Baetens for your supervision and thorough corrections. Second of all, I would like to thank Wouter Naessens for being an awesome supervisor and providing guidance throughout this thesis. I am truly sorry for the hour-long, brain frying discussions on the force balance and the heavy workload I put on your shoulders during the first and last weeks. Michael Ghijs, for showing me the relativeness of deadlines and the relevance of the Beatles in data storage. May the lift force be with you! \\ \\
On a more serious note, thank you Timothy Van Daele for helping me with all my OpenFOAM, LaTeX and Github questions and good luck with the finalisation of your PhD. Stijn Van Hoey, for aiding me with the MATLAB, Github and Linux related questions. Next, a special notification for my fellow simulation lab thesis friends; Arthur, Laurentijn, Sofie, Annelies and Bavo for the occasional lighthearted chat and all members BIOMATH and KERMIT for the friendly and welcoming 
environment. Muchas gracias Jos\'e por los d\'atiles. \\ \\
On a more personal note, a special thanks to my girlfriend, Nancy, for the corrections, support and for being an amazing person. My parents, for all their unconditional support. Finally, I would like to thank \emph{``Wa boel \'e da ier?''} for all the great experiences during these five years at 't Boerekot and I wish you all the best of luck.

%giacomo
 
% ------------ summary ------------
\chapter[Summary]{Summary}
\vspace{-2cm}
At present time, membrane filtration processes suffer from a high operational cost due to fouling abatement measures. The description of fouling mechanisms is still highly empirical and does not provide an adequate framework for the development of decision support tools to aid a cost-effective operation of these processes. The objective of this master dissertation is the development and evaluation of a spatio-temporal model to describe particle behaviour in a realistic and accurate manner in order to unravel the mechanisms of filter cake formation. This model aims to pinpoint the most influential processes that should be included in the next generation of decision support tools for filtration processes. \par In the context of this dissertation, a literature review was performed to identify the current fouling modelling approaches and to provide a overview of the available profilometric techniques for the calibration and validation of the model under development. The filter cake formation model of \cite{Ghijs2014} was further extended to a three-dimensional model, the representation of the feed flow was extended to multidisperse suspensions, a new and highly efficient collision detection algorithm was implemented and a user-friendly graphical user interface was developed for the analysis of the simulation results. \par 
A laboratory scale microfiltration device was designed with computational fluid dynamics for the calibration/validation of the model and a qualitative validation, based on the Segr\'e-Silberberg effect, was performed. The latter indicated some imperfections in the force balance on the flowing particles. A scenario analysis was carried out to assess the behaviour of the model under various operational conditions. \par
The simulations and qualitative validation led to a clear understanding of some of the modelling deficits and shortcomings, such as the formation of narrow filter cake patches and the absence of appropriate wall repulsion effects. Still, with an eye on the future development of the model, guidelines are provided to resolve these issues. \par
In the end, it can be concluded that a lot of progression was made towards a realistic representation of filter cake formation processes. Furthermore, a lot of knowledge was gained concerning their underlying mechanisms which was, after all, the main objective of this master thesis.
\chapter[Samenvatting]%
{Samenvatting}
\hyphenation{
be-schrijft
wij-ten
ge-ne-ra-tie
si-mu-la-tie-re-sul-ta-ten
ont-wikkeling
be-schrij-ving
}
\vspace{-2cm}
Momenteel hebben membraanfiltratieprocessen een grote operationele kost die vooral te wijten is aan de genomen maatregelen tegen de vervuiling van het membraan. De beschrijving van deze vervuilingsmechanismen is nog steeds zeer empirisch en biedt geen goed kader voor de ontwikkeling van beslissingsondersteunende systemen. Deze systemen kunnen onder \mbox{andere} de kosteneffectiviteit van deze processen verbeteren. Het doel van deze thesis is de ontwikkeling en evaluatie van een spatio-temporeel model dat het gedrag van gesuspendeerde partikels in een filtratiesysteem op een realistische en accurate manier kan beschrijven. Dit model kan waardevolle inzichten leveren in de belangrijkste onderliggende mechanismen van filterkoekvorming. Deze mechanismen kunnen dan opgenomen worden in de volgende generatie aan beslissingsondersteunende systemen. \par
In het kader van dit proefschrift werd een literatuuronderzoek uitgevoerd dat een overzicht biedt van de huidige membraanvervuilingsmodellen. Tevens wordt een overzicht gegeven van de beschikbare profilometrische technieken voor de kalibratie en validatie van het model in ontwikkeling. Het spatio-temporeel model van \cite{Ghijs2014}, dat de filterkoekvorming beschrijft in membraanbioreactoren, werd verder uitgebouwd naar een driedimensionaal model. Een polydisperse voorstelling van de disperse fase werd bewerkstelligd en een nieuw en \mbox{effici\"ent} collisiedetectie-algoritme werd ge\"implementeerd. Voor de analyse van de simulatieresultaten werd een gebruiksvriendelijke grafische gebruiksomgeving ontwikkeld. \par
Een microfiltratie pilootopstelling  werd ontwikkeld via numerieke stromingsleer voor de kalibratie/validatie van het model en een kwalitatieve validatie werd uitgevoerd op basis van het Segr\'e-Silberberg effect. Dit laatste toonde een aantal onvolmaaktheden aan in de krachtenbalans over de gesuspendeerde partikels. Vervolgens werd een scenarioanalyse uitgevoerd om het gedrag van het model onder verschillende operationele condities te evalueren. \par
De simulatieresultaten en kwalitatieve validatie hebben geleid tot een duidelijk inzicht in enkele tekortkomingen van het model zoals de vorming van smalle filterkoektorens en de afwezigheid van de gepaste wandrepulsie-effecten. Met het oog op de toekomstige ontwikkeling van het model werden een aantal richtlijnen verschaft om deze kwesties aan te pakken. Er kan besloten worden dat er veel vooruitgang geboekt is naar een realistische voorstelling van filterkoekvormende processen, waarbij veel inzicht verworven is in de onderliggende mechanismen; het uiteindelijke hoofddoel van deze dissertatie.





 
% ------------ table of contents ------------
%intermezzo about making titles conform
\titleformat{\chapter}[display]{\flushright\vspace{-3cm}\pagecolor{white}}
{\usefont{OT1}{pag}{b}{n}\fontsize{30}{34}\selectfont {CHAPTER~\thechapter}}{10pt}{\usefont{OT1}{pag}{b}{n}\fontsize{30}{34}\selectfont}[{\thispagestyle{plain}\vspace{2cm}}]

\tableofcontents
\addcontentsline{toc}{chapter}{Contents} 

%intermezzo about making titles conform
\titleformat{\chapter}[display]{\flushright\vspace{-3cm}\pagecolor{white}}
{\usefont{OT1}{pag}{b}{n}\fontsize{30}{34}\selectfont {CHAPTER~\thechapter}}{10pt}{\usefont{OT1}{pag}{b}{n}\fontsize{30}{34}\selectfont}[{\thispagestyle{plain}\vspace{3cm}}]
%\newpage
%\thispagestyle{empty}
\cleardoublepage % makes sure the next part starts at the left again

% ------------ glossaries --------------
\setglossarystyle{super} %list, listgroup, tree, super

% \renewcommand{\glossarypreamble}{\label{glossary}}
\printglossary[title = List of Symbols, toctitle = List of Symbols, style=symbunitlong, type=symbolslist]  
\renewcommand{\glossarypreamble}{\label{acronym}}
\printglossary[type= acronym, title = List of Abbreviations,toctitle = List of Abbreviations] %, style=super4col
\newpage

% ------------ list of figures and tables ------------
%\addcontentsline{toc}{chapter}{\listfigurename}
%\listoffigures
%\addcontentsline{toc}{chapter}{\listtablename}
%\listoftables
% \glsresetall % reset all glossaries

%intermezzo about making titles conform
\titleformat{\chapter}[display]{\flushright\vspace{-3cm}\pagecolor{white}}
{\usefont{OT1}{pag}{b}{n}\fontsize{30}{34}\selectfont {CHAPTER~\thechapter}}{10pt}{\usefont{OT1}{pag}{m}{n}\fontsize{22}{24}\selectfont}[{\thispagestyle{empty}\vspace{3cm}}]

% ------------ TEXT ------------
\mainmatter

%=======================================
%We want specific numbering here
\fancyhf{}
\fancyhead[LE,RO]{\helv \thepage}
\fancyhead[LO]{\helv \leftmark}
\fancyhead[RE]{\helv \rightmark}
\renewcommand{\headrulewidth}{0.15pt}
%=======================================

%classification is dependent on the work itself
\chapter[Introduction]%
{Problem statement, research objectives}%and outline

\section{Introduction}
Membrane filtration is a purely physical separation process where a suspension is drawn through a semi-permeable membrane and the suspended constituents larger than the membrane pores are retained. In descending pore size, a distinction is made between microfiltration, ultrafiltration, nanofiltration and reverse osmosis. Figure \ref{fig:memFilt} gives an overview of the different membrane filtration types along with the class of solids that are typically retained by the membrane.\par
\begin{figure}[H]
 \centering
 \includegraphics[width= 0.5\textwidth]{figs/membraneFiltration.png}
 \caption{An overview of the different membrane filtration types \citep{Luque2008}. \label{fig:memFilt}}
\end{figure}
The presence of membrane filtration in wastewater treatment processes is well established by its main application, the \gls{MBR}.
An \gls{MBR} is a combination of a biological reactor for the conversion of solutes to suspended solids (activated sludge) combined with a membrane filter to remove these solids and thereby produce clean water \citep{MBRBook}. Membrane filtration replaces the traditional, gravitational sedimentation of activated sludge and produces water with a significantly higher quality \citep{Judd2008}. Moreover, \gls{MBR}s are more robust and are able to handle higher concentrations of suspended solid, leading to a higher loading capacity and consequently more compact treatment plants. Other, more exotic, applications of membrane filtration in wastewater treatment are, among others, the removal of emulsified oils and the recovery of heavy metals \citep{Fu2011,Cheryan1998}. \par
Nowadays, membrane filtration is also being employed in a vast range of industrial, medical and biotechnological processes and the market of membrane filtration is ever growing \citep{BOEKCFMMarket,Luque2008}. %and has a high impact on industry \citep{Luque2008}. 
Important applications of membrane filtration can be found in the paper production industry for lignosulfonate fractionation and color removal \citep{Luque2008}, in the food industry for the clarification of beer, wine and vinegar \citep{Cimini2013,Ulbricht2009,adnan2012}, in the medical sector for the continuous filtration of blood plasma and in biotechnology for the clarification  of fermentation broth \citep{Homsy2012,dsp}.
Several potential uses for membrane filtration in renewable energy applications, such as biogas upgrading and biodiesel purification, are proposed in literature \citep{Charcosset2014,Dube2007}. In short, membrane filtration has become indispensable in industrial as well as wastewater treatment processes. \par %In biodiesel production, the use of a membrane reactor for the continuous seperation of fatty acid methyl esthers (FAME), methanol and glycerol from the tryglycerides has been documented in . 
%In short, membrane filtration \textbf{zijn niet meer weg te denken van uit de industrie}. 
There is, nonetheless, a major disadvantage coupled with membrane filtration, i.e.\ membrane fouling. The continuous feed and retention of suspended solids leads to the formation of fouling layers on top (filter cake) or inside the membrane and increases the resistance towards liquid permeation. Hence, operation of a membrane filtration unit requires continuous fouling control which includes backwashing, aeration and chemical cleaning of the membrane. These procedures are not able to fully regenerate the membrane due to irrecoverable fouling, leading to the insuperable decay of the membrane. The effectuation of fouling control measures and the replacement of membranes gives rise to considerable operational expenses, making membrane filtration a costly technique \citep{Owen1995}.
% \todo[inline]{misschien goed om te benadrukken dat fouling ontegensprekelijk vasthangt aan membraanfiltratie, en dus niet te vermijden is, maar dat het wel zoveel mogelijk kan ingedijkt worden.}
% https://en.wikipedia.org/wiki/Filter_cake
% machtige afbeelding Luque2008

%Chemical analysis \citep{Hong2016}, \\% Highlights: The new method for pesticide analysis by membrane filtration, The recovery and purification characteristics can be adjusted by membranes and eluting solvent and The membrane filtration method has high purification ability.

% Why MBR?
% \begin{itemize}
%  \item high effluent quality
%  \item produces less sludge
% \end{itemize}
% \textbf{the MBR book}

\section{Problem statement}
In spite of the growing importance of membrane filtration in industry and wastewater treatment processes, there is still little understanding of the underlying processes of membrane filtration. In order to 
suppress fouling and prolong the lifetime of pressure-driven membrane systems as much as possible, the operation is highly conservative. Fouling remediation procedures %such as backwashing, \textbf{...}, chemical cleaning 
are performed frequently, leading to a sub-optimal operation %with a lot of downtime 
which is costly and inefficient both from an energetic and material perspective. Quite some efforts have been put in fouling research, but there is still poor insight in the system dynamics. Efforts to model membrane filtration and the fouling build-up are mostly empirical and typically based on the \gls{RIS} approach (Section \ref{sec:RIS}). Such models are capable of accurately predicting flux decline and \gls{TMP} increase over time, but only under the specific operational conditions for which they are calibrated. Hence, they lack extrapolation capacity to accurately predict fouling rates and optimal backflushing frequencies in a general setting. %a lot of variation in MBR function properties and conditions. 
Moreover, \gls{RIS} models are often overparameterised and %as a consequence, 
require frequent recalibration with a vast amount of data for accurate parameter estimations, making real-time recalibration typically not feasable. 

These reasons have led to the fact that efforts have not yet resulted in a universally applicable membrane filtration model, neither do they agree on fundamental fouling principles. An in-depth understanding of the underlying fouling processes will allow for a more responsive, dynamic operation and a better design of filtration installations. Furthermore, the availability of accurate and robust fouling models will enable the implementation of more advanced control strategies, such as internal model control, model-based predictive control, or in silico tuning of process controllers. \par

% requires a mechanistic approach that models the core processes as realistically as possible 
%   \item zotte fenomenen, rollen van particles expanded layer
%  \item Modelling can be an important tool in the elucidation of underlying mechanims. However,  currently membrane filtration is modelled through RIS-models which are empirical and deliver no insight in the underlying mechanism

% Constructing an accurate and universally applicable fouling model is a complicated task as such a model needs to encompass all the core mechanims while keeping a restraint on the complexity to limit the computational requirements.

\section{Objectives of this research}
The first objective of this master thesis is the extension of the mechanistic, spatially explicit modelling framework proposed by \cite{Ghijs2014}. The model has its limitations and will be critically analysed, extended and polished, so to progress towards a physically more accurate description of all relevant processes contributing to filter cake formation. \par
The second objective is the model-based design and realisation of a laboratory scale membrane filtration system for the calibration and validation of the abovementioned model. \par
It should be kept clear at all times that the purpose of this model is not the real-time evaluation of operational conditions, but rather unraveling fouling mechanisms. Hopefully, this model will be able to provide valuable insights into the key mechanisms of membrane fouling needed for the future development of computationally efficient filtration models for dynamic control, backflushing prediction, \gls{CAD}, etc. 
% \textbf{Afkomstig van de poster uit de hang:}
% objectives
% \begin{itemize}
%  \item to unravel cake layer build-up mechanims
%  \item improving empirical filtration models
% \todo[]{empirical?!}
%  \item Towards dynamic control and optimised water recuperation 
% \end{itemize}

% The idea is to build a very advanced model that models filter cake formation as realistically as possible, such a model However will be computationally very demanding and will not be able to use for real time simulation in control loops or for backflushing predictions. the process knowledge from this model will be used to develop an intermediate model that is more advanced than the empirical RIS approaches but less demanding than the full model which can be used for process optimisation etc...  

\section{Outline: the roadmap through this dissertation}
This dissertation starts with a literature review of membrane fouling models and profilometric techniques. Chapter \ref{spatModel} provides a description of the model developed by \cite{Ghijs2014}. The improvements of this model, performed in this thesis, are discussed in Chapter \ref{ModelDev}. Next, the results of a qualitative validation of the improved model and a scenario analysis are elucidated in Chapter \ref{ch:results}. \mbox{Chapter \ref{ch:disc}} addresses the modelling imperfections and provides guidelines for the future development of the model. Finally, this master dissertation presents the conclusions in \mbox{Chapter \ref{ch:concl}}.
% 
% Results
% \begin{itemize}
%  \item A working model for particle-fluid interactions
%  \item Simulating build-up for multidisper mixturess.
%  \item Designing a calibration/validation device
% \end{itemize}

%----------------------------------------------------------------------------------------------------------
\clearpage
\clearpage{\pagestyle{empty}\cleardoublepage} %introduction
\hyphenation{}
\chapter[Literature Review]%
{Literature Review \label{litRev}}
% handy link for kozeny equation: https://books.google.be/books?id=YjahdHIfon0C&pg=PA28&lpg=PA28&dq=darcy%27s+law+filtration&source=bl&ots=6W1q2uOF-e&sig=pVIHxWmQmGvVwa3XPlWqenVdVJ4&hl=nl&sa=X&sqi=2&ved=0ahUKEwi90I-p3bnMAhWKF8AKHS0wD90Q6AEINzAD#v=onepage&q=darcy's%20law%20filtration&f=false

In order to further develop and improve the spatio-temporal model of filter cake formation elaborated by \cite{Ghijs2014}, it is necessary to thoroughly review the various modelling approaches in literature. In this manner, all processes and accompanying interdependencies can be mapped out, which enables the development of a realistic model comprising all relevant processes. The first part of this chapter is dedicated to this. In the second part of this chapter, a few promising profilometric techniques, necessary for model calibration, will be discussed.

\section{Membrane fouling}
Membrane fouling comes in many forms and types. Furthermore, generally accepted membrane fouling terminology is altered by different authors in the field. Before exploring the different models, some important concepts and the associated lexicon will be defined. \\ \\
\cite{Mohammadi2003} defines membrane fouling as \textit{`` [...] the existence and growth of micro-organisms and the irreversible collection of materials on the membrane surface which results in a flux decline."} However, a more suitable definition is \textit{`` [...] the process resulting in loss of performance of a membrane due to deposition of suspended or dissolved substances on its external surfaces, at its pore openings, or within its pores"} by \cite{Shimidzu1996} because it also includes pore fouling. \\ \\
Depending on the type of membrane filtration (nanofiltration, microfiltration, etc.\ ), different types of fouling can be encountered during operation. Generally, fouling can be classified as follows:

\begin{itemize}
\item scaling: precipitation of substances due to exceeding the solubility product induced by the process of concentration polarisation;
\item particulate fouling, inorganic and organic;
\item biofouling: fouling effects due to colonisation by bacteria;
\item fouling by macromolecular substances;
\item chemical reaction of solutes with the membrane polymer and/or its boundary layer;
\end{itemize}  
It is nevertheless important to keep in mind that these are only the main types. It is possible that other, more specific types exist. Some authors prefer to classify fouling in terms of persistency (reversible, irreversible and irrecoverable fouling) whilst others favor a mechanical categorisation (pore blocking, cake formation, intermediate blocking, etc.\ ). Unfortunately, these classifications cannot be compared mutually, as there is no complete parity between any of these types. \par
% 
% \begin{figure}[H]
% \begin{center}
% \hspace{-1.5cm}
% %\includegraphics[]{figs/.PNG}
% \caption{....\label{foulingImage}}
% \end{center}
% \end{figure}
% \todo[inline]{JB: Hier zou een foto van zo een cake kunnen passen om de gedachten te vestigen, BDJ: heb nog geen verduidelijkende afbeelding gevonden}
Another important phenomenon to elucidate is concentration polarisation. This is the tendency of solutes to accumulate near the membrane. Materials, rejected by the membrane, accumulate in the vicinity of the membrane surface. The thickness of this layer is governed by the hydrodynamics; increasing crossflow velocities and a decreasing membrane flux result in a decreasing thickness. This is partially compensated by diffusion (back diffusion) and under steady-state conditions a balance is established between the forces that transport the solutes to, through and away from the membrane \citep{MBRBook}. 

\begin{figure}[H]
\begin{center}
\hspace{-1.5cm}
\includegraphics[width=0.7\textwidth]{figs/croppedCP.png}
\caption{Schematic representation of the driving forces leading to concentration polarisation in crossflow membrane filtration processes \citep{MBRBook}.\label{CP}}
\end{center}
\end{figure}

\section{Membrane fouling models}
Now that the lexiconic framework has been established, it is possible to unambiguously describe the different existing modelling approaches that are most relevant for this thesis. Both mechanistic and data-driven models will be discussed. Here, ``data-driven" is regarded as black-box modelling using machine learning techniques, whilst ``mechanistic" is regarded as gray-box modelling.

\subsection{Resistance-in-series models \label{sec:RIS}}

The majority of fouling models are based on the \gls{RIS} concept. This approach, based on Darcy's law, Eq.\ \eqref{Darcy}, defines a membrane resistance $\gls{R}$ [\unit{\reciprocal\metre}] to relate the flux $\gls{J}$ [\unit{\cubic\metre \, \rpsquare\metre \, \reciprocal\second}] across the membrane to the \gls{TMP} or $\gls{dp}$ [\pascal],
\begin{equation}
\gls{dp} = \gls{J}\, \gls{R}\, \gls{fluidDyn}  \, ,
\label{Darcy}
\end{equation}
with \gls{fluidDyn} [\unit{\kilogram \, \reciprocal\metre \, \reciprocal\second}] the dynamic viscosity of the fluid. \par
Typically, the total membrane resistance consists of different resistance terms in series, the clean membrane resistance $\gls{Rm}$ [\unit{\reciprocal\metre}] and various resistances originating from different fouling layers. Mostly they comprise of fouling types from different classifications which is tricky as overlap of the underlying process is possible, e.g.\ pore blocking and irreversible fouling are not independent. The clean membrane resistance is the inherent resistance of the membrane, which is constant, and is provided by the membrane manufacturer or obtained from pure water filtration experiments \citep{Naessens2012}.

Numerous \gls{RIS} models are proposed in literature, each  introducing different resistance terms. For each resistance term a separate model needs to be developed. These models can be mechanistic, describing the real mechanisms of the process or semi-empirical, requiring  more careful calibration \citep{Naessens2012}. \\ \\One of the more simple, empirical \gls{RIS} model is presented in \cite{Khan2009}. Here, the total hydraulic resistance $\gls{Rt}$ [\unit{\reciprocal\metre}] is defined as the sum of the cake resistance $\gls{Rc}$ [\unit{\reciprocal\metre}] caused by deposition of particulate matter on top of the membrane, the fouling resistance $\gls{Rf}$ [\unit{\reciprocal\metre}] due to pore blocking and adsorption of matter within the membrane and the abovementioned clean membrane resistance $\gls{Rm}$, i.e.
\begin{equation}
\gls{Rt} = \gls{Rm} + \gls{Rc} + \gls{Rf}
\label{RISKhan}
\end{equation}

The resistance terms in Eq.\ \eqref{RISKhan} are calibrated with different filtration experiments.
Measurements of the \gls{TMP} and flux in combination with Darcy's law result in values for the different resistance terms. $\gls{Rm}$ is determined through filtration experiments on a chemically cleaned membrane, while $\gls{Rf}$ is measured with a membrane where the cake was removed after a previous filtration experiment. $\gls{Rt}$ was determined from the final flux and \gls{TMP} at the end of the filtration experiments. Finally, $\gls{Rc}$ can be obtained by re-arranging Eq.\ \eqref{RISKhan} and filling in the known resistances. This approach is very straightforward, the resistance terms are obtained by directly fitting Eq.\ \eqref{Darcy} to the experimental data. \par
A more elaborate approach is described by \cite{Wintgens2003}. The same resistance terms %($\gls{Rm}$, $\gls{Rc}$ and $\gls{Rf}$)
are used as in the previous model, but each term is described by semi-empirical equations instead of deriving them directly from Darcy's law. The cake resistance $\gls{Rc}$ is assumed to be dependent on the concentration of the cake layer forming component at the membrane surface $\gls{cM}$ [\unit{\kilogram\, \rpcubic\metre}] as follows,
\begin{equation}
\gls{Rc} = \gls{kc}\, \gls{cM}
\label{RcWint} ,
\end{equation}
with \gls{kc} [\unit{\metre\squared\, \reciprocal\kilogram}] an empirical parameter. \par
When considering concentration polarisation effects, $\gls{cM}$ follows from
\begin{equation}
\gls{J} = \gls{kp}\, ln \left(\cfrac{\gls{cM}}{\gls{cb}} \right)
\label{polarisationWint} ,
\end{equation}

with $\gls{kp}$ [\unit{\cubic\metre\, \rpsquare\metre\, \reciprocal\second}] the local mass transfer coefficient and $\gls{cb}$ [\unit{\kilogram\, \rpcubic\metre}] the bulk concentration of suspensed solids. \par
The authors assume that the fouling resistance $\gls{Rf}$ is dependent on the total permeate volume produced during filtration as
\begin{equation}
\gls{Rf} = \gls{Sf}\, (1-e^{-\gls{kf}\, \int_0^t J(t)\, \mathrm{d}t})
\label{foulingWint} .
\end{equation}
with \gls{Sf} [\,-\,] a factor that represents the specific surface area of the membrane that can be covered by fouling products and \gls{kf} [\,-\,] an empirical parameter. \par

The model parameters $\gls{kc}$, $\gls{Sf}$, $\gls{kf}$, $\gls{kp}$ and the clean membrane resistance $\gls{Rm}$ are obtained through calibration.\par
Model validation shows that this approach is able to accurately predict the flux. It is important to note that validation was done with data from another filtration unit, independent from the calibration dataset. Some of the \gls{RIS} models, discussed in this section, are capable of closely approximating the impact of fouling on process variables such as flux and \gls{TMP}. Still, these semi-empirical models do not yield insight into the different fouling mechanics. Our objective is to better characterise the processes and mechanics behind fouling. With this in mind it is necessary to move towards more advanced mechanistic models that aim at fully describing the major physical processes in play. Additionally, such models have the tendency to be more widely applicable, in contrast to the empirical models that need to be recalibrated when applied in other operational conditions.

\subsection{Advanced mechanistic models}
As previously mentioned, \gls{RIS} models dominate the fouling modelling landscape. This section describes the more advanced, mechanistic fouling models. Most of these models also use the \gls{RIS} approach in which the different resistance terms are described mechanistically. A summary of the basic ideas behind these approaches will be given, followed by a critical review of their strengths and weaknesses.
%Most of these models are semi-physical models i.e.\ some of the more well-known processes are described through knowledge-based modelling while other processes are still described by parametric black-box models.\par
\subsubsection{Force balance}
\cite{Lu1993} start from the idea that the cake resistance is largely dependent on the structure of the filter cake and the size of the bulk particles. Consequently, a model is developed that is first of all capable of accurately predicting the cake layer structure in a dead-end, constant pressure filtration system and secondly, incorporates the particle size dependencies. \par
For a particle that is depositing on the filter cake a critical friction angle is calculated to determine if the particle will ``stick" to the surface (membrane or filter cake) or not. This angle of friction $\theta$ [\,-\,] is defined as the angle between the gravity vector and the line between the particle centers, A (depositing particle) and B (deposited particle) (Figure \ref{critFric}).
Below a certain critical friction angle $\gls{thetaC}$, the friction between particles A and B is large enough for particle A to deposit on particle B. In contrast, for $\gls{theta}$ values larger than $\gls{thetaC}$, not enough friction is occurring and particle A will not be able to deposit and slide past particle B.
\begin{figure}[H]
\begin{center}
\hspace{-1.5cm}
\includegraphics[]{figs/critFric.PNG}
\caption{Representation of the considered forces on a settling particle at the cake surface and the friction angle \citep{Lu1993}.\label{critFric}}
\end{center}
\end{figure}
$\theta_c$ is determined from a force balance on the depositing particle (A). At the critical condition ($\theta = \gls{thetaC}$), the tangential forces are in equilibrium with the friction forces, so
\begin{equation}
(F_g +F_d)\, \sin(\theta_c) = f_c\,(F_i+(F_g + F_d)\, \cos(\theta_c))
\label{Lu1993FB} ,
\end{equation}
with $\gls{Fi}$ [\unit{\newton}], the interparticle force, $\gls{Fd}$ [\unit{\newton}] the drag force, $\gls{Fg}$ [\unit{\newton}] the gravity force and $\gls{fc}$ [\,-\,] the friction factor between particles \citep{Lu1993}. When the values of all forces in Eq.\ \eqref{Lu1993FB} are known it is possible to determine $\theta_c$. A detailed explanation and derivation of the different equations for the forces and parameters can be found in the original article by \cite{Lu1993}. The number of particles arriving at the cake surface is controlled by the concentration and flux. The deposition point is determined by ``dropping" particles from a random position onto the cake or membrane and evaluating the angle of friction, as explained above. Hence, a cake structure is obtained for a certain value of \gls{thetaC}. For the particle stacking, perfectly spherical particles are assumed but the porosity can be corrected with a shape factor $\phi_s$ [\,-\,] for other shapes,
\begin{equation}
  \phi_s= \cfrac{1-\epsilon}{1-\epsilon_\mathrm{sph}} \, ,
\end{equation}
with $\epsilon$ [\,-\,] the actual porosity of the filter cake for non-sperical particles, $\epsilon_\mathrm{sph}$ [\,-\,] the porosity of the filter cake determined by the model, assuming perfect spheres. A full mathematical description on the determination of $\phi_s$ is given in \cite{Cross1985}. \par
Also the process of compression is considered, by calculating a porosity change based on a fluid mass balance. The flux across the filter is calculated with the Kozeny-Carman equation,
\begin{equation}
\gls{J} = \cfrac{\mathrm{\Delta}p}{l}\, \cfrac{\epsilon^3}{\gls{fluidKin} \, K\, S^2 \, (1-\epsilon)^2}
\label{KozenyCorrect} .
\end{equation}
This relation indicates that the flux $\gls{J}$ through a filter with depth $\gls{dl}$ [\unit{\metre}] is influenced by the \gls{TMP} $\gls{dp}$, the specific surface area $\gls{S}$ [\unit{\metre\squared}], the kinematic viscosity of the fluid $\gls{fluidKin}$ [\unit{\metre\squared\, \reciprocal\second}], the cake porosity $\gls{eps}$ and modulated by the Kozeny constant $\gls{K}$ [\,-\,]. \par
An expression for the Kozeny constant in function of the porosity is obtained by solving the Navier-Stokes equations and a continuity equation. Which in turn, when combined with Eq. \eqref{KozenyCorrect}, gives rise to an equation for the specific filtration resistance $\gls{alpha}$ [\unit{\metre\, \reciprocal\kilogram}] of the filter cake (Eq. \ref{HwangRes}). 
%A continuity equation in physics is an equation that describes the transport of some quantity. It is particularly simple and particularly powerful when applied to a conserved quantity, but it can be generalized to apply to any extensive quantity.
\begin{equation}
\alpha = K \, S^2 \, \frac{1-\epsilon}{\epsilon^3 \, \rho_s}
\label{HwangRes} ,
\end{equation}
where $\gls{rhoS}$ [\unit{\kilogram\, \rpcubic\metre}] is the density of the solids. \par
The predicted average porosity and average specific resistance of the filter cake closely approximate the experimental values. However, this model is restricted to dead-end filtration which is rarely used in practice and this model is therefore not applicable to crossflow filtration systems. For example, considering Eq.\ \eqref{Lu1993FB}, one also needs to take into account the lift force, history force, added mass force, etc.\ \citep{Ghijs2014}. The assumption of a spatially homogeneous flux is also too straightforward and might influence the rate of local cake layer build-up considerably, as well as the architecture and porosity. Nevertheless, the introduction of a critical friction angle and the correction of the cake porosity with the shape factor of the particles are valuable ideas. 
%it can be expected that this   The force balance is only used in the initialisation to calculate $\theta_c$.   }
\subsubsection{Pore blocking}
A three-dimensional fouling model for the microfiltration of a polydisperse, charged solution was developed by \cite{Yoon1999}. An important improvement compared to \cite{Lu1993} is the consideration of both pore blocking and cake layer formation. The model allows for the simulation of flux in function of time for any concentration of iron oxide particles, within the validated concentration range. \par
The effective particle deposition rate is calculated for each particle in the premised \gls{PSD}, taking into account both processes that ``push'' the particle towards the membrane and backtransport processes. This balance takes into consideration: inertia lifting $\gls{vl}$ [\unit{\metre\, \reciprocal\second}] particle interaction $\gls{vi}$ [\unit{\metre\, \reciprocal\second}], convection $\gls{v}$ [\unit{\metre\, \reciprocal\second}], diffusion $\gls{vd}$ [\unit{\metre\, \reciprocal\second}] and shear induced diffusion $\gls{vs}$ [\unit{\metre\, \reciprocal\second}]. The effective deposition velocity is the difference between the backtransport velocity:
\begin{equation}
\gls{vtot}(\gls{partDia})=\gls{vd}+\gls{vl}+\gls{vs}+\gls{vi}
\label{yoonsum}\, ,
\end{equation}
and the velocity toward the membrane $\gls{v}$ [\unit{\metre\, \reciprocal\second}], which is governed by the flux. With $\gls{partDia}$ [\unit{\metre}] the particle diameter. The gravitational settling velocity $\gls{vg}$ [\unit{\metre\, \reciprocal\second}] is not included in this balance, so it is assumed that the effect of gravity is negligible. Subsequently, the distribution of particle sizes that are able to deposit on the membrane, given the current flux and backtransport velocity, is given by 
\begin{equation}
F_{\mathrm{num}}(\gls{t},\gls{partDia})=F_{\mathrm{num}}(\gls{t}_\mathrm{0},\gls{partDia}) \, \cfrac{\left( J(\gls{t}) - \gls{vtot}(\gls{partDia})\right)}{J(\gls{t})}
\label{Yoon1999Dist} ,
\end{equation}

with $F_{\mathrm{num}}(\gls{t}_\mathrm{0},\gls{partDia})$, the initial particle size distribution. \par
The cake layer is built up by depositing particles, 
sampled from $F_{\mathrm{num}}(\gls{t},\gls{partDia})$, one by one on the membrane surface. The particle is dropped from a random location above the membrane. A rolling algorithm is applied whenever a particle comes into contact with a previously settled particle. The rolling continues until a stable position is reached, i.e. the particle touches three already settled particles or the membrane surface. After deposition, the particle is tested for pore blocking (Figure \ref{poreBlock}).  
\begin{figure}[H]
\begin{center}
\def\svgwidth{0.6\columnwidth}
\input{figs/yoon.pdf_tex}
\caption{Pore blocking rule implemented in the model of \cite{Yoon1999}. Only particles that touch the membrane surface and are centered within the pore boundaries constitute to pore blocking. Figure adapted from \cite{Yoon1999}. \label{poreBlock}}
\end{center}
\end{figure}

The flux through the membrane is obtained with Darcy's law in conjunction with a \gls{RIS} model, taking into account the inherent membrane resistance and pore blocking resistance \gls{Rb} [\unit{\reciprocal\metre}]. The latter is computed from  
\begin{equation}
\gls{Rb}= \cfrac{\gls{N0}}{(\gls{N}-1)} \, \gls{Rm} ,
\label{Yoon1999Rb}
\end{equation}
the flux follows from
\begin{equation}
J=\frac{\gls{dpteff}}{\gls{fluidKin} \, (\gls{Rm}+\gls{Rb})} .
\label{Yoon1999J}
\end{equation}
with $\gls{N0}$ [\,-\,] the initial number of membrane pores, $\gls{N}$ [\,-\,] the free pores.
Eq. \eqref{Yoon1999J} does not contain a resistance term for the cake layer, but its effect is incorporated in $\gls{dpteff}$ [\unit{\pascal}] where the pressure drop over the membrane and cake is lowered by the pressure drop over the cake $\gls{dpc}$ [\unit{\pascal}].
The specific surface area and porosity vary with the cake layer depth. Consequently, $\epsilon$ in the Kozeny-Carman equation (Eq. \eqref{KozenyCorrect}) is not a constant. To deal with this issue, the pressure drop is calculated over different ``slices'' in a recursive manner (Eq.\ \eqref{KozenySlice}, Figure \ref{YoonSlice})
\begin{equation}
\Delta p_{\mathrm{T}}(i+1) =\Delta p_\mathrm{{T}}(i) - \cfrac{\gls{fluidKin}\, \epsilon_{\mathrm{i}}^3 \, J}{5 \, S_{\mathrm{i}}^{2}(1-\epsilon_{\mathrm{i}})^{2}} \, \gls{di},
\label{KozenySlice}
\end{equation}

with $S_\mathrm{i}$ [\unit{\metre\squared}] the specific surface area of the particles and $\gls{di}$ [\unit{\metre}] the thickness of a slice. It is not clear why the Kozeny constant $k$ is missing from Eq.\ \eqref{KozenySlice}. \par
After applying this scheme to every slice, $\gls{dpc}$ is obtained and the flux across the membrane is computed with Eq.\ \eqref{Yoon1999J}.
\begin{figure}[H]
\begin{center}
\hspace{-1.5cm}
\includegraphics[width=0.6\textwidth]{figs/pressuredropYoon.PNG}
\caption{Representation of the cake layer and the subdivision in different slices \cite{Yoon1999}.\label{YoonSlice}}
\end{center}
\end{figure}

Each time step, one particle is sedimented on the cake or membrane surface. The elapsed time between two particle depositions is evaluated with Eq.\ \ref{timestep}. 
\begin{equation}
\Delta \gls{t} =\left( \left( J(\gls{t}) - \gls{vtot}(\gls{partDia})\right)\, \gls{Am} \, \gls{cb}\, F_{\mathrm{num}}(\gls{t},\gls{partDia}) \right)
\label{timestep}
\end{equation}
with $\gls{Am}$ [\unit{\metre\squared}] the specific membrane area.
In the next time step, Eq. \ref{Yoon1999Dist} is re-evaluated with the new flux and a new particle is sampled from the new \gls{PSD}.\par
The model performs quite well in the early stages of the filtration but the prediction accuracy gradually declines with time. The simulated flux evolves to a steady state while experimental values show that the flux keeps decreasing. The authors mention that unfulfilled assumptions for the backtransport equations as the probable cause, but the discrepancy between the simulations and experimental data can also be due to the overestimation of the inertial lift force. The channel inlet velocity is used as the velocity component in this force and this component is overestimated considerably for particles in the slow moving fluid close to the membrane surface. A more involved calculation of the velocity would probably enhance model performance. The added value of this model lies in the introduction of pore blocking in a less empirical manner, the polydispersity and charge interactions.\\ \\
% \textbf{Strengths/weaknesses}
% \begin{itemize}
% \item strength: introduces particle rolling after sedimentation
% \item strength: also takes into account pore blocking.
% \item strength: three dimensional.
% \item strength: packing rules in the cake-layer
% \item strength: multi-disperse.
% \item strength: charge interactions.
% \item weakness: \textbf{no dynamic fluid the calculation of the forces is based on the maximum velocity, the velocity at the entrance, which is only partially true because of the parabolic profile the velocity in the middle of the tube, away from the entrance is higher than the entry velocity.}
% %nog eens kijken naar die pore blocking condition ben hier nog niet van overtuigd 
% \item weakness: only one cake layer with no local fluctations of the cake layer properties (no sectional approach)
% \end{itemize}

%Vraag: mij is het niet duidelijk waar in de filter de krachtenbalans uitgerekend wordt, is dit enkel aan de rand? Waar halen ze de waarde van de fluidum snelheid?
In the previously discussed models, the filter cake is either assumed spatially homogeneous, with an average value for porosity and thickness, or heterogeneous along the depth. Yet, heterogeneity along the longitudinal axis is never considered. This implies that there is no spatial variation of the filtration resistance and flux. Consequently, these models are not able to account for the spatial heterogeneity of membrane fouling. \cite{Li2006} propose a sectional approach to deal with this problem. \par
This sectional method allows the inclusion of turbulence, induced by aeration. The membrane surface is subdivided into different sections with equal length in which the different variables are tracked. The subdivision is along the longitudinal axis of the membrane, in contrast to the method discussed in \cite{Yoon1999}, where the ``slices" are taken parallel to the membrane. \par
For simulating the attachment of a sludge particle with a certain diameter ($\gls{partDia}$) on the membrane surface, two forces are taken into account: the drag force $\gls{Fd}$ and the lift force $\gls{Fl}$ [\unit{\newton}]. The permeate flux drags the particles to the membrane and the lift force, a consequence of the turbulent flow, is the opposing force. The balance between these two forces controls the rate of particle deposition, expressed as a probability of deposition $\gls{E}$ [\,-\,]: \par
\begin{equation}
\gls{E}=\cfrac{\gls{Fd}}{\gls{Fd}+\gls{Fl}}
\label{Prob}
\end{equation}
For a probability $\gls{E}$, the rate of biomass attachment becomes, 

\begin{equation}
\cfrac{\mathrm{d} M}{\mathrm{d} t} = E\, C\, J
\label{LiAttRate}
\end{equation}

with $\gls{cb}$ [\unit{\kilogram\, \rpcubic\metre}], the concentration of \gls{SS}.\par

The effects of the continuous scouring through aeration is expressed as a rate of detachment described by Eq.\ \ref{LiDeattRate}.

\begin{equation}
\cfrac{\mathrm{d} M}{\mathrm{d} t} = -K_d \, \gls{Msf}
\label{LiDeattRate}
\end{equation}
 
with $\gls{Kd}$ [\unit{\reciprocal\second}] the rate coefficient of sludge detachment and $\gls{Msf}$ [\unit{\kilogram\, \rpsquare\metre}] the mass of sludge in the filter cake. \par
A Langmuir model is used for $\gls{Kd}$, as it reaches a maximum for a very thick filter cake and decreases with the cake thickness. The rate of detachment is essentially proportional to the shear intensity, biomass stickiness and other properties of the sludge layer. The net rate of sludge accumulation during a certain filtration period is obtained by solving the abovementioned equations for biomass attachment and detachment. \cite{Li2006} also describe equations for the rate of sludge removal during the idle-cleaning period. \par
The filtration resistance is calculated using a \gls{RIS} approach. The total resistance $\gls{Rt}$ is the sum of the intrinsic membrane resistance, the resistance of the dynamic and stable cake layer and the pore fouling resistance. All of these resistance terms involve an empirical resistance parameter that needs calibration. Finally, Darcy's law is used to calculate the flux through the different membrane sections. As opposed to other models, sludge detachment is considered and a sectional approach is established to capture the heterogeneity of membrane fouling. However, this model involves a lot of parameters and the article does not provide any information on the calibration. \par
% Moreover, the definition of a `` probability of deposition" seems to be misplaced when considering laminar conditions, a particle deposits or it stays in the bulk phase, there is no gray area. \par
Due to the many empirically described mechanisms, the performance of the model varies in different operational conditions, which might just indicate that the model does not comprise all relevant processes.
% Wouters review ends here
%
% possible inclusion of \cite{Busch2007}
%\textbf{Strengths/weaknesses}
%\begin{itemize}
%\item strength: Sectional approach
%\item strength: SMBR model with air scouring
%\item strength: takes into account sludge removal and sludge detachment
%\item strength: no taking into into account biofouling in a MBR
%\item weakness: the model has a lot of parameters that need calibration
%\item weakness: the model is very empirical
%\item weakness: calibration is not described, eventough this model involves a lot of empirical parameters.
%\end{itemize}
\subsubsection{Cut-off diameter}
A one-dimensional model describing the \gls{TMP} in a submerged hollow fiber membrane is proposed by \cite{Broeckmann2006}. The hydrodynamics at the outer side of the membranes are determined through a multiphase flow model for which the details are unfortunately not discussed in the article. \par 
The bulk phase particles are divided into two fractions. The first one is able to enter the pores and constitutes to pore blocking. The other fraction is not able to enter the pores and constitutes to cake layer formation. Mathematically, this is achieved by multiplying the retained weight fraction distribution of the membrane pore size distribution $\gls{Rpor}(\gls{partDia})$ with the distribution of the bulk particle sizes $\gls{gbulk}(\gls{partDia})$. For a particle with size $\gls{partDia}$ that is approaching a random pore, the probability of the pore being larger than $\gls{partDia}$ is given by $\gls{Rpor}(\gls{partDia})$. Thus, the \gls{PSD} of the fraction entering the pores is given by,
\begin{equation}
\gls{gmem}(\gls{partDia})=\gls{Rpor}(\gls{partDia})\, \gls{gbulk}(\gls{partDia})
\label{BroekEntering} .
\end{equation}

Hence, the \gls{PSD} of the retained solids follows from,
\begin{equation}
\gls{gcake}(\gls{partDia})=(1-\gls{Rpor}(\gls{partDia}))\, \gls{gbulk}(\gls{partDia})
\label{BroekRetained} .
\end{equation}

For what concerns pore blocking, it is assumed that every particle able to penetrate the membrane is completely retained within the membrane pores. Hence, the membrane porosity decreases when particles enter the pores. The rate of porosity change is calculated through a mass balance,
\begin{equation}
\gls{rhopm}\, \gls{Vm} \, \cfrac{\mathrm{d} \gls{epsm}}{\mathrm{d}\gls{t}} \, =-\gls{J}\, \gls{cbm}\, \gls{Am}
\label{broekPor} ,
\end{equation}

with $\gls{Vm}$ [\unit{\cubic\metre}] the total membrane volume, $\gls{rhopm}$ [\kilogram\, \rpcubic\metre] the density of particles in the membrane pores and $\gls{cbm}$ [\kilogram\, \rpcubic\metre] the mass concentration of particles that will penetrate the membrane, based on $\gls{gmem}(\gls{partDia})$. \par 
\cite{Broeckmann2006} also employs a \gls{RIS} approach.
The RIS model consists of four resistances; the cake resistance, the intrinsic membrane resistance, the pore blocking resistance and the irreversible resistance.
The Kozeny-Carman equation (Eq. \ref{KozenyCorrect}) can be rewritten to relate the membrane resistance to the membrane porosity, obtained from Eq.\ \ref{broekPor},
\begin{equation}
\gls{Rb} + \gls{Rm} = \cfrac{(1-\gls{epsm})^2\, \gls{Kp}}{(\gls{epsm})^3}
\label{Broeker} ,
\end{equation}

where $\gls{Kp}$ [\reciprocal\metre] is a membrane specific constant. $\gls{Rc}$ is obtained from,
\begin{equation}
\cfrac{\mathrm{d}\gls{Rc}}{\mathrm{d} \gls{t}} = \cfrac{\mathrm{d} l}{\mathrm{d}t}\, \gls{Kc}
\label{BroekCake} ,
\end{equation}
with \gls{dl} [\unit{}{\metre}] the filter cake thickness and the specific cake resistance $\gls{Kc}$,
\begin{equation}
\gls{Kc} = \cfrac{ \gls{k}\, 90}{\gls{dpHead}^2}\, \cfrac{ \left( \cfrac{ \gls{cc}}{\gls{rhoS}} \right) ^2}{\left( 1-\cfrac{ \gls{cc}}{\gls{rhoS}} \right) ^3}
\label{BroekBlakeKozeny} ,
\end{equation}

%hier had ik nog een vraagje over zie opmerking artikel in mendeley
%\todo[inline]{paper broekcake nog eens uitspekken}
with \gls{cc} [\unit{\kilogram\, \rpcubic\metre}] the mass concentration of the cake layer and \gls{dpHead} [\unit{\metre}] the mean diameter of bulk particles. \par
The cake layer formation is determined through a force balance (Figure \ref{BroeckmannForceBal}). \par
\begin{figure}[H]
\begin{center}
\hspace{-1.5cm}
\includegraphics[width=0.6\textwidth]{figs/BroeckmannForceBal.PNG}
\caption{Considered forces on a particle during filtration \citep{Broeckmann2006}.\label{BroeckmannForceBal}}
\end{center}
\end{figure}
$F_t$ [\unit{\newton}] is the tangential shear stress resulting from the liquid flow, $F_{\tau}$ [\unit{\newton}] is the friction force, the normal force $F_N$ [\unit{\newton}] is the drag force resulting from the permeate flux and $F_A$ [\unit{\newton}] is the adhesion force between the particles and the membrane. It is interesting to note that the lift force is not included in this mass balance. All forces point towards the membrane, no backtransport forces are considered to oppose this. With this fact in mind, a particle ``sticks" to the surface when the horizontal forces cancel out one another (Eq.\ \ref{Stickcondition}) and 
\begin{equation}
\gls{tauW}\, \gls{partDia}^2-\gls{muMax}\, (\gls{FN} + \gls{FA}) =  0 \, ,
\label{Stickcondition}
\end{equation}
where \gls{tauW} [\unit{\pascal}] is the shear stress and \gls{muMax} [\,-\,] is the maximum friction coefficient. \par
From Eq.\ \ref{Stickcondition} an equation is derived for the maximum diameter of particles that are able to adhere to the membrane surface, under the current filtration conditions. Particles larger than this cut-off diameter will stay in the bulk phase.\par
The growth of the cake layer is described by  Eq.\ \ref{BroeckCakeGrowth}. 
\begin{equation}
\gls{cc} \, \cfrac{\mathrm{d}\gls{dl}}{\mathrm{d} \gls{t}} = \gls{J}\, \gls{omega} \, \gls{cbc}
\label{BroeckCakeGrowth}
\end{equation}

with $\gls{omega}$ [\,-\,] the bulk concentration of particles that are retained on the membrane surface, i.e. the fraction of \gls{cbc} [\unit{\kilogram\, \rpcubic\metre}] that is smaller than the cutoff diameter and $\gls{cc}$ the mass concentration of the cake layer. \cite{Broeckmann2006} do not specify how $\gls{cc}$ is obtained even though it is a crucial variable/parameter. Hence, it is assumed that $\gls{cc}$ is a parameter that needs calibration. \par
During backflushing, particles are removed from the cake layer and pores. This process is incorporated through simple, empirical models. \par
The strength of this model is the implementation of particle and pore size distributions as typically only the former is included. Additionally, the implementation of backflushing processes definitely improves the applicability of this model even though the formulation is simple and empirical. The model however, has a few weaknesses including the lack of a sectional approach, many parameters and the low prediction accuracy when operational conditions differ from calibration conditions. Furthermore, some assumptions that can be valid for hollow fiber membrane systems might be invalid for other types of membrane filtration. The application of the model on these systems is therefore less appealing. Both a strength and a weakness of the model is its focus on constant flux filtration as most models address constant pressure filtration instead.
%The model contains a lot of parameters that need calibration. Validation shows that the model does not perform well when predicting the TMP with a flux that differs from the calibration flux. Also no sectional approach is used. Back flushing models are implemented even though there simple and empirical. Some assumptions can be valid for hollow fibre membrane systems, e.g. complete retention of particles inside the membrane but might be invalid for other types of filtration which makes the model less appealing to apply on these systems. This model addresses constant flux filtration, which can be seen as a strength or a weakness as most models address constant pressure filtration.}
%\begin{itemize}
%\item \textbf{New to this force balance is the addition of adhesive force between the particle and membrane.}
%\item weakness: No sectional approach is used.
%\item weakness: A lot of model parameters that need to be calibrated
%\item strength: This approach includes backflushing.
%\item weakness: the back flushing models are simple and empirical
%\item weakness: the results show that the model doesn't perform great when predicting TMP under a different flux
%\item weakness: some of the assumptions can be valid for fibre membranes but might be invalid for other membrane types, example= Complete retention of particles inside the membrane
%\item strength: not a lot of models with constant flux filtration 
%\end{itemize}

\subsubsection{Force balance, rolling and backwashing}
\cite{Smets} elaborates a model to predict the \gls{TMP} in an \gls{MBR} that combines the deposition criteria described in \cite{Broeckmann2006} and the particle depositing rules from \cite{Yoon1999} in a \gls{RIS}. \par
%Each series of particle depositions is defined as a ``filtration cycle". A filtration cycle ends when the total volume of depositing particles reaches the total volume of particles which can deposit during one cycle, following from Eq.\ref{CaoVmax}. \par
The cake layer is formed by particles that deposit one by one from a random location within the boundaries of the simulated membrane surface. A particle drops until it reaches the membrane surface or an already deposited particle; of the latter, a rolling algorithm is initiated until a stable position is reached. Each series of particle depositions is defined as a ``filtration cycle". Such a cycle ends when the total volume of deposited particles fulfills, 
\begin{equation}
V_{\mathrm{p,m}} = \cfrac{ \gls{Am} \, \gls{J}\, \gls{tf}\, \gls{pf} \, \gls{cb}}{\gls{rhoS}}
\label{CaoVmax} ,
\end{equation}

with $\gls{tf}$ [\unit{\second}], the filtration time per filtration cycle and $\gls{pf}$ [\,-\,] the total fraction of depositable particles. Eq.\ \ref{CaoVmax} is a combination of the equations presented in \cite{Yoon1999} and \cite{Broeckmann2006}. \par 
The porosity of this newly formed cake layer is evaluated at the end of each filtration cycle and is afterwards modified with a compression factor taking into account compression.  With both the porosity and cake thickness established it is possible to calculate the \gls{TMP} using a \gls{RIS}. This model is furthermore extended with a backwashing model simulating cake removal due to air scouring and backwashing sensu stricto. \par
The comprisal of different ideas and concepts results in a good performing model.
It also shows the importance of different factors such as PSD, compression and shear stress on the characteristics of the filter cake. This model offers a porosity profile along the cake thickness. Ideally this should be extended with a porosity profile along the cake length. Cake compression is included albeit through a simple compression factor.
The authors state the need for a hydrodynamic model in order to provide a more realistic shear stress.
 % description of the depositing rules misschien in appendix steken als we dit algoritme toepassen op ons model. Dit stuk zou beter bij broeckmann besproken worden
\subsubsection{Biofouling}
All of the abovementioned models describe particulate fouling. Nonetheless, it is important to keep in mind that this is not the only fouling type, biofouling has a considerable impact on different systems as this kind of fouling not solely occurs at membranes but also at heat exchangers, pipes, feed spacers, etc. For this reason, a great deal of effort has been put in the modelling of this fouling type. Consequently, these models are generally more advanced than the \gls{RIS} models mentioned above. Such a model is described in \cite{Picioreanu2009} and \cite{Vrouwenvelder2010}. This three-dimensional biofouling model simulates liquid flow, mass transport of a soluble substrate and biofouling in the feed channels of reverse osmosis and nanofiltration systems. The hydrodynamics are modelled via the steady-state Navier-Stokes equations for incompressible laminar flow. The distribution of substrate in the system is obtained through a mass balance. The biofilm is mimicked using an overlaying cellular automaton including terms for growth, decay, convective and diffusive biomass spreading, biomass attachment and detachment. \par
The simulations results (Figure \ref{picioPic}) really show the importance of coupling the fouling model with a fluid dynamics model. Figure \ref{picioPic} shows that when fouling persists, the fluid flow is redirected and high shear channels form where no fouling occurs. \par

\begin{figure}[]
    \centerline{
    \subfigure[]{\includegraphics[width=0.47\textwidth]{picioPica.PNG}}
    \subfigure[]{\includegraphics[width=0.47\textwidth]{picioPicb.PNG}}
    }
    \caption{Distribution of viscous shear and biofilm on the feed spacers for the initial condition (a) and after 2.5 days (b). The arrows show the direction and magnitude of the fluid velocity. The magnitude of the velocity is in the order of $1$ [m\, s\textsuperscript{-1}] \citep{Picioreanu2009}.   \label{picioPic}}
\end{figure}

The authors show that the model is able to accurately predict feed channel pressure drop and  biomass accumulation on the feed spacers. The simulated, three-dimensional distribution profiles of biomass and velocity agree qualitatively with the experimental measurements. The implementation of a fouling model in combination with \gls{CFD} is a major improvement towards accurate, mechanistic models. This methodology, implemented for biofouling is also highly relevant for particulate fouling as the fluid flow at the membrane greatly affects cake formation and vice-versa. This approach should be carried out in a sectional framework to incorporate heterogeneous cake formation. \\ \\
Table \ref{overviewtable} provides an overview of the processes, fouling types, force balances, etc.\ that are included in the abovementioned mechanistic models. The processes comprise cake compression, detachment of cake layer/biofilm, backflushing and dynamic fluid simulations. Two force balances are considered, one for particles in the cake layer and one for particles in the fluid. The fouling types include particulate fouling, pore blocking and biofouling and the spatial heterogeneity of the fouling is taken into account. Also, particle shape corrections and multidispersity of the particles are included in the comparison.     
%Bronnen [9-21] \cite{Picioreanu2009} eens bekijken voor CFD studies in membrane systems 
% 2D variants exist: [22-36]


%vette afbeelding is obligatory !

%\begin{itemize}
 %\item advantage: Very good model for the design and operation of membrane systems because of the 3D a lot less assumptions are being made.
%\end{itemize}


%    \begin{landscape}
%      \noindent
%      \thispagestyle{empty}
%      \begin{table}[ht]
%        \caption{Overview of the mechanistic models, FB = force balance, NM= not mentioned, PB= pore blocking , BF= biofouling}
%        \begin{flushleft}
% 	 \begin{tabular}{lccccccccccccc}
% 	   \hline
% 	   &  & \multicolumn{2}{c}{force balance} & & & & & & \multicolumn{3}{c}{fouling types}  \\
% 	   \cline{3-4} \cline{10-12}
% 	   \textbf{model} & crossflow & cake & fluid & heterogeneity & fluid dyn. & part. shape  & comp.  & multidisp. & cake & PB & BF & detach. & back flush.\\
% 	   \cite{Lu1993} &  & X &  & 1D  &  & X & X & NM & X & & & &\\
% 	   \cite{Yoon1999} & X & X &  & 1D  &  &  & X & X & X & X & & &\\
% 	   \cite{Li2006} & X & X &  & 2D  &  &  &  & NM & X & X & & X & N/A \footnote{hollow fiber} \\
% 	   \cite{Broeckmann2006} & X & X &  & 1D  & X \footnote{albeit not described in the article} &  &  & X & X & X & &  & X\\
% 	   \hline
% 	
% 	 \end{tabular}
%        \end{flushleft}
%        \label{tab:multicol}
%      \end{table}
%    \end{landscape}
 
   \begin{table}[]
     \caption{Overview of the incorporated processes, fouling types, etc.\ and general details of the advanced mechanistic models.}
       \begin{center}
	 \begin{tabular}{lccccccc}
	   & \rotlabel{-1in}{75}{\cite{Lu1993}} & \rotlabel{-1.1in}{75}{\cite{Yoon1999}} & \rotlabel{-1in}{75}{\cite{Li2006}} & \rotlabel{-0.85in}{75}{\cite{Broeckmann2006}} & \rotlabel{-1.29in}{75}{\cite{Smets}}& \rotlabel{-0.75in}{75}{\cite{Vrouwenvelder2010}}  \\
	   \toprule
	   \textbf{Operational mode} & DE & CF & CF & CF & CF & FF \\ \midrule
	   \textbf{Processes} & & & & & & \\
	   cake compression & X & X & & & X & \\
	   fouling detachment & & & X & &  \\
	   backflushing & & & & X & X & \\
	   fluid dynamics & & & & X & X & X \\ \midrule
	   \textbf{Force balance} & & & & & \\
	   fouling layer  & X & X & X & X & X & X \\
	   fluid & & & & & & X \\ \midrule
	   \textbf{Fouling} & & & & \\
	   particulate & X & X & X & X & X &  \\
	   pore blocking & & X & X & X & X & \\
	   biofouling & & & & & & X  \\
	   spatial heterogeneity & 1D & 1D & 2D & 1D & 1D & 3D \\ \midrule
	   \textbf{Other} & & & & \\
	   particle shape & X & & & &  \\
	   multidispersity & NM & X & NM & X & X & &  \\
	   \bottomrule
	  \end{tabular}
       \end{center}
       \label{overviewtable}
       \caption*{NM: not mentioned, DE: dead-end filtration, CF: crossflow filtration, FF: feed channel fouling.} 
	 %$\mathrm{^1}$ hollow fiber membrane filtration
     \end{table}
      % suggestions: multidispersity
	   % iets met crit fric
	   % charge interaction yoon1993
	   % hollow fibre/ tubular membrane, membrane type?
   
   



%\cite{Smets}
%
%\begin{itemize}
%\item Uses activated sludge floc size information
%\item Implemented in the model proposed by \cite{Broeckmann2006}
%\item 3D modelling and visualisation of the cake layer to give input information for the fouling model.
%\item The model starts from a particle size distribution (PSD)
%\item Broeckmann model, is used to get the cut-off diameter $d_{p,max}$. Only diameter < $d_{p,max}$ can deposit => new PSD (closely resembles the approach from \cite{Yoon1999}) 
%\item Building cake layer with sampled particles from new PSD sampling stops if the maximum amount of biomass that can deposit during one filtration cycle is reached. Particle depositing rules are followed.
%\begin{itemize}
%\item rules described in \cite{Yoon1999}), \cite{Smets} uses another rolling algorithm 
%\item Pore blocking
%\end{itemize}
%\item Compression of the cake layer at the end of the filtration and backwash cycle.
%\item evaluation of the cake porosity => broekmann => new $d_{p,max}$, evaluation of the cake layer resistance, TMP and cake layer thickness.
%\item TMP calculation using darcy's law
%\item RIS approach for $R_total= R_{mem}+R_{cake}$,$R_{cake}$ is the sum of resistances of all the sub-layers in the deposited cake => Blake-Kozeny
%\item for compression: check the original article
%\item Model calibration:
%\begin{itemize}
%\item empirical $R_{mem}$,$k_s$,$k_{kozeny}$,$tau_w$ are kept constant (I think they are obtained from the literature, need verification)
%\end{itemize} calibration: compression factor
%\end{itemize} calibration: 
%
%
%\textbf{empirical models ref 1-4 \cite{Broeckmann2006}}
%


%In this section an overview of the current filter cake modelling landscape is given. This critical overview will depict the strengths and weaknesses of the models.    %Beetje letterlijk van Michael maar dit klinkt zo mooi


%glossaries are used with gls or glspl (plural)
%\gls{rsa} is a sensitivity method. \gls{glue} is based\ldots

%Using \gls{rsa} again is now automatically shortened.

\subsection{Data-driven models}
Even though the majority of fouling models are mechanistic, it is important to review some of the data-driven modelling approaches as well. Typically, machine learning is used to fit a model to the experimental data.
If sufficient data are available and precautions are taken to avoid overfitting, this type of ``black-box" modelling can lead to good results.\\ \\
In \citet{Shetty2003} an artificial neural network was built to predict fouling of flat and spiral-wound nanofiltration membranes. The fouling is quantified in terms of the total resistance to water permeation $\gls{Rt}$, described by Darcy's law (Eq.\ \ref{Darcy}). The input layer consists of different operational and feed water quality parameters that are typically monitored during municipal wastewater treatment, e.g., influent flow rate, permeate flux, operation time, feed water quality parameters, etc. The neural network was trained on an extensive amount of data from different ground and surface water filtration experiments using different membranes to predict the total hydraulic resistance $\gls{Rt}$. In this way, a model was build that predicts membrane fouling in various operational conditions. \par
Artificial neural networks excel at describing complex nonlinear relationships between input and output variables \citep{Tu1996} and are therefore a popular data-driven approach to model membrane fouling with numerous applications described in literature. A summary of these studies can be found in \cite{Mirbagheri2015}. \\ \\  %ergens heb ik ooit eens gelezen wat de sterkte is van ANN en waarom ze hier gebruikt wordne volgens mij iets te maken dat het goed ANN 
\cite{Dalmau2015} propose the use of model trees in predicting membrane fouling, which combines linear regression with decision trees. Linear regression is a simple technique resulting in a model with high bias but low variance, prone to underfitting nonlinear data. In contrast, decision trees capture nonlinear patterns in the data, giving rise to models with low bias and high variance, making it prone to overfitting \citep{Dalmau2015}. The combination of both approaches leads to decision trees where each ``leaf" is a linear model. The model tree developed in the article consists of 35 multivariate linear equations. Each equation predicts the \gls{TMP} in various operating conditions. Model trees are capable of partially explaining the system, unlike many other data driven methods \citep{Dalmau2015}. \\ \\
Data-driven models can also be applied from a different point of view, that is membrane state monitoring or fouling mechanism prediction. The former is elaborated by \cite{Maere2012} and uses principal component analysis in combination with clustering to monitor the fouling behaviour of \gls{MBR}. A distinction is made between three different membrane states; clean, reversibly fouled and irreversibly fouled, allowing for a real-time decision on possible maintenance actions. A similar method is developed in \cite{Drews2009} where the dominant fouling mechanism is identified by fitting  different models to the data, each describing different fouling mechanisms. The dominant fouling mechanism follows from the best fitting model. Figure \ref{foulingID} presents the results of this approach. 

\begin{figure}[H]
\begin{center}
\hspace{-1.5cm}
\includegraphics[width=0.7\textwidth]{figs/foulingIDB.PNG}
\caption{Comparison between experimental and simulated  flux/time curves \citep{Drews2009}.\label{foulingID}}
\end{center}
\end{figure}

\section{Profilometry}
%The current model still has a few shortcomings. The generation of the flow profile is only performed once, at the beginning of the simulation. Consequently, this static profile does not change during simulation. As fouling progresses, the filtration resistance increases and for a constant pressure filtration this means that the permeate flux declines over time. Secondly, one can expect the velocity profile at the membrane surface to change as the cake layer grows disproportionally in space. Periodically adjusting the liquid flow profile would be a rational improvement of the model. \par 
An important part of this thesis is the development of an experimental setup for the calibration and validation of the spatio-temporal model for filter cake formation. %Currently, the attachment of the particles to the membrane and filter cake is realised with an empirical equation that is modulated with a parameter that needs calibration ( \textbf{zie sectie...}). After calibration, the model performance will be assessed through validation. Calibration and validation are traditionally performed by gathering data of the flux and TMP during a filtration experiment. 
Membrane filtration models are mostly calibrated and validated with data on the \gls{TMP} and flux during operation. However, considering that the main goal of this model is the characterisation of filter cake formation mechanisms, it is meaningful to gather data about the filter cake properties for a goal-directed calibration/validation. This will be accomplished by a profilometric characterisation of the filter cake. \\ \\ 
Profilometry, surface metrology, surface topography, etc.\ are different terms used in literature to more or less describe the same process i.\ e.\ the three-dimensional characterisation of a surface. The distinction between these terms lies in subtleties that are not relevant for this study. These terms are therefore regarded as interchangeable. \par
Profilometric techniques can be classified in numerous categories, according to the characteristics of the technique. In order to achieve a simple and straightforward classification, a distinction is made between optical methods and non-optical methods. It is nonetheless important to keep in mind that the main goal of this overview is to find a suitable technique for the calibration/validation of the model and not a perfect classification of the profilometric techniques.
\subsection{Non-optical methods}
The prominent method in this category is \emph{\gls{SP}}. In stylus profilometry, the surface is characterised by the interaction of a sensing tip with the sample. The vertical displacement of the tip is recorded while the stylus is moving across the sample's surface \citep{Lonardo2002}. %The tip size plays a critical role in the resolution of the profilometer \citepLonardo2002}). 
Two main disadvantages of stylus profilometry can be identified \citep{stout2000,Lonardo2002}. Firstly, this method is generally quite slow, taking a long time to characterise a small area of the specimen. Secondly, the contact between the instrument and sample can result in the deformation of the sample and an underestimation of the height of soft surfaces.  Consequently, stylus profilometry is not suitable for the surface characterisation of a filter cake as it most certainly classifies as ``soft". \par
\emph{\Gls{AFM}} is generally not regarded as stylus profilometry because there is no contact with the surface, since it is characterised via repulsive forces exterted on the sensing tip. Nevertheless, the other working principles are quite similar. Atomic force microscopy is mostly used for submicron measurements and its nanometer scale resolution would be excessive for filter cake measurements, which are in the micrometer scale \citep{Search1997}. \par
In \emph{\gls{STM}} a metal tip scans the surface of the sample. A voltage is applied over the gap between the tip and the specimen. When the conducting tip is close to the surface,  electrons will bridge the gap between the surface and the tip, resulting in a current. Changes in surface height result in changes in magnitude of the current that are subsequently recorded. STM is used for the surface characterisation on an atomic level and is therefore not a suitable technique for this research \citep{Binnig1982,Hansma1987}.\par
%\todo{afbeelding misschien wel handig hier?}
\emph{\Gls{SEM}} utilises a beam of electrons to form a three-dimensional image of the investigated surface. This beam is produced in an electron gunner or electron emitter, accelerated by a set of anodes and focused on the specimen by a series of electromagnets. On collision, one part of the electrons will reflect of the surface, another part excites the atoms of the specimen, thereby producing secondary electrons, and other electrons penetrate the sample producing X-ray radiation. Both types of electrons and the X-rays can be captured by specialised detectors giving rise to a profilometric image when the electron beam is moved over the surface of the sample in a scanning motion \citep{Reimer2013,sem2}.
%https://books.google.be/books?id=j5nsCAAAQBAJ&printsec=frontcover&dq=scanning+electron+microscope&hl=nl&sa=X&redir_esc=y#v=onepage&q=scanning%20electron%20microscope&f=false
\subsection{Optical methods}
The number of optical methods for the three-dimensional characterisation of a surface is vast. It is impossible to discuss all these techniques, therefore a selection is made of the most distinct types in order to still give a comprehensive overview. \\ \\
Interferometry is based on the interaction of multiple light beams and uses the superposition of waves to gather information about the surface characteristics \citep{Hariharan}. The number of measurement techniques based on interferometry is enormous. However, the working principle of all these techniques is basically the same. A beam of light, is split by a glass plate with a semi-reflective coating; one beam acts as a reference and is directly reflected via a mirror to the detector while the second beam is reflected on the surface of the sample. %A compensator plate is introduced to correct for the fact that one beam crosses the splitter thrice and the other only once. 
These two beams are recombined by the beamsplitter and will interfere. Figure \ref{interferometer} presents a schematic overview of such a generic interferometer. Constructive interference is observed when the path length is the same for both beams, giving rise to the interference signal with the largest amplitude and hence the highest intensity. The signal intensity drops with bigger path differences (the difference cannot exceed the wavelength). Consequently, an image is obtained with different intensity values for each pixel. By altering the path length of one of the beams it is possible to scan for other heights in the sample and a profilometric image is formed. For these kind of measurements, white light is mostly used as it produces more accurate results than monochromatic light. \par
\emph{\Gls{WLI}} has a resolution limit of approximately 0.5\,$\mathrm{\mu m}$ due to diffraction effects \citep{Conroy2006,Hariharan}. 
Problems can arise from the presence of thin films in the specimen that cause errors in the measurements \citep{Conroy2006}. Hence, the presence of an aqueous layer in the filter cake poses a potential risk for its profilometric characterisation. \par
% \todo{WN: Misschien even opzoeken hoe dik een waterfilm rond slibpartikels is?, BDj niet gevonden}
\begin{figure}[H]
\begin{center}
\hspace{-1.5cm}
\includegraphics[width=0.5\textwidth]{figs/interferometer.png}
\caption{The basic outline of an interferometer \citep{Search1997}.} 
\label{interferometer}
%https://books.google.be/books?id=sWbGSSQ6fPYC&pg=PR1&lpg=PR1&dq=Basics+of+Interferometry,+Second+Edition.&source=bl&ots=zXvmd_7pNt&sig=PE1YIfLl179kpaDoTfvV6ar-8AE&hl=nl&sa=X&ved=0ahUKEwix75WE9svJAhVDVBQKHSM6ApQQ6AEIVzAI#v=onepage&q=Basics%20of%20Interferometry%2C%20Second%20Edition.&f=false}.\label{foulingID}}
\end{center}
\end{figure}
%Misschien nog uitwerken omtrent de compensator plate
%direct technique?
  %nadeel en voordeel van deze techniek \cite{Conroy2006} p460  % enige bron die deftig het werkingsprincipe uitlegt is http://www.polytec.com/int/solutions/3-d-surface-profiling/basics-of-white-light-interferometry/
% https://books.google.be/books?id=sWbGSSQ6fPYC&pg=PR1&lpg=PR1&dq=Basics+of+Interferometry,+Second+Edition.&source=bl&ots=zXvmd_7pNt&sig=PE1YIfLl179kpaDoTfvV6ar-8AE&hl=nl&sa=X&ved=0ahUKEwix75WE9svJAhVDVBQKHSM6ApQQ6AEIVzAI#v=onepage&q=Basics%20of%20Interferometry%2C%20Second%20Edition.&f=false https://books.google.be/books?hl=nl&lr=&id=oBw7AAAAIAAJ&oi=fnd&pg=PR13&dq=H.+Steel,+Interferometry&ots=GcK3r5DCXK&sig=rrYM-kd6cBsAq3PhEIMBLiPnlNI#v=onepage&q=H.%20Steel%2C%20Interferometry&f=false
\emph{\Gls{DICM}} or Nomarski microscopy also adopts the principles of interferometry, but both the reference and sample ray go through the sample in two adjacent points. The difference in height of those points translates in a phase shift between both rays. The rays are recombined and the resulting interference is proportional to the difference in path length of the rays. A differential image is obtained of the surface \citep{Lang}. \par
Other interferometry related methods are \emph{\gls{DHM}} \citep{Kemper2007a} and \emph{\gls{OCT}}  \citep{Podoleanu2012}. The working principle of these techniques is very similar to \gls{DICM} and will therefore not be elaborated. \par
% And whatever this technique is:  https://www.osapublishing.org/view_article.cfm?gotourl=https%3A%2F%2Fwww%2Eosapublishing%2Eorg%2FDirectPDFAccess%2FBE2C8123-AB46-92F6-5908192D5578E43C_86034%2Foe-13-22-8693%2Epdf%3Fda%3D1%26id%3D86034%26seq%3D0%26mobile%3Dno&org=
%Het staal moet doorzichtig zijn dus dat is voor onze toepassing niet haalbaar: het membraan zou er afmoeten en dit vergt teveel manipulatie van het staal. Kan weerlegt worden want het principe kan ook toegepast worden op een systeem waarbij gewerkt wordt met reflectie ipv erdoor te gaan \cite{Lang}
  \emph{\Gls{CLSM}} is basically a conventional light microscope with the ability to illuminate a small section of the sample. The reflected light from the sample is filtered through a pinhole that filters out all the out-of-focus light. Hence, it is possible to illuminate a certain viewing depth, which is not possible with a conventional light microscope.  The combination of images from different depths gives rise to a three-dimensional structure \citep{Hocken2005}. This method is technically not a profilometric technique as it actually visualises the internal structures of a specimen. The maximal atainable resolution of confocal laser scanning microscopy is about the same as that of a conventional light microscope \citep{PawleyJBandMasters1996}. %Confocal laser scanning microscopy is part of the focus detection techniques which also includes \emph{focus variation} \citep{focVar}, \emph{intensity detection} and \emph{contrast detection} \citep{Toru2015}. \par JB: weglaten
  The use of \gls{CLSM} as a tool for the profilometric characterisation of surfaces is restricted by its low scanning rate and the problems that arise with the occurence of different optical properties in one specimen \citep{Cha2000}.  %http://emerald.ucsd.edu/Docs/3Dprofilometry.pdf
\emph{Pattern projection profilometry} or structured light profilometry establishes a three-dimensional profile by projecting light patterns on the object and capturing the distortion of the pattern, from a different angle, with a camera. From an angle, the differences in height of the object results in a phase shift of the projected pattern (Figure \ref{ZHanger}). The captured image is subsequently analysed to calculate the underlying phase distribution. This technique is, along with the interferometric techniques, ``indirect'' which means that calibration is needed to map the phase distribution to height measurements \citep{Gorthi2010}. 
Pattern projection profilometry includes \emph{\gls{FPP}} \citep{Zhang2010} and \emph{\gls{FP}} \citep{Su2001}. Also \emph{\gls{MP}} is considered as pattern projection profilometry. This technique projects two light gratings on the object, but the underlying principles are the same \citep{moire}.
\begin{figure}[H]
\begin{center}
\hspace{-1.5cm}
\includegraphics[width=0.5\textwidth]{figs/fringeprojection.png}
\caption{Illustration of the principle of a digital fringe projection profilometer as an example of the working principle of pattern projection techniques \citep{Zhang2010}.} 
\label{ZHanger}
\end{center}
\end{figure}

In \emph{\gls{WLAC}}, a white light source is passed through a concave objective lense. The different refractive indices for each wavelength result in a dispersion of the light beam and each wavelength is refocused at a different distance from the lense. The confocal configuration assures that only the in-focus wavelength reaches the detector for which the wavelength is determined and the corresponding height of the sample is obtained. This technique is insensitive to ambient light and stray reflection, there is no need for vertical scanning to sense the surface height and it is a direct technique. For these reasons, this technique seems to be ideal for the profilometric characterisation of a filter cake \citep{Leach2011,nanovea}. A schematic outline of such a system is presented in Figure \ref{wlccp}.   
\begin{figure}[]
\begin{center}
\hspace{-1.5cm}
\includegraphics[width=0.5\textwidth]{figs/Chromaticconfocal.png}
\caption{Schematic representation of a white light axial chromatic confocal profilometer \citep{nanovea}.} 
\label{wlccp}
\end{center}
\end{figure}
The lateral resolution limits and relative prices of the different profilometric techniques are denoted in Table \ref{tab:prof}.
% \todo[inline]{IN: Tabel 2.2 niet echt op zijn plaats in de literatuurstudie, WN \& JB: We vinden dit wel echt een meerwaarde}
%\todo[inline]{dat is onze microsoop zie wouter zijn samenvatting}
%\todo[inline]{resolution nog eens kwantificeren van sommige technieken staat het in Search1997} 
%\todo[inline]{tabel techniek classification, resolution, price, remarks\\ for stm the surface needs to be conducting \citepSearch1997})} 
% \todo[inline]{IN: lijkt me  ook niks voor literatuurstudie}
\begin{table}[H]
  \caption{Overview of the lateral resolution and price class of the profilometric thechniques. References: \cite{Song} [1], \cite{sem} [2], \cite{Conroy2006} [3], \cite{Kemper2007a} [4], \cite{PawleyJBandMasters1996} [5], \cite{nanovea} [6] \label{tab:prof}}
  \begin{flushleft}
    \begin{tabular}{lllllc}
      \toprule 
      & & \textbf{resolution} & \textbf{price} &  \textbf{remarks} \\
      \midrule
      \large{\textbf{Non-optical}} \\
      & \gls{SP} & \unit{0.5}{\micro\metre} \textsuperscript{[1]} & \$ & dependent on the stylus size  \\ 
      & \gls{AFM}& \unit{0.1}{\nano\metre} \textsuperscript{[1]} & \$\$ \\
      & \gls{STM}& \unit{0.1}{\nano\metre} \textsuperscript{[1]} & \$\$  \\
      & \gls{SEM}& \unit{1}{\nano\metre} \textsuperscript{[2]} & \$\$\$ \\
      \toprule
       \large{\textbf{Optical}} \\
      & \textbf{Interferometry} &  & & diffraction limited\\
      & \gls{WLI} & \unit{0.5}{\micro\metre} \textsuperscript{[3]}& \$ \\ 
      & \gls{DICM} & \unit{0.5}{\micro\metre} & \$   \\ % diffraction limited
      & \gls{DHM} & \unit{0.5}{\micro\metre}& \$ & good axial resolution (5 \unit{\nano\metre}) \textsuperscript{[4]} \\ 
      & \textbf{Focus detection} \\
      & \gls{CLSM} & $<$ \unit{0.5}{\micro\metre} \textsuperscript{[5]} & \$\$ & close to diffraction limit \\
      & \textbf{Pattern projection} & & & supermicron measurements \\
      & \textbf{Other} \\
      & \gls{WLAC}  & \unit{1}{\micro\metre} \textsuperscript{[6]} & \$ \\
      \bottomrule
    \end{tabular}
  \end{flushleft}
  \label{tab:multicol}
\end{table}
%\subsection{Optical methods}
%\subsubsection{Interferometry}
%An interferometer is any optical configuration that causes light of finite coherence [finite coherence nog uitleggen] to take more then one path through the system, and where the light paths subsequently cross or combine with a path-length difference. Interferometry can be subdivide into two distinct classes, wavefront-splitting interferometry and amplitude-splitting interferometry. \par
%In wavefront-splitting interferometry the optical paths are not in parallel and cross, in contrast to amplitude splitting where a single optical path splits to take different paths before crossing or combining. [meer uitleg nodig]{Optical source: interferometry for Biology and Medicine}
%
%
%\subsubsection{White light Axial chromatism}
%\subsubsection{Differential interference contrast microscopy}
%\subsubsection{Focus detection methods}
%\subsubsection{Pattern projection method}
%\subsubsection{Confocal Microscopy}
%To understand the principle of confocal microscopy, the basic working principle of light microscopy needs to be briefly ellaborated. \par
%
%In light microscopy, objects are enlarged or magnified with a convex lens that bends light
%rays by refraction. Diverging rays from points within the object (object points) are made to
%converge behind the convex lens and cross over each other to form image points (i.e., a
%focused image). In the compound microscope there are usually two magnifying systems in tandem, one defined by the objective and the other defined by the eyepiece \cite{Keller2006}.
%[Afbeelding]
%
%In a confocal microscope the conventional microscope condenser is replaced by a second objective lens. A pinhole, which limits the field of view, is positioned on the microscope axis. The field of view is further limited by a second pinhole in the image plane placed confocally [woord verklaren] to the illuminated spot in the specimen and to the first pinhole. With this methodology light scattered from parts other than the illuminated point on the specimen is rejected. 
%The specimen is scanned with a point of light by moving the specimen over short distances in a raster pattern [HandBook of Biological confocal].
%
%[Afbeelding invoegen HandBook of Biological confocal microscopy figuur1.1]
%[kan nog iets vermeld worden over het feit dat dit niet echt profilometry is omdat je doorheen je sample kijkt]
%
%
%\subsection{Contact and pseudo-contact methods}
%\subsubsection{Stylus profilometry}
%In stylus profilometry the surface is characterised by the vertical movement of a tip, which follow the profile of the surface. This movement is acquired as a function of the horizontal displacement. It is trivial that the tip size plays a critical role in the resolution of the profilometer.
%\par
%An important disadvantage of stylus profilometry is the long time required to acquire data. [Zoeken naar enige tijd indicatie van hoelang dit is]
%stylus profilometers are available in portable format, which can be important in the conduction of field experiments\cite{Lonardo2002}.
%\par 
%These types of profilometers tend to be robust and have fundamentally superior measurement accuracies in many engineering environments. stylus profilometers can be designed with nanometer level sensitivity, has an inherent immunity to dusty or wet surfaces and can measure either metallic or nonmetallic materials. \cite{Bauza2006} Stylus profilometry is a direct technique.
%\subsubsection{Atomic Fore Microscopy (AFM)}
%In an atomic force microscope a small needle, also referred to as the tip, interacts with the sample's surface. The tip moves over the surface of the sample and is deflected by the interaction forces betwaan the tip atoms and the sample atoms.The tip is attached to a cantilevered spring which is capable of sensing these small forces. The sensed forces are transduces to generate molecular images\cite{Lal1994}. 
%\par  
%The needle can be operated in contact mode where the needle touches the sample or the needle can oscillate at a finite distance from the surfaces. This mode is called the noncontact mode. The noncontact mode has the advantage that is doesn't disrupt the surface.\cite{Lal1994}
%\par
%In both modes a deflection force is``perceived" by the tip. This deflection force needs to be translated into a detectable signal. Often a optical system is used where a laser beam reflects off the cantilever. The reflected beams are captured and converted into electrical signals by position-sensitive photo detectors. A more detailed explanation of this mechanism can be found in \cite{Lal1994}.
%\par 
%A more recent mode of operations is called ``tapping mode", this is a combination of the contact and noncontact mode. The cantilever is oscillated (noncontact) but also comes into contact with the surface of the sample (contact){Lal1994}.
%\par
%Other modes exist but fall outside the scope of this dissertation. 
%
%\subsubsection{Scanning Tunneling Microscopy}
%Oude techniek ?
%
%\subsubsection{Electron microscopy}
%In electron microscopy a destinction can be made between transmission electron microscopy (TEM), scanning electron microscopy, reflection electron microscopy and scanning transmission electron microscopy.
%
%
%
%
%
%


%----------------------------------------------------------------------------------------------------------
\clearpage
\clearpage{\pagestyle{empty}\cleardoublepage} 
\hyphenation{}
\chapter[Spatio-temporal model of filter cake formation]%
{Spatio-temporal model of filter cake formation \label{spatModel}}
The general framework for the spatio-temporal model of filter cake formation, established by \cite{Ghijs2014}, has an Euler-Langrangian stucture with two model layers (Figure \ref{fg:modellayers}). The dispersed phase is modelled through a Lagrangian approach where each particle and its modelled quantities (velocity, forces, etc.\ ) are individually tracked by a moving frame of reference (\gls{ABM}). The continuous phase is modelled through computational fluid dynamics for which a stationary, Eulerian reference frame is adopted.
This chapter will first summarise the assumptions of the model developed by \cite{Ghijs2014} and justify the revision of these assumptions to account for the model extensions. In the remainder, each of the model layers will be elucidated in order to have a thorough understanding of the model before it is extended  in Chapter \ref{ModelDev}.

\begin{figure}[]
    \centering
    \includegraphics[width=0.7\textwidth]{figs/modellayers.png}
    \caption{Schematic representation of the model layers \citep{Ghijs2014}.}
    \label{fg:modellayers}
\end{figure}

\section{Assumptions \label{assumptions}}
This study aims to develop a model that describes filter cake formation in \gls{MBR}s as realistically as possible. It is nevertheless necessary to simplify certain processes in order to reduce the computational demands and avoid the overcomplication of the model \citep{bender2000}. Therefore, assumptions were made in \cite{Ghijs2014} about the nature of the particles and the system in which they are modelled. Some of these assumptions are well-founded and will be retained here, while others are revised and the model is extended with the necessary processes to account for the extra complexity. \\ \\
A first assumption is that all particles in the system are \textbf{rigid} and \textbf{perfect spheres}. This originates from the force balance on the particles as some of these equations are derived for rigid, perfect spheres in a flow field. This assumption might seem too simplistic, but in a bottom-up modelling approach this is a good starting point. It is also assumed that all particles are of the \textbf{same size with diameter \gls{partDia}}. For most of the filtration processes the dispersed phase is made up of particles of different sizes and shapes and monodisperse solutions are rare. Moreover, polydispersity has a significant impact on filter cake formation and this assumption is therefore relaxed in this study. The corresponding model extension involves a considerable programming effort as the implementation was strongly relying on the assumption of monodispersity. This extension is discussed in Chapter \ref{ModelDev}. %Secondly, the way in which the filter cake is build up needs to be revised. The current model implementation also depends on fixed particle diameters, therefore a new collision-detection algorithm needs to be worked out. \par
\par
There is \textbf{no interaction} of free moving particles \textbf{in the bulk phase}. Hence, there are no collisions and free moving particles do not exert forces on each other. Coagulation of free moving particles is consequently not considered. This assumption is justified, given the fact that this model is constructed for laminar flows and the number of particle collisions in the bulk fluid is limited.\par
The system is modelled in \textbf{two dimensions}, the forces and velocities describing the movement of the particles in the bulk phase and the \glspl{PDE} governing the fluid behaviour are described in the $x$- and $y$-direction. In \cite{Ghijs2014} it was assumed that adding a third dimension would not impact the simulation results significantly so for computational efficiency this dimension was left out. This assumption is also revised in this study as simulation results in two dimensions showed an unrealistic ``piling'' of the particles in the filter cake. The addition of a third dimension might resolve this issue and a system modelled in three dimensions is more generic. \par
The \textbf{flow profile} of the fluid \textbf{is only computed once} at the beginning of the simulation. This assumption implies a steady-state flow of the continuous phase throughout the entire simulation, so that the formation of filter cake on the membrane surface has no effect on the liquid flow profile. This ``one-way'' assumption is further discussed in Chapter \ref{ch:disc} as it is expected that the formation of filter cake greatly influences the liquid flow in close proximity to the filter cake \citep{Picioreanu2009}. This, in turn, will have an impact on the filter cake formation and for the accurate representation of this process, the model needs to be extended with a bidirectional coupling between the two model layers. \par
Concerning filter cake formation as such, particles attach with a probability that is inversely proportional to their absolute velocity (section \ref{sec:FCForm}) and attached particles \textbf{cannot detach}, regardless of the shear stresses induced by the continuous movement of liquid over the filter cake surface. This shortcoming is partially compensated by the abovementioned adhesion probability. However, a complete force balance that supports detachment of particles from the cake layer will lead to more realistic simulation results (Chapter \ref{ch:disc}). \par
The \textbf{flux through the membrane is assumed to be constant} in space and time. As the formation of a filter cake layer on top of the membrane results in an increased filtration resistance, the flux is expected to change locally according to the fouling intensity in constant flux membrane filtration systems. The flux will be higher in ``cleaner'' regions and lower in heavily fouled regions. Hence, it is necessary to extend the model to account for this, which is discussed in Chapter \ref{ch:disc}. %the flux decrease due to the increased filtration resistance, to account for this process.

\section{Dispersed phase \label{sec:dispPhase}}
The Lagrangian model of the dispersed phase consists of a system of differential equations that describes a force balance of the free moving particles. The particle's velocity is determined at every time step and is expressed relatively with respect to the fluid velocity, i.e.\ $\gls{Ur}=\gls{Up} - \gls{Uc}$, with \gls{Up} [\unit{\metre\,\reciprocal\second}] the relative velocity of the particle and $\gls{Uc}$ [\unit{\metre\,\reciprocal\second}] the local fluid velocity. Newton's second law is used to determine the particle velocity and follows from \citep{Worner2003,Ghijs2014}:
\begin{equation}
 m_\mathrm{p} \, \frac{d}{dt} \, \gls{Up}(t)= \vecb{F}_\mathrm{surf} + \vecb{F}_\mathrm{body},
 \label{newton2}
\end{equation}
with $m_p$ [\unit{\kilogram}] the particle mass. The force balance is divided in body forces $\vecb{F}_\mathrm{body}$  [\unit{\newton}] that are distributed over the entire volume of the particle and surface forces $\vecb{F}_\mathrm{surf}$ [\unit{\newton}] that act on the surface of the particles (Figure \ref{forcebal}). For a particle immersed in a fluid, gravity \gls{Fgrav} [\unit{\newton}] is the only body force. The surface forces originate from \citep{Worner2003,Ghijs2014}:

\begin{itemize}
 \item $\vecb{F}_\mathrm{arch}$ [\unit{\newton}] the Archimedes force;
 \item $\vecb{F}_\mathrm{p}$ [\unit{\newton}] the force induced from the pressure gradient \gls{nabP};
 \item $\vecb{F}_\mathrm{hydr}$ [\unit{\newton}] the resulting hydrodynamic force, which consists of:
 \begin{itemize}
  \item $\vecb{F}_\mathrm{drag}$ [\unit{\newton}] the drag force that minimises the difference between fluid velocity and the particle velocity;
 \item $\vecb{F}_\mathrm{am}$ [\unit{\newton}] the added mass force that accelerates the fluid surrounding the particle;
 \item $\vecb{F}_\mathrm{hist}$ [\unit{\newton}] the history force, which originates from the lagging fluid boundary layers surrounding the accelerating particle;
 \item $\vecb{F}_\mathrm{lift}$ [\unit{\newton}] the lift force, describing shear lift that is caused by the intertia effects in the viscous flow around the particle.
  \end{itemize}
 \end{itemize}

\begin{figure}[H]
    \centering
    \def\svgwidth{0.7\columnwidth}
    \input{figs/forcebalance.pdf_tex}
    \caption{Schematic representation of the forces that act on free moving particles in a fluid, a negative \gls{Ur} is assumed for the direction of the forces \citep{Ghijs2014}.}
    \label{forcebal}
\end{figure}

The total force balance on the particle can be written as:

\begin{equation}
 \vecb{F}_\mathrm{tot}=\vecb{F}_\mathrm{g}+\vecb{F}_\mathrm{arch}+\vecb{F}_\mathrm{p}+\vecb{F}_\mathrm{drag}+\vecb{F}_\mathrm{am}+\vecb{F}_\mathrm{hist}+\vecb{F}_\mathrm{lift} \, .
 \label{totalForcebal}
\end{equation}
 Now, the acceleration of the particles can be derived from Newton`s second law (Eq.\ \eqref{newton2}). However, this approach does not take into account the effects of stationary fluid boundaries at the membrane surface. \cite{Faxen1922} proposes the addition of a correction factor to include these effects. Introducing the Fax\'en correction factor in Eq.\ \eqref{totalForcebal} results in the Maxey-Riley equation for the overall force balance on a particle submersed in a fluid \citep{Maxey1983}:

 \begin{equation}
\begin{split}
	m_\mathrm{p}\frac{d}{dt}\mathbf{U}_p(t) & = -3 \pi \eta d_\mathrm{p} \left( \mathbf{U}_\mathrm{r} - \frac{1}{24}d_\mathrm{p}^2 \nabla^2 \mathbf{U}_\mathrm{c} \right) + \left( m_\mathrm{p} - V_\mathrm{p} \rho_\mathrm{f} \right) \mathbf{g} \\
	 & - \frac{1}{2} V_\mathrm{p} \rho_\mathrm{f} \left( \frac{d \mathbf{U}_\mathrm{r}}{dt} - \frac{1}{40} d_\mathrm{p}^2 \frac{d}{dt}\left( \nabla^2 \mathbf{U}_\mathrm{c}\right) \right) \\
	 & - V_\mathrm{p} \nabla  p - \frac{3}{2} \sqrt{\pi \eta \rho_\mathrm{f}} d_\mathrm{p}^2  \int_{\displaystyle 0}^t \frac{d\mathbf{u}(\tau)/d\tau}{\sqrt{t-\tau}}d\tau \\
	 & - 1.615 \rho_\mathrm{f} d_\mathrm{p}^2 \left( \mathbf{U}_\mathrm{r} - \frac{1}{24}d_\mathrm{p}^2 \nabla^2 \mathbf{U}_\mathrm{c} \right) \sqrt{\mu_\mathrm{f} \left|\mathbf{\kappa} \right|} \; \mathrm{sgn}\left(\mathbf{\kappa} \right),
	 \label{maxeyRiley}
\end{split}
\end{equation}
 with $ \mathbf{u}  = \mathbf{U}_\mathrm{r} - \frac{1}{24}d_\mathrm{p}^2 \nabla^2 \mathbf{U}_\mathrm{c}$ in the history term, \gls{Vp} [\unit{\cubic\metre}] the particle volume and $\kappa$ [\unit{\reciprocal\second}] the fluid velocity gradient defined as,
 \begin{equation}
  \kappa_\mathrm{i}= \frac{d \mathbf{U}_\mathrm{c,j}}{di}
 \end{equation}
where $i$ and $j$ represent directions in a two-dimensional Cartesian coordinate system.
  \section{Filter cake formation}
 By solving Eq.\ \eqref{maxeyRiley} one can accurately simulate the behaviour of the dispersed particles for a given fluid flow profile. %The filtration imposes a lateral movement of the continuous phase, this momentum is transferred to the particles by means of the drag force \textbf{$\vecb{F}_\mathrm{drag}$}. and the particles are consequently ``pulled'' towards the membrane.
 The first step in modelling the filter cake formation is the detection of collisions of particles with the already formed cake layer or the membrane surface. This is accomplished with a collision detection algorithm which is detailed in Chapter \ref{ModelDev}. To ensure that not all collisions of the particles with the filter cake result in adhesion, this process is governed by a probability. This function defines the probability \gls{PUp} [\,-\,] for a particle to attach to the filter cake in the event of a collision and is inversely proportional to the particle`s momentum:
 \begin{equation}
	P(\gls{Up}) = e^{-k\, \gls{Up}},
	\label{eq:stickychance}
\end{equation}

with \gls{k} [\,-\,] a parameter representing the ``stickiness'' of the particles which needs to be determined experimentally.
Equation \eqref{eq:stickychance} is the only empirical equation and does not fit in the model's mechanistical mindset. Its role, however, is temporary and will become redundant when a full force balance is in place for particles in the filter cake (chapter \ref{ch:disc}).

\section{Continuous phase}
This model layer describes the behaviour of the continuous phase as it flows continuously into the filtration unit via the inlet and is discharged at the outlet and through the membrane pores.
\subsection[Fluid dynamics]{Fluid dynamics}\label{sec:FD}
The continuous phase is modelled through an Eulerian approach, the control volume is included in a stationary reference frame and the flow quantities such as velocity and pressure are described for each fixed point inside this control volume. The flow pattern and fluid velocities of the continuous phase are obtained by solving the mass, momentum and energy conservation equations, written as a set of \glspl{PDE}. %\gls{CFD} is the scientific domain that uses numerical methods to solve these equation in order to \textbf{simulate} fluid flow, heat transfer or other phenomena such as chemical reactions.
Depending on the fluid characteristics of the modelled system, such as the rheology and flow regime (turbulent/laminar flow, single phase/multiphase system), different sets of \glspl{PDE} are used \citep{Versteeg1995,boekMS}.\\ \\
The first equation describes the conservation of mass within a given volume \gls{Omega} [\unit{\cubic\metre}] and is based on the following axiom: the rate of increase of mass in a fluid element is equal to the net rate of flow into the fluid element since no mass can be created or destroyed, assuming incompressibility. So, %Versteeg1995

\begin{equation}
\frac{\partial \rho_\mathrm{f}}{\partial t} + \nabla ( \rho_\mathrm{f} \mathbf{U}_\mathrm{c}) = 0, \label{masscon}
\end{equation}

with $\mathbf{U}_\mathrm{c}$ [\unit{\metre\,\reciprocal\second}] the fluid velocity in the volume $\Omega$ and $\rho_\mathrm{f}$  [\unit{\kilogram\,\rpcubic\metre}] the density. The compressibility is negligible for most liquids and $\rho$ can be assumed as constant.  Consequently, for incompressible fluids \citep{Versteeg1995}:
\begin{equation}
\frac{\partial \rho_\mathrm{f}}{\partial t} = 0,
\end{equation}
and
\begin{equation}
 \nabla ( \rho_\mathrm{f} \mathbf{U}_\mathrm{c} ) = \rho_\mathrm{f} \nabla \mathbf{U}_\mathrm{c},
\end{equation}
so that
\begin{equation}
 \nabla \mathbf{U}_\mathrm{c} = 0.
\label{simpmasscon}
\end{equation}

The conservation of momentum, as a second equation, is governed by a force balance acting on $\Omega$.
By taking into consideration the body forces such as gravity and the surfaces forces such as the compressive force and friction force and inserting them into Newton's second law of motion, one obtains:
\begin{equation}
 \rho_\mathrm{f} \left( \frac{\delta \mathbf{U}_\mathrm{c}}{\delta t} + \mathbf{U}_\mathrm{c}\cdot \nabla \mathbf{U}_\mathrm{c} \right) = - \nabla p + \nabla \cdot \mathbf{\tau}  + f,
\label{momcon}
 \end{equation}

with \gls{tau} [\unit{\newton}] the viscous stress tensor and $f$ [\unit{\newton\,\reciprocal\metre}] the external body forces. In most of the cases, $f$ only consists of the gravitational force \citep{Versteeg1995,boekMS}. \par
For an incompressible, Newtonian fluid the viscous stresses are proportional to the rates of deformation and constant viscosity can be assumed. Consequently, the stress tensor $\tau$ in \mbox{Eq.\ \eqref{momcon}} can be written in terms of the viscosity $\mu_\mathrm{f}$ and fluid velocity, leading to the Navier-Stokes equation for incompressible flow \citep{Versteeg1995},
\begin{equation}
 \rho_\mathrm{f} \, \left( \frac{\delta \mathbf{U}_\mathrm{c}}{\delta t} + \mathbf{U}_\mathrm{c}\cdot \nabla \mathbf{U}_\mathrm{c} \right) = - \nabla p + \mu_\mathrm{f} \, \nabla^2 \mathbf{U}_\mathrm{c} + f \, .
\label{simpmomcon}
\end{equation}

For this study, an isothermal flow is considered and the equations for the conservation of energy are not relevant and are therefore not discussed. The set of \glspl{PDE} \eqref{simpmasscon} and \eqref{simpmomcon} is called the Navier-Stokes system for incompressible flow \citep{Versteeg1995}.

\subsection{Computational fluid dynamics}
The Navier-Stokes equations provide an accurate theoretical basis for modelling dynamic fluid behaviour. However, these equations are complex, non-linear and coupled, making the solution quite complex and in practice more often than not, analytical solutions are non-existent. It is therefore necessary to use numerical methods in order to obtain a solution for the modelling problem at hand, which is the scientific domain of \gls{CFD} \citep{boekMS}.
\begin{quote}
\textit{``Computational fluid dynamics is a set of numerical methods applied to obtain approximate solutions of problems of fluid dynamics and heat transfer.''}\\
\vspace{-0.75cm}\flushright{--- \citep{Zikanov2010}}
\end{quote}
Depending on the characteristics of the fluid in the system, different fluid dynamics equations need to be used. The Navier-Stokes equations \eqref{simpmasscon} and \eqref{simpmomcon} are derived for a Newtonian, incompressible fluid in an isothermal system which is a good representation for modelling wastewater and other liquids used in membrane filtration systems. \par %\todo{opmerking JB die ik niet kan lezen}
After selecting the appropriate flow equations and boundary conditions, the equations are discretised either using \glspl{FVM} or \glspl{FEM}. The \gls{FEM} is not used in this work and is therefore not elucidated.\par
In the \gls{FVM}, the domain is subdivided into a finite number of cells with a certain volume, where the values of the different variables are integrated values of their respective cells, stored in the center points of the cells. The discretised \glspl{PDE} are obtained by constructing the conservative form for which volume integrals are used for the conservative components and surface integrals of the divergence terms \citep{boekMS,Patankar1980}. For a scalar property $\phi$, the discretised transport equations can be written as follows:

\begin{equation}
\int_t^{t+\Delta t} \left( \int\limits_V \frac{\delta(\rho\, \phi)}{\delta t} dV + \int\limits_S \rho\, U \, \phi \cdot n \, dS = \int\limits_S \Gamma \, \nabla \phi \cdot n \, dS + \int\limits_V S_\phi \, dV \right)\, dt\, ,
%equation van boekMS
\label{eq:disctransp}
\end{equation}
with $\Gamma$ the diffusion coefficient and $Q_\phi$ the source of $\phi$.
The first term in Eq.\ \eqref{eq:disctransp} denotes a temporal derivative and is zero for the steady-state solution. The second term describes the convective flux of $\phi$ through the control volume. The third and fourth term, respectively, represent diffusion and the volumetric sources and sinks of $\phi$ that cannot go under the first three terms. A solution for the flow problem can be obtained by applying Eq.\ \eqref{eq:disctransp} to every cell in the discretised spatial domain, and utilising a numerical solving method \citep{boekMS}.

%\todo[inline]{addition: turbulence models}

%----------------------------------------------------------------------------------------------------------
\clearpage
\clearpage{\pagestyle{empty}\cleardoublepage}

\hyphenation{}
\chapter[Model development]%
{Model development \label{ModelDev}}
The profound study of other membrane fouling modelling approaches described in literature (Chapter \ref{litRev}) and the critical analyses of the assumptions and model framework proposed by \cite{Ghijs2014} (Chapter \ref{spatModel}) identified the processes that require further development and those that are not yet incorporated. The proposed model extensions include changes in the model framework, sensu strictu, and the development of accessory tools such as a \gls{GUI} for the analysis of the simulation results and post-processing. This chapter discusses the implemented model extensions and tools in-depth and is concluded by an overview of the model's architecture.

\section{Polydispersity}
% \gls{MBR} sludge is a complex mixture of suspended solids, \gls{EPS}, water and solutes. The suspended solids are mostly made out of the heterogeneous microbial community in the \gls{MBR} which consists of microorganisms in all sizes and shapes T\citep{Juretschko2002}. %Bigger aggregates can be formed due to the bioflocculation of these organisms and breakage can be induced by the shear regime in the \gls{MBR} \citep{Li2007}. %\todo{The EPS-matrix which consis(EPS matrix is very heterogeneous, in which a variety of polymeric materials have been found: carbohydrates, proteins, lipids and nucleic acids. In this work,) \textbf{bron}}
\cite{Broeckmann2006} demonstrate the influence of particle and membrane pore size distributions on cake layer formation and pore blocking. \cite{Meng2006} also identify the \gls{PSD} as an important factor that affects membrane fouling. Moreover, it is generally known that %the \gls{PSD} has an important impact on the filter cake structure, leading to changes in cake porosity. 
a heterogeneous mixture of particle sizes gives rise to a closer packing of particles in the filter cake, %and consequently a denser filter cake, 
which has an important impact on the filtration resistance. It is expected that a dense filter cake is more resilient towards liquid shear stresses and thus less prone to detachment. \par
It is clear that characterising the feed flow of filtration processes as a homogeneous and monodisperse suspension of perfect spheres is not realistic. The strong depency of the filter cake architecture and characteristics on the particles size demonstrates the importance of a more accurate representation of the dispersed phase. Extending the model towards polydispersity of the bulk phase is therefore a crucial step in establishing a realistic membrane fouling model. 
The extension of the model towards polydispersity is somewhat complicated as many processes described by the model, e.g.\ the attachment of particles, are dependent on the particle diameter and the model implementation often relies on the assumption of monodispersity. Hence, this modification required the revision and adaptation of some process implementations. \\ \\
The flocculation of free moving particles in the bulk phase is not considered by this model. Although, the inclusion of polydispersity makes it possible to account for particle aggregates by representing sludge flocs as one particle with a bigger diameter. It is also possible to model non-spherical particles as a multidisperse aggregate of spheres \citep{Hubbard1996}.
%The \gls{PSD} has been determined for \gls{MBR} sludge in \cite{Lee2003,Meng2006} and \cite{Wisniewski1998}.

\subsection{Particle sampling \label{sec:ParSamp}}
For a monodisperse stream of particles entering the membrane filter, the number of particles entering each time step $N$ [\,-\,] is given by,
\begin{equation}
 N=6\, \cfrac{\gls{Ucf}\, \gls{cb}\, A}{\gls{rhoS}\, \pi\, \gls{partDia}^3 }\, \Delta t \, ,
\label{eq:NumPartPerStep}
 \end{equation}
with $\gls{Ucf}$ [\unit{\metre\,\reciprocal\second}] the crossflow velocity at the inlet, \gls{A} [\unit{\metre\squared}] the inlet surface area and $\Delta t$ [\unit{\second}] the elapsed time per time step. \par
When modelling a multidisperse bulk phase, Eq.\ \eqref{eq:NumPartPerStep} does not hold as \gls{partDia} is not constant. In order to have a quantitative measure of incoming particles, this flow is expressed as a mass flow rate, %Indirectly, the quantity of entering particles can be expressed as the mass flow rate of particles,   
\begin{equation}
 \cfrac{dm}{dt} = \gls{Ucf}\, \gls{cb}\, A\, ,
\label{eq:incomMass} 
\end{equation}

with \gls{m} [\unit{\kilogram}] the mass of incoming particles. Multiplying Eq.\ \eqref{eq:incomMass} with the total simulation time $t_\mathrm{tot}$ [\unit{\second}] results in an estimated value of the total mass inflow during the entire simulation,
\begin{equation}
 m_\mathrm{tot} = \gls{Ucf}\, \gls{cb}\, A\,  t_\mathrm{tot} .
\label{eq:incomMass2} 
\end{equation}
The next step is the development of a sampling procedure to represent the multidisperse feed flow, characterised by a \gls{PSD}. Particles are sampled from this \gls{PSD} until the total mass, as determined by Eq.\ \eqref{eq:incomMass2}, is attained. \par
A \gls{PSD} depicts the relative amount of particles present in the bulk phase as a function of the particle size, based on volume or number. Experimentally determined \gls{PSD}s are based on a finite number of measurements, making \gls{partDia} a discrete variable for which the inverse \gls{CDF} sampling algorithm is the most efficient pseudo-random number generator, given a cumulative \gls{PSD}. %(Algorithm \ref{alg:Sampling}). 
This sampling procedure randomly generates a number in the interval [0,1] and searches in the given \gls{PSD} for the diameter value that corresponds to the probability value that is the closest to the generated number. It can be proven that for a large number of samples, this procedure results in the original \gls{PSD}.\\ \\
% \todo[inline]{Move x up a bin to the... and move x down a bin...: JB niet duidelijk wat dit juist doet BDJ: ik zou echt niet weten hoe ik dat anders moet schrijven}
% \todo[inline]{Bron voor algoritme neerschrijven? (wiskunde 4)}
% \todo[inline]{misschien algorithm 1 eruit en in woorden schrijven}
% \begin{algorithm}[H]
%  \SetAlgoLined
%  \KwData{Cumulative particle-size distribution $F(\gls{partDia})$}
%  \While{sampled mass\, \textless\, required mass}{
%   \vspace{0.1cm}
%   Simulate observation $y$ of $Y$ from the uniform distribution [0,1]; \\
%   Choose an initial value for $\gls{partDia}$, $x$;  \\ 
%   \eIf{$F(x) \leq y $}{
%    \While{$F(x) < y$}{ \vspace{0.1cm} Move $x$ up a bin to the next value of $x$; \\
%    Evaluate $F(x)$;
%    }}{\While{$F(x) > y$}{ \vspace{0.1cm} Move $x$ down a bin to the previous value of $x$; \\
%    Evaluate $F(x)$;}}
%   Use $x$ as the simulated observation of \gls{partDia};\\ 
%   Add $x$ to the previously sampled observations;\\
%   Calculate the sampled mass;\\
%  }
%  \caption{The inverse CDF sampling algorithm for sampling particles from a \gls{PSD}.\label{alg:Sampling}}
% \end{algorithm} \vspace{0.75cm}
Lastly, the number of particles that enters each time step is simulated stochastically. For a sufficiently small time step $\Delta t$, the average number of particles entering each step (Eq.\ \eqref{eq:NumPartPerStep}) is smaller than one. As it is not possible to introduce a non-integer number of particles, the latter represents the probability of a particle entering on a certain time step, this procedure is depicted in Algorithm \ref{alg:stoch}. \par
%For typical operational values of $\gls{Ucf}$ (\unit{1}{\metre\,\reciprocal\second}) and $C$ (\unit{10}{\kilogram\,\rpcubic\metre}), the number of particles entering is typically in the order of $10^{-1}$. More extreme operational conditions might give rise to numbers higher than one and this approach should be revised, but this is straightforward. \par

\begin{algorithm}[]
 \SetAlgoLined
 Calculate average number of incoming particles per time step; \\
 \For{$i=1$ \KwTo number of time steps}{
 \eIf{random number [0,1] $<$ number of particles per time step }{1 particle enters system;}{No particles enter system;}
 }
  \caption{Stochastic implementation of the number of particles entering the system.\label{alg:stoch}}
\end{algorithm}
% \vspace{0.5cm}
%\par In practice, the \gls{PSD} are provided in a text file which is loaded into matlab and transformed to a CSD PSD => CFD naar N samples
When considering the introduction of a dispersion into a tubular system with a fully developed parabolic flow, the particles should not be equally distributed across the inlet, since they would then accumulate in the regions with low velocities close to the boundaries and diminish in the middle regions with high velocities, creating a parabolic concentration profile. This phenomenon is shown in Figure \ref{fig:parSampComp} (b). In order to prevent such an unrealistic behavior, the inlet positions of the particles are determined by a sampling procedure so that the introduction of particles at a certain position is corrected for the local variation in the fluid velocity. This is accomplished by first normalising the parabolic velocity profile at the inlet so that it sums up to one, yielding a proper \gls{PMF}. The \gls{PMF} is integrated and the resulting \gls{CDF} can be used in the above-mentioned inverse CDF sampling procedure. However, this inverse CDF sampling procedure can be computationally demanding for discrete distributions with a lot of bins. The latter is relatively low for the PSD but for the velocity profile this number is too high, resulting in an inefficient sampling procedure which slows down the model considerably. Hence, a power law ($f(x)= a\, x^{b}$) is fitted to the \gls{CDF} and the inverse CDF is determined analytically (Figure \ref{fig:veloFit}).
This approach gives rise to a faster sampling procedure than its discrete counterpart. Figure \ref{fig:parSampComp} shows the simulation of particles entering a fully developed parabolic flow with and without the parabolic sampling procedure. 
 
 \begin{figure}[H]
\centering
%\raisebox{0.5cm}{
\centerline{
\subfigure[]{\includegraphics[height=0.32\textwidth]{figs/ParSamp.png}}
\subfigure[]{\includegraphics[height=0.32\textwidth]{figs/nonParSamp.png}}
}
\caption{Simulated flow of particles close to the wall of a tube with a diameter of  \unit{8}{\milli\metre}. The incoming particles are sampled with (a) and without the velocity correction (b). \label{fig:parSampComp}}
\end{figure}

\begin{figure}[H]
 \begin{center}
   \includegraphics[width=0.65\textwidth]{figs/fitPara2.png}
 \caption{Comparison of the fitted power law ($f(x)= a\, x^{b}$) and the discrete CDF. \label{fig:veloFit}}
 \end{center}
 \end{figure}
%  \todo[inline]{nog een goodness of fit grootheid misschien gewoon in de legende}
\subsection{Filter cake formation \label{sec:FCForm}}
Another consequence of a variable particle diameter is the necessity to modify the collision detection algorithm, registering collisions of particles with the membrane or the filter cake. In \cite{Ghijs2014}, the filter cake is implemented in a discrete manner such that the filter cake thickness is tracked at a predefined longitudinal resolution. However, as the filter cake is built up out of spherical particles, the surface is highly irregular and curvy. In order to accurately represent this surface in a discrete way, the discretisation step should be at least one order of magnitude lower than the smallest particle diameter. Hence, storing filter cake thicknesses explicitely is only computationally efficient when considering a thick cake layer. Furthermore, for the evaluation of certain filter cake characteristics such as porosity, segregation and the \gls{PSD}, it is better to store the position and the diameter of the filter cake particles, instead of the thickness. The filter cake formation algorithm was thus completely revised to step away from this explicit and discrete representation. This required the development of a collision detection algorithm to detect collisions between spheres instead of collisions between spheres and a surface; this procedure is denoted below.\\ \\
Tracking all collisions within the system is computationally demanding and would be responsable for a large part of the computing time. In order to keep the simulation time reasonable, it is important to implement this procedure as efficiently as possible. First, a boundary layer $b$, whose thickness is the sum of the maximal filter cake thickness $\Delta l_\mathrm{max}$ and maximal particle diameter $d_\mathrm{p,max}$, is considered (Figure \ref{coldet}). Only particles within this boundary layer are close enough to the filter cake and are checked for collision. Next, the deposition trajectory $\vecb{V}$ is constructed between the particle's old and new position along with two other parallel vectors ($\vecb{c}_\mathrm{1}$ and $\vecb{c}_\mathrm{2}$), defining the search range for collision. These vectors represent the position of the biggest particles ($\gls{partDia}=d_\mathrm{p,max}$) that are just able to touch the depositing particle along its trajectory towards the membrane. Consequently, all particles centered outside this band are not able to collide with the depositing particle and are not considered. Finally, collision is detected by evaluating Eq.\ \eqref{euclid} for a predetermined number of points along the deposition trajectory $\vecb{V}$.  %Attached particles with the maximum diameter, centered on $\vecb{c}_\mathrm{1}$ and $\vecb{c}_\mathrm{2}$
\begin{equation}
 \sqrt{(x_\mathrm{1}-x_\mathrm{2})^2 + (y_\mathrm{1}-y_\mathrm{2})^2} \leq (d_\mathrm{1}/2+d_\mathrm{2}/2), 
 \label{euclid}
\end{equation}
with $x_\mathrm{1}$, $x_\mathrm{2}$ and $y_\mathrm{1}$, $y_\mathrm{2}$ respectively the coordinates of the downward moving particle and the particle in the filter cake and $d_\mathrm{1}$, $d_\mathrm{2}$ the diameter of these particles. If Eq.\ \eqref{euclid} is valid for a node in the collision trajectory, a collision occurs and the particle will attach to the filter cake at that position, if the stochastic adhesion criterion Eq.\ \eqref{eq:stickychance} is fulfilled. \par %This procedure is illustrated in Figure \ref{coldet}. \par
\begin{figure}[H]
    \centering
    \def\svgwidth{0.7\columnwidth}
    \input{figs/collisionDetPlainComplex.pdf_tex}
    \caption{Schematic representation of the collision detection algorithm. The white particles are outside the collision boundaries and are not evaluated for collision.}
    \label{coldet}
\end{figure}
To summarise, the new collision detection algorithm is more accurate due to the continuous and implicit representation of the filter cake and its performance is improved by constraining the detection of collisions to a boundary layer and limiting the search for possible collisions in a narrow zone spanning the collision search band. 

%\todo[inline]{addition: equations for the collision detection boundaries}

\section{Extension to a three-dimensional model \label{sec:3D}}
The model extension from two dimensions to three dimensions is a logical step towards a realistic filter cake representation. Yet it %the addition of a dimension
leads to a considerable increase in the number of modelled particles and one may wonder whether the added value of  process knowledge %of the obtained process knowledge 
%to question if the benefit of additional process knowledge gained by this transition %towards a three-dimensional model 
outweighs the extra computational burden. There are, however, two key reasons to chose for a three-dimensional model. \par
In \cite{Ghijs2014}, the simulations suffered from an unrealistic ``piling'' of particles in the filter cake. %as the particles mostly settle on the first piece of filter cake they encounter. 
Even with the possibility of an elastic collision, it was observed that most particles are caught by previously formed filter cake patches. Due to the lack of a third dimension, a piece of filter cake will catch all incoming particles and keep growing as there is no mechanism that counteracts this. %For lower values of $k$ this process will be a lot slower, but the end result will the same %as is demonstrated in 
It is hypothesised that by extending the model to three dimensions, a single piece of filter cake will have less impact on the further development of the fouling. Nonetheless, the increased dimensionality will most likely not entirely solve this issue; to do so, the model should be extended with a rolling algorithm. Such an algorithm is described in \cite{Smets} and simulates the three-dimensional rolling of spheres. To include this in the model, a transition to three dimensions is imperative. \\ \\
As can be expected, the flow of particles in a three-dimensional flow field is similar to that in a two-dimensional flow field. Still, it is necessary to modify some of the implemented processes. First, the force balance  (Eq.\ \eqref{maxeyRiley}) is extended with a third direction ($z$). The resulting force in this direction is not influenced by $\vecb{F}_\mathrm{g}$ and $\vecb{F}_\mathrm{arch}$ and can be written in terms of $\vecb{F}_\mathrm{p}$, $\vecb{F}_\mathrm{drag}$, $\vecb{F}_\mathrm{am}$, $\vecb{F}_\mathrm{hist}$ and $\vecb{F}_\mathrm{lift}$. \par
Secondly, the lift force in the $x$- and $y$-direction has to be extended with an additional gradient in the $z$-direction. In two dimensions, the inertial lift force has one component in both the $x$-direction and $y$-direction \citep{saffman1965}:
\begin{equation}
 \displaystyle \vecb{F}_\mathrm{lift,x} = - 1.615\, \rho_f\, d_p^2\, {U}_{r,x}\, \sqrt{\mu_f \left| \cfrac{d U_{r,x}}{dy} \right|} \, \mathrm{sign}\left(\cfrac{d U_{r,x}}{dy}  \right)\, , 
\label{eq:lift}
 \end{equation}
\begin{equation}
 \displaystyle \vecb{F}_\mathrm{lift,y} = - 1.615\, \rho_f\, d_p^2\, {U}_{r,y}\, \sqrt{\mu_f \left| \cfrac{d U_{r,y}}{dx} \right|} \, \mathrm{sign}\left(\cfrac{d U_{r,y}}{dx}  \right)\, ,
\label{eq:lift2}
 \end{equation}
with $U_{r,x}$ and $U_{r,y}$ the relative particle velocity in the $x$- and $y$-direction respectively . \par
In three dimensions, Eqs.\ \eqref{eq:lift} and \eqref{eq:lift2}
 %Eq.\ \eqref{eq:lift} shows that the direction of the lift force is perpendicular to the considered gradient and it is clear that these equations 
 need to be extended with a gradient in the $z$-direction.  %Therefore, these equations do not suffice for a three-dimensional systems as 
 The $x$-component of the lift force, for example, should now be written as:
\begin{equation}
\begin{split}
 \displaystyle \vecb{F}_\mathrm{lift,x} = - 1.615\, \rho_f\, d_p^2\, {U}_{r,x}\, \left[ \sqrt{\mu_f \left| \cfrac{d U_{r,x}}{dy} \right|} \, \mathrm{sign}\left(\cfrac{d U_{r,x}}{dy}  \right) \right.\\ 
 \left. + \sqrt{\mu_f \left| \cfrac{d U_{r,x}}{dz} \right|} \, \mathrm{sign}\left(\cfrac{d U_{r,x}}{dz}  \right) \right].
\end{split}
\label{eq:lift3D}
\end{equation}
With the correct force balance established, the model is now capable of describing the particle's motion in a three-dimensional flow field. \\ \\%The generation of this profile in three-dimensions proves to be even simpler than in two dimensions as this is where openFOAM is made for}. 
%As discussed in section \ref{sec:calVal} the entire \gls{MBR} is modelled in \gls{CFD}. However, for the agent-based modelling of the particles it is computationally impossible to consider the entire reactor. Hence, a small part of the \gls{CFD} simulation is extracted and used in the agent-based model (Figure \ref{}).   
%but the area of interest for the agent-based model is close to the membrane and most of the 
The adaptation of the filter cake formation process for three dimensions is straightforward. More precisely, the collision trajectory vector is described in three dimensions and the collision search boundaries are now represented by a cylinder. In spite of these changes, the main working principle remains the same.
% \begin{figure}[H]
%  \centering
% % \includegraphics{}
%  \caption{temp}
%  \label{}
% \end{figure}
\section{Parallelisation \label{sec:Parallel}}
As mentioned in Section \ref{sec:3D}, the extension to three dimensions comes at a higher computational expense. For a \unit{10}{\kilogram\,\rpcubic\metre} dispersion of \unit{50}{\micro\metre} particles ($\rho_\mathrm{p}$  = \unit{1000}{\kilogram\,\rpcubic\metre}) flowing in a two-dimensional tube (\unit{10}{\centi\metre} $\times$ \unit{10}{\milli\metre}), approximately $8 \times 10^{8}$ particles have to be processed each second of simulated time, excluding the particles in the filter cake. When considering a box with the same dimensions and a width of \unit{5}{\milli\metre}, this number increases to $8 \times 10^{10}$, which indicates the difficulty of maintaining a reasonable \gls{SCR}. For one processor, the \gls{SCR} is in the order of  $1 \times 10^{-4}$. The implementation of the filter cake formation, sampling procedures and force balance are already very efficient, leaving little room for optimisation. Hence, the best solution is to parallelise the implementation to enable simultaneous computation on multiple processors (cores). This is done by assigning every processor a part of the spatial domain, so that it performs the calculations for the particles present in that part. Every predefined number of time steps, information is exchanged between the cores and the particles that moved to a neighbouring part of the spatial domain are reallocated to the appropriate cores. In this way, a good \gls{SCR} ($\sim 1 \times 10^{-3}$) is obtained without the need of convoluted code optimisation.   %e.g. the inclusion of new particles in the filter 
%A subdivision in time is not possible due to the time dependency of the particle's position 

\section{Model architecture}
%Due to the implementation of a handfull of extensions and related adaptations of the model, its architecture has deviated from the original proposed by \cite{Ghijs2014}. 
The overall model architecture is depicted in Figure \ref{fig:ModelStruct}.
\begin{figure}[]
    \centering
    \def\svgwidth{\columnwidth}
    \input{figs/ModelStructureBendArrow.pdf_tex}
    \caption{Overview of the model architecture. The \gls{MBR} geometry, \gls{PSD} and the general simulation parameters are the inputs of the model. The output of the model is the filter cake structure, represented by the ($x$, $y$, $z$)-coordinates and diameter of the particles in the cake. }
    \label{fig:ModelStruct}
\end{figure}
The model takes three inputs, the first being the \gls{MBR} geometry. This geometry is defined in a Python script and converted to a suitable mesh by SALOME's Python interpreter. Then, this mesh is exported to OpenFOAM and the solution for the pressure and velocity fields are obtained using the simpleFOAM solver. These fields are subsequently imported to MATLAB. %by the \textit{``readFoamFiles''}-function.
The second input is the \gls{PSD} of the particle diameters in the feed flow. During the initialisation, the total mass of particles that will flow through the system during the simulation is estimated and sampled as discussed in Section \ref{sec:ParSamp}. The last input is a list of general simulation parameters, such as the concentration of particles \gls{cb}, the stickiness parameter $k$, the time window considered by the history force, etc. Based on these values, %information of the simulation parameters and the \gls{PSD} 
it is first checked if the calculation of the Stokes drag force is stable \citep{Ghijs2014}. If so, the main loop is entered. If not, an appropriate time step is suggested that leads to stable simulations.\par
In the main loop, the computational domain is partitioned and assigned to a number of processors (Section \ref{sec:Parallel}). Each processor caries out a calculation loop to simulate the dispersed phase. Each time step, new particles are added to the system, the new positions of all particles are determined and the filter cake formation is simulated. After a fixed number of time steps, the information of the processors is collected and the domain is repartitioned. This procedure is repeated until the predefined simulation time is reached. %The main loop is repeated until the final simulation time is met.

\section{Software}
\subsection{MATLAB}
MATLAB (v2015a, MathWorks), an abbreviation for matrix laboratory, is a high level matrix-based scripting language licensed by MathWorks. This commercial software package has an extensive library of readily available functions, mostly for engineering and scientific applications \citep{MathWorks2016}. MATLAB plays a central role in this master thesis, it is the backbone of the model linking all the different processes, software and files and acts as a basis for the \gls{GUI}.  %gathers the information from OpenFOAM and performs the calculation of the \gls{ABM} and 

\subsection{Python}
Python (v2.7.3, Python Software Foundation) is an open source, high-level programming language that is more general-purpose than MATLAB. Python is licensed by the Python Software Foundation and comes with a vast number of modules, both from the standard library and community-contributed libraries, making it widely applicable \citep{Foundation2016}. For this master thesis, Python is mainly used as scripting language for the automated generation of meshes through SALOME's Python interpreter.

\subsection{SALOME \label{sec:SALOME}}
SALOME (v7.6.0, OPEN CASCADE) is an open source software for the pre- and post-processing of numerical simulations. It is used as a pre-processor for the generation of the geometry and mesh before using \gls{CFD}. SALOME provides an \gls{API} for Python that enables scripting of the geometry and mesh, providing a generic workflow for the creation of variable geometries and meshes. SALOME is licensed by OPEN CASCADE \citep{Cascade2016}.

\subsection{OpenFOAM \label{sec:OpenFOAM}}
OpenFOAM (v2.2.2, OpenFOAM foundation) (Open Source Field Operation and Manipulation) is an open source software package that provides a vast number of solvers and utilities for the calculation of continuum mechanics \citep{OpenFOAM-Foundation2016}. OpenFOAM is written in C++ and has no \gls{GUI}, all manipulations are done by changing files and running applications from the command line. Two solvers are used for this master thesis, the simpleFOAM solver is a steady-state solver for turbulent and laminar flow and the pimpleFOAM solver is a large time-step transient solver for turbulent flow.
%\todo[inline]{addition: paraview}
% hier kan er nog altijd wet uitgewijd worden over de BC enz. zoals bij Michael
\section{Post-processing user interface}
This advanced mechanistic membrane fouling model aims at identifying the key mechanisms of membrane fouling in filtration processes. In addition to the accurate modelling of membrane fouling, it is important that the simulation results are analysed critically. Therefore, a \gls{GUI} was developed as post-processing tool that enables, amongst other things, a detailed visualisation of the filter cake and force balance (Figure \ref{fig:GUI}). The prominent utilities of this \gls{GUI} are elaborated below. \par
By clicking the \textit{``Play''} button [1] the in silico filtration experiment is displayed in real-time speed. 
With the \textit{``Track''} button [2], one can track individual particles throughout the simulation, while useful information about the particle's velocity, acceleration, size, etc.\ is displayed at [2a]. While tracking a particle, the force balance for the two main directions ($x$,$y$) is shown in two top right plots [2b]. This information is very useful for determining the main driving forces of filter cake formation under various operational conditions. The button \textit{``Detail''} [3] visualises the filter cake at a user-defined position in the form of a two-dimensional projection. The user is able to ``scroll'' through the filter cake along the length of the membrane for a visual inspection. A three-dimensional visualisation of the filter cake can be requested by clicking \textit{``3D''} [4]. 
\textit{Diameter coloring} [5] assigns a different color to each particle size which can be convenient for the visualisation of certain phenomena, such as particle size segregation in the cake layer due to preferential sedimentation (Section \ref{sec:QualVal}). The \gls{PSD}s of the feed flow, filter cake, dispersed phase and the entire system (control) can be visualised by checking of \textit{``PSD''} [6]. 
Finally, by clicking
\textit{``Save''} [7] all the information of the detailed filter cake is stored in a MATLAB workspace variable for further manipulation outside the \gls{GUI}.

\begin{sidewaysfigure}
    \centering
    \def\svgwidth{\columnwidth}
    \input{figs/GUI.pdf_tex}
    \caption{Graphical user interface for post-processing. Note that the bottom right plot of the filter cake is a two-dimensional projection, there is no overlap of particles.}
    \label{fig:GUI}
\end{sidewaysfigure}

%  \begin{figure}[H]
%     \centering
%     \begin{subfigure}[]{}
%         \centering
%         \todo[inline]{enkel hier en op de laatste de schaal toevoegen, hier ook inlet aanduiden}
%         \includegraphics[width=0.95\textwidth]{CFDCylboxCyl.png}
%     \end{subfigure}\\ 
%     \begin{subfigure}[]{}
%         \centering
%         \todo[inline]{deze moet ingezoomd worden}
%         \includegraphics[width=0.85\textwidth]{shortCFD.png}
%     \end{subfigure} \\
%     \begin{subfigure}[]{}
%         \centering
%         \includegraphics[width=0.85\textwidth]{wedgeSmall.png}
%     \end{subfigure} \\
%         \label{fig:CFDMBR}
% \end{figure}
%     
% \begin{figure}[H]   
%     \begin{subfigure}[]{}
%         \centering
% %         \includegraphics[width=0.85\textwidth]{wedge2InletBot.png}
% \todo[inline]{(d) moet nog weg deze is precies dezelfde als hierboven en zou het laatste schuine design moeten zijn met perfecte stroming}
%     \end{subfigure}    
%     \begin{subfigure}[]{}
%         \centering
%         \includegraphics[width=0.85\textwidth]{FinalCFD.png}
%         \todo[inline]{fliplr, de inlet zit hier rechts}
%         \label{fig:finalCFD}
%     \end{subfigure}
%     \caption{bla}
% \end{figure}
%misschien die coloring gebruiken: http://www.paraview.org/Wiki/images/8/89/Beginning_filters_7.png

%\textbf{boekMS p22-25 zal handig zijn hier}

%\textbf{boekMS p22-25 zal handig zijn hier}




%----------------------------------------------------------------------------------------------------------
\clearpage
\clearpage{\pagestyle{empty}\cleardoublepage} 
\hyphenation{}
\chapter[Results and discussion]%
{Results and discussion \label{ch:results}}

\vspace{-2cm}
\begin{quote}
	One sometimes finds what one is not looking for. \\
\textit{	 --- Sir Alexander Fleming
}\end{quote}
\vspace{2cm}

This chapter presents the results of this work and is structured in two major sections. In the first section (\ref {sec:calVal}), the model-based design of a laboratory scale microfiltration unit is discussed. This unit will be used to collect data for the calibration and validation of the model, and it is of utmost importance that a structured flow is achieved near the membrane. In \mbox{Section \ref{sec:casestud}}, a scenario analysis is performed to assess the movement of particles and the filter cake formation under different operational conditions. First, a description of the modelled system and the assumptions are discussed and the benchmark operational parameters are disclosed. Next, a qualitative validation of the bulk phase force balance is performed in order to assess the accuracy of the predicted movement. In the end, the spatial segregation of the particles and the effect of polydispersity is investigated.  
\section{Calibration and validation: model-based design of a \mbox{filtration} unit \label{sec:calVal}}
In order to determine the value of the adhesion parameter $k$, which can be interpreted %\todo{note JB die ik niet kan lezen}
as the stickiness of the filter cake, a calibration has to be performed. Furthermore, as the objective of this research is the characterisation of filter cake formation, it is important to assess the model's accuracy through validation. Both calibration and validation require the acquisition of experimental data. For that reason, a laboratory scale, crossflow microfiltration unit was designed. Such an experimental setup enables full control of the operational conditions and measured quantities during filtration. Moreover, in addition to tracking typical filtration variables such as the transmembrane pressure and flux, it is possible to use profilometric techniques for the acquisition of specific filter cake properties such as the cake thickness, surface roughness and other, more advanced features for a goal-directed calibration/validation. These filter cake properties are not readily available in literature and can provide valuable insight in filter cake formation. \\ \\
%The ideal shape of the filtration device that induces a good flow regime for a facilitated calibration and validation was obtained through model-based design with \gls{CFD}. 
In order to facilitate the calibration/validation of the model, an experimental filtration unit was designed with \gls{CFD} that leads to an ``ideal'' flow behaviour which is postulated in the following three design criteria.\par
The geometry should, first of all, stimulate laminar flow as turbulence is characterised by unstructured and chaotic flow which, in modelling terms, is a lot harder to describe than the structured laminar flows. Furthermore, it will be a lot easier to unravel the fouling mechanisms taking place in the filtration device, when turbulence effects can be neglected. It is important to keep in mind that this model serves as a proof-of-principle of the physical laws and the scenarios under consideration do not necessarily have to be realistic. \par
The flow regime is characterised by the ratio of inertial forces to viscous forces, represented by the Reynolds number Re [\,-\,] \citep{boekMS}:
\begin{equation}
Re = \cfrac{L\, U_r}{\mu/ \rho} \, , 
\label{eq:Reynolds}
\end{equation}
with $L$ [\unit{\metre}] the characteristic length. Flow is said to be turbulent for $Re > 4000$, laminar for $Re < 2300$ and transient for $ 2300 < Re < 4000 $. The \emph{first design criterion} is thus a laminar flow with a Reynolds number below 800 in the membrane compartment. %, excluding the in- and outlet. 
This conservative Reynolds number is instituted because it is only an indication of the flow regime and not a sufficient criterion, therefore it is to be used with caution. Turbulent flows can further be induced by rough surfaces, inlet phenomena and other processes that are not encompassed by the Reynolds number \citep{Taylor2006}. \\ \\
The \emph{second design criterion} involves the minimisation of dead volumes within the filtration device. The fluid velocity approaches zero in these regions which will lead to an increased local fouling rate and a smaller effective volume through which there is flow, resulting in higher Reynolds numbers than calculated from Eq.\ \eqref{eq:Reynolds}. \\ \\
The \emph{third and last design criterion} states that recirculation streams are to be avoided. This criterion is introduced so that a slice of the MBR can be used as the representation of the full system, lowering the computational demands of the model. Such an assumption is not valid when recirculation patterns occur inside the geometry. \\ \\
For the membrane configuration, a flat sheet membrane was selected for two reasons; the shape of the filtration device for this membrane type is less bound to geometric constraints compared to other membane configurations and this configuration eases removal of the membrane for profilometric characterisation without perturbing the filter cake. This is much harder (if not impossible) to accomplish with e.g.\ tubular or spiral wound membranes. \par
%In order to facilitate the profilometric characterisation of the filter cake, a flat membrane configuration is chosen. It is a lot easier than other configurations such as tubular and spiral membranes. 
The membrane used in all designs has a width of \unit{10}{\centi\metre} and a length of \unit{30}{\centi\metre}, based on the measureable surface. Figure \ref{fig:schemMBR} shows a schematic outline of the filtration device and all components. The membrane sheet is positioned on the bottom of the filtration device [2]. \par
\begin{figure}[H]
    \centering
    \def\svgwidth{\columnwidth}
    %\input{figs/MBR.pdf_tex}
    \input{figs/MBRWithArrow.pdf_tex}
 \caption{Schematic representation and dimensions of a potential experimental filtration device with inlet [1], membrane [2], membrane compartment [3] and outlet [4]. The arrow represents the movement of liquid from inlet to outlet and through the membrane. \label{fig:schemMBR}}
 \label{fig:schemMBR}
\end{figure}
With the body of the filtration unit established, the effect of different in- and outlet structures was evaluated by means of \gls{CFD}. Although the flow in the membrane compartment should be laminar due to the chosen crossflow velocity, it is still possible for the inlet to have a turbulent regime. Hence, the turbulent \texttt{simpleFOAM} and \texttt{pimpleFOAM} solvers were used to obtain accurate results of the internal velocity and pressure fields. The latter were analysed for each design and the in- and outlet structures were modified with targeted adjustments in order to obtain a design that meets all the abovementioned design criteria. \\ \\
The first inlet structure that was evaluated is a standard cylindrical hose with a diameter of \unit{2}{\centi\metre}. Figures \ref{fig:CADMBR-1} (a) and \ref{fig:CADMBR-1} (b), respectively, show the three-dimensional overview and the sideview for this geometry, while the fluid streamlines are depicted in Figure \ref{fig:CFD-1} (a). The streamlines represent the tangent of the velocity field in each point and describe the path of a massless fluid element in the velocity field.
% deze laatste zin is van wikipedia maar ik vind hier geen andere bron van: https://en.m.wikipedia.org/wiki/Streamlines,_streaklines,_and_pathlines#Applications
This figure clearly shows that such a configuration leads to a considerable recirculation along with the establishment of dead zones in the corners of the membrane compartment. Furthermore, the small cylindrical inlet induces turbulent swirling (which is not visible on figure \ref{fig:CFD-1} as it is a time dependent phenomenon). In order to avoid recirculation, stagnation and turbulent swirling, the transition from the inlet to the membrane compartment should be more gentle. Therefore, more gradual inlet and outlet geometries were considered. \par 
\begin{figure}[]
    \centerline{
    \subfigure[]{\def\svgwidth{0.47\columnwidth}
		 \input{figs/cylboxOverviewClippedVec.pdf_tex}}
    \subfigure[]{\def\svgwidth{0.47\columnwidth}
		\input{figs/cylBoxSideClippedVec.pdf_tex}}
    }
      
     \centerline{
    \subfigure[]{\def\svgwidth{0.47\columnwidth}
		 \input{figs/WedgeShortOverviewClippedVec.pdf_tex}}
    \subfigure[]{\def\svgwidth{0.47\columnwidth}
		 \input{figs/shortWedgeSideClippedVec.pdf_tex}}
    }
    \centerline{
    \subfigure[]{\def\svgwidth{0.47\columnwidth}
		 \input{figs/WedgeSquareLastCirc.pdf_tex}}
    \subfigure[]{\def\svgwidth{0.47\columnwidth}
		 \input{figs/WedgeSquareSmallSideClippedVec.pdf_tex}}
    }
    \centerline{
    \subfigure[]{\def\svgwidth{0.47\columnwidth}
		 \input{figs/wedgeoverviewClippedVec.pdf_tex}}
    \subfigure[]{\def\svgwidth{0.47\columnwidth}
		 \input{figs/wedgeSideClippedVec.pdf_tex}}
    }
    \centerline{
    \subfigure[]{\def\svgwidth{0.47\columnwidth}
		 \input{figs/MBRFinalOverviewClippedVec.pdf_tex}}
    \subfigure[]{\def\svgwidth{0.47\columnwidth}
		 \input{figs/MBRFinalSideClippedVec.pdf_tex}}
    }
   
    \caption{Tested laboratory scale filtration device geometries with \gls{CFD}: three-dimensional geometry (left), sideview (right) \label{fig:CADMBR-1}}
\end{figure}   
    
% \begin{figure}[H]    
%     \captcont{Evolution of the laboratory scale \gls{MBR} geometry (continued). Overview (left), sideview (right) \label{fig:CADMBR-2}}
% \end{figure}

\begin{figure}[]
 \centerline{
     \subfigure[]{ \includegraphics[width=0.65\textwidth]{figs/FixedCFD/cylbox.png}}
 }
 \centerline{
     \subfigure[]{ \includegraphics[width=0.65\textwidth]{figs/FixedCFD/CFDshort.png}}
 }
\centerline{
     \subfigure[]{ \includegraphics[width=0.65\textwidth]{figs/FixedCFD/lastcirc.png}}
 }
 \centerline{
     \subfigure[]{ \includegraphics[width=0.65\textwidth]{figs/FixedCFD/wedgesmall.png}}
 }
\centerline{
     \subfigure[]{ \includegraphics[width=0.65\textwidth]{figs/FixedCFD/FInal.png}}
 }
\caption{Velocity and streamlines in the different designs tested in Figure \ref{fig:CADMBR-1} (top view). The arrow indicates the inlet of the filtration device.}
\label{fig:CFD-1}
\end{figure}

The next design had a funnel-shaped inlet/outlet structure (Figures \ref{fig:CADMBR-1} (c), (d)) and should result in a less aggressive entry of the feed flow. Figure \ref{fig:CFD-1} (b) indicates that this design has a flow pattern that is less chaotic, while turbulent swirling is absent. However, there is still a considerable recirculation of the fluid and presence of dead volumes. Hence, the feed flow still disturbs the overall fluid dynamics too much. \par
In an attempt to get rid of the recirculation, the inlet length was doubled from 10 to \unit{20}{\centi\metre} (Figures \ref{fig:CADMBR-1} (e), (f)). Figure \ref{fig:CFD-1} (c) shows the streamlines for this configuration. Although there is a substantial improvement, it still does not meet the requirements set by the design criteria. By using an inlet with a rectangular cross section rather than a square, it is possible to obtain a flow profile without recirculation (Figures \ref{fig:CADMBR-1} (g), (h) and Figure \ref{fig:CFD-1} (d)), but these changes introduce two stagnant regions at the entrance of the membrane compartment. 
Finally, a suitable design was obtained by introducing a bigger inlet and reducing the height of the membrane compartment so to obtain a flat configuration (Figures \ref{fig:CADMBR-1} (i), (j)). This configuration also has the advantage of being a lot easier to manufacture. Since this geometry generates a flow profile that fulfills all design criteria (Figure \ref{fig:CFD-1} (e)), it will be constructed and used for the experimental measurements. The technical drawings for this design are given in Appendix \ref{ch:tech}.

% \begin{figure}[H]
%  \captcont{CFD simulations of the \gls{MBR} designs proposed in Figure \ref{fig:CADMBR-1} (top view) (continued). The arrow denotes the inlet of the \gls{MBR}.}
% \label{fig:CFD-2}
% \end{figure}

\section[Case studies]{Filtration model: case studies \label{sec:casestud}}
\subsection{Setup \label{sec:setup}}
In order to investigate the impact of the model extensions and to verify if the filter cake build-up occurs in a realistic way, a tubular membrane was chosen as subject of a scenario analysis. This geometry has the advantage of being symmetrical, it gives rise to a parabolic flow profile and enables the possibility of simulating a long length. Therefore, the tubular geometry was selected instead of the optimal geometry according to the results reported in Section \ref{sec:calVal}. The tubular membrane had an inner diameter of 8\,mm and a length of 60\,cm. The feed was introduced at the channel inlet and a constant flux was instituted through the porous wall (membrane). A crossflow velocity and flux of respectively \unit{0.04}{\metre\, \reciprocal\second} and 36\,LMH ($1 \times 10^{-5}$\,\unit{\metre\,\reciprocal\second}) were chosen as benchmark values. The \gls{PSD} of the feed flow was experimentally determined with a particle size analyser (DIPA 2000) on a sample of \gls{MBR} sludge from municipal wastewater (Figure \ref{fig:PSDplot}) and the number average diameter, \unit{40.82}{\micro\metre}, was used as benchmark particle size for monodisperse simulations. \cite{Ghijs2014} demonstrated that high values of $k$ result in a slow filter cake formation while low values induce an increased piling of the particles. Therefore, $k = 2$ results in a good trade-off between the rate of filter cake formation and the roughness of the cake. The other benchmark parameters are summarised in Table \ref{tab:benchmark}. \par
\begin{figure}[H]
\centering
 \includegraphics[width=1\textwidth]{figs/PSDPlot.png}
 \caption{Experimentally determined particle size distribution of a sample of \gls{MBR} sludge from municipal wastewater, obtained with a particle size analyser (DIPA 2000). \label{fig:PSDplot}}
\end{figure}
\begin{table}[H]
\centering
    \caption{Benchmark parameter values used in the filtration model\label{tab:benchmark}.}
    \begin{tabular}{lll}
      \toprule
      Parameter &  & Value \\
      \midrule
      \gls{Ucf}  & crossflow velocity &  \unit{0.04}{\metre\,\reciprocal\second}\\ 
      \gls{J}  & flux &  36\,LMH\\ 
      \gls{rhoS}  & density of the suspended solids &  \unit{1.003}{\kilogram\,\rpcubic\metre}\\ 
      \gls{rhof}  & density of the continuous phase &  \unit{1.000}{\kilogram\,\rpcubic\metre}\\ 
      \gls{fluidKin} & kinematic viscosity of the continuous phase  &  \unit{1.004}{\metre\squared\,\reciprocal\second}\\ 
      \gls{k}  & adhesion parameter &  \unit{2}{\second\,\reciprocal\metre} \\ 
      \gls{partDia}  & number average particle diameter &  $40.82 \times 10^{-6}$ \unit{}{\metre}\\ 
      \gls{cb}  & concentration of suspended solids &  \unit{10}{\kilogram\,\metre\rpcubed}\\ 
      \bottomrule
    \end{tabular}
  \label{tab:multicol}
\end{table}
%Due to %computational constraints and the 
Due to the symmetry of the tubular membrane, it can be represented as a cuboid with a thickness of 0.6\,mm (Figure \ref{fig:tube}). Such a representation does not take into account the effects of the curvature. However, this effect is assumed negligible as the height of the cuboid is only 2\,\% of the cylinder circumference and the fluid dynamics are consequently computed in two dimensions. \par 
In order to simulate the flow fields in OpenFOAM, a mesh was generated with a resolution of 0.6\,cm in the $x$-direction and 0.04\,mm in the $y$-direction, resulting in a $300 \times 100$ mesh with $30,000$ cells. Next, the boundary faces were defined on the mesh and the boundary conditions were imposed. The inlet and membrane face were defined as the \texttt{patch}-type which is able to account for a flux. Dirichlet boundary conditions were imposed of \unit{0.04}{\metre\,\reciprocal\second} for the $x$-velocity at the inlet, $1 \times 10^{-5}$\,\unit{\metre\,\reciprocal\second} for the $y$-velocity at the membrane and \unit{0}{\metre\,\reciprocal\second} for all other directions, assuming zero slip at the walls. For the pressure, Neumann zero gradient boundary conditions were set at the inlet, membrane, and symmetry plane. The outlet was also defined as the \texttt{patch}-type with a zero gradient Neumann boundary condition for the velocity and a Dirichlet boundary condition of atmospheric pressure, assuming free outflow. The inside face was defined as a \texttt{symmetryPlane}-boundary implying a periodic boundary condition for the velocity, while the front and back faces are set to \texttt{empty} to indicate a two dimensions simulation. The Reynolds number for these conditions is 320, which is well within the laminar region. \par
The steady-state solution for this flow problem was obtained via the \texttt{simpleFoam} solver which is primarly used for turbulent, incompressible Newtonian flow but is also valid under laminar conditions.
% \todo{WN: Wouter bedoel je niet slice. BDJ: is cuboid niet professioneler?}

\begin{figure}[H]
\centering
 \def\svgwidth{0.8\columnwidth}
 \input{figs/tubularMembrane.pdf_tex}
 \caption{Three-dimensional representation of the modelled system, boundary faces and their respective fluxes, denoted by the arrows. The boundary faces consist of inlet [1], outlet [2], membrane [3], inside [4], front and back [5]. \label{fig:tube}}
\end{figure}


The steady-state solution for the relative kinematic pressure $\psi$ [\unit{\metre\squared\,\second\rpsquared}], $x$-velocity and $y$-velocity is displayed in Figure \ref{fig:CFDTube}.
% pressure calculated from p_open relates to p_real through p_real = p_open*rho + 101325
The absolute pressure shows a linear decrease from $101,371$\,Pa to atmospheric pressure, $101,325$\,Pa. Concerning the fluid velocity, a transition from the uniform profile at the inlet to a parabolic profile is observed. Due to the zero-slip boundary condition at the membrane, the fluid is moving slower near the walls and the incoming fluid is forced towards the middle section. After this transition, the $x$-velocity exhibits a fully developed parabolic flow and the $y$-velocity remains quasi constant at $-1 \times 10^{-5}$ \unit{}{\metre\,\reciprocal\second}. In order to facilitate the interpretation of the results, all simulations performed in the remainder of this chapter are based on the fully developed middle section of the tubular membrane (30\,cm). The involvement of inlet phenomena will only complicate the interpretation and are removed from the equation. Lastly, it is important to note that all simulations employ the benchmark parameter values as given in Table \ref{tab:benchmark}, unless indicated otherwise.
\begin{figure}[H]
    \centerline{
    \subfigure[]{\includegraphics[width=1\textwidth]{figs/pressuretubeAdapt.png}}
    }
    \centerline{
    \subfigure[]{\includegraphics[width=0.5\textwidth]{figs/xvelotubeAdapt.png}}
    \subfigure[]{\includegraphics[width=0.5\textwidth]{figs/yvelotubeAdapt.png}}
    }
\caption{(a) Relative kinematic pressure field in the slice of the tubular membrane (\unit{60}{\centi\metre}). (b) $x$-velocity in the first 2.5\,cm of the modelled system. (c) $y$-velocity in the first 2.5\,cm of the modelled system. \label{fig:CFDTube}}
\end{figure}
% \todo[inline]{JB: eigenlijk zou je hier de distance to the inlet in de figuur moeten weergeven}
\subsection{Mesh independence of the continuous phase \label{sec:MeshIndep}}
The mesh size is a critical factor that defines the accuracy of the \gls{CFD} simulations. In order to achieve accurate results, the mesh size should be sufficiently fine. However, the accuracy increases asymptotically and there is a refinement level beyond which there is no further significant effect of the mesh size. At that level, mesh independency is attained \citep{Roache1997}, which is to be pursued. \par
To determine the independency of the mesh, three meshes of different sizes were constructed; Mesh A had a cell size of \unit{20}{\milli\metre} by \unit{40}{\micro\metre}, mesh B a cell size of \unit{60}{\milli\metre} by \unit{200}{\micro\metre} and mesh C a cell size of \unit{60}{\milli\metre} by \unit{400}{\micro\metre}. These meshes were expected to cover a good range in order to indicate mesh independency. The $x$- and $y$-velocity were subsequently analysed for several cross sections along the length of the tube. The velocity profiles at $x$ = \unit{15}{\centi\metre} are shown in Figure \ref{fig:xmeshindep}. \par
It can be seen that the velocity profiles of mesh A and B are close to identical whereas a considerable difference is noticeable for mesh C. As a result, one can conclude that mesh A attains great accuracy, but is too fine and hence computationally inefficient, while mesh C is too coarse, giving rise to inaccurate results. Therefore, mesh B provides the best trade-off between accuracy and speed and is the ideal mesh size for the geometry at hand. Figure \ref{fig:xmeshindep}  also indicates that the discrepancies increase with the velocity magnitude, making it possible to use a coarser mesh in low velocity regions. However, due to the imposed flux at the membrane, the regions with a low $x$-velocity are also the regions with a high $y$-velocity. Hence, a non-uniform mesh should be a fine mesh near the membrane, a coarser mesh in the middle and finer mesh at the center of the tube. Due to the simple geometry and the fast convergence for mesh B, this non-uniform meshing strategy is considered as too complex for a minor efficiency gain, and was therefore not implemented. 
\begin{figure}[H]
\centerline{
    \subfigure[]{ \includegraphics[width=0.6\textwidth]{figs/meshIndepX.png}}
    }
\centerline{
    \subfigure[]{\includegraphics[width=0.6\textwidth]{figs/meshIndepY.png}}
    }
 \caption{Velocity profile of the velocities in $x$-direction (a) and $y$-direction (b) at $x$ = \unit{15}{\centi\metre} from the inlet for three meshgrids. Mesh A with a cell size of  \unit{20}{\milli\metre} by \unit{40}{\micro\metre}, mesh B with a cell size of \unit{60}{\milli\metre} by \unit{200}{\micro\metre} and mesh C with a cell size of \unit{60}{\milli\metre} by \unit{400}{\micro\metre}. \label{fig:xmeshindep}\label{fig:ymeshindep}}
\end{figure}
%
% particle groote 100µm checken
% mesh1000-200: \textit{tubular1LMH4cms1000-200mesh20cmExtractionFromMiddle.csv} \\
% mesh300-200: \textit{tubular1LMH4cms300-100mesh20cmExtraction0.csv}\\
% mesh300-20: \textit{tubular1LMH4cms300-20mesh20cmExtractionFromMiddle.csv}\\
% mesh300-10: \\ % shear- and gradient induced lift forces

\subsection{The {Segr\'e}-Silberberg effect: a qualitative validation \label{sec:QualVal}}
Rigid spheres flowing in tubular channels under laminar conditions are subjected to a radial migration due to shear gradient and inertia-induced lift forces. This effect is called the Segr\'e-Silberberg effect and was experimentally observed by \cite{Segre1961}. Following the work of \cite{Segre1961}, \cite{Matas2004} formulated a theoretical underpinning for this effect, which can be summarised as follows; \par
Neutrally buoyant particles, which are assumed to move at the speed of the surrounding fluid, are subjected to a lift force away from the channel's center axis when flowing in a channel due to the curvature of the Poiseuille flow . This force is counteracted by wall repulsion effects due to the asymmetric wake of the particles near the wall, inducing a lift force in the opposite direction \citep{Zeng2005}. For low Reynolds numbers, an equilibrium position is reached at approximately 60 \% of the distance from the center axis to the wall.\par
Non-neutrally buoyant particles, which are assumed to lead or lag %\todo{WN: weet je al waar dit van afhangt? BDJ Nope } 
the surrounding fluid velocity, are additionally affected by the Saffman lift force, Eq.\ \eqref{eq:lift3D}, and the equilibrium position is not fixed but depends on the size of the channel, the Reynolds number (Eq.\ \eqref{eq:Reynolds}) and the particle Reynolds number \gls{Rep}:	 
\begin{equation}
 Re_p = \cfrac{\mathbf{U}_\mathrm{m}\, \gls{partDia}}{\gls{fluidKin}\, D_\mathrm{h}} \, ,
 \label{eq:PartRe}
\end{equation}
with \gls{Um} [\unit{\metre\,\reciprocal\second}] the maximal channel velocity and \gls{Dh} [\unit{\metre}] the hydraulic diameter \citep{Segre1961,Matas2004}. \\ \\
High Reynolds numbers indicate a flow regime that is governed by inertial forces. Saffman's lift force is one of these forces, so it is not suprising that high Reynolds flows result in a more pronounced radial migration %\todo{weet je ook welke richting: naar ax of naar wall? BDJ: een artikel zegt als ze leaden dan ist naar beneden} 
of dispersed particles. The same reasoning is valid for the particle Reynolds number \citep{DiCarlo2007}.  \par
The theoretically and experimentally acknowledged Segr\'e-Silberberg effect provides an ideal opportunity for a qualitative validation of the force balance. By means of a series of in silico experiments, an attempt was made to replicate this effect and study its dependency on both Reynolds numbers. As a full analysis of the effect of all variables in Eq.\ \eqref{eq:Reynolds} and \eqref{eq:PartRe} is not feasable within the context of this master thesis, only the effect of the particle diameter \gls{partDia} and the crossflow velocity \gls{Ucf} was investigated. It should be noted that the particle diameter only affects \gls{Rep} while the crossflow velocity impacts both \gls{Re} and \gls{Rep} (Eqs.\ \eqref{eq:Reynolds} and \eqref{eq:PartRe}).

% effect of Re \\
% effect of dp \\
% preferential concentration \\
% 
% 50:10:100µm particles for 1LMH and zero 1LMH 10cms 0.5 MLSS \textit{1LMHFilterTorus10cmsMLSS05.h5}
% 100µm particles for 1 LMH and zero LMH 4cms 10 MLSS
% F
% 1000*200 lijst de goeie mesh te zijn
% simulaties:
% \textbf{effect of Re}
%  \begin{itemize}
%  \item 1000*200 4cms 1LMH: \textit{particleTracerExperiment4cms1LMH-1000-200mesh.h5}
%  \item 1000*200 1cms 1LMH: \textbf{Bezig}
%  \item 1000*200 10cms 1LMH: \textit{particleTracerExperiment10cms1LMH-1000-200mesh.h5}
% \end{itemize}
% \textbf{effect of flux}
% 4cms:
% \begin{itemize}
%  \item 1 LMH 1000-200: \textit{particleTracerExperiment4cms1LMH-1000-200mesh.h5}
%  \item 0 LMH
%  \item 2 LMH
% \end{itemize}
% 
% \begin{itemize}
%  \item wss 10cms stromingen nemen
% \end{itemize}
% \textbf{effect of dp}
% komt uit de rest van de simulaties
% wss nemen we die bij 10cms

\subsubsection{Setup}
For this validation, the tubular setup was used from Section \ref{sec:setup}. No flux was imposed at the membrane and an infinitely long tube was simulated by employing periodic boundary conditions at the inlet and outlet of the modelled system (Figure \ref{fig:tube}). Figure \ref{fig:CFDTube} demonstrates that this periodic boundary condition does not impose a problem for the $x$- and $y$-velocity as they are constant with respect to the $x$-direction in the fully developed middle section of the tubular membrane (30\,cm). The same conclusion is valid for the pressure, even though it is not constant, because only the gradient is considered by the force balance, which is also constant with respect to $x$. For the initial condition, particles were inserted at fixed points at the inlet and their $y$-position is followed in time.
\subsubsection{Effect of the particle diameter \label{sec:effectOfDiameter}}
First, the effect of the particle diameter is investigated. Particles of \unit{25}{\micro\metre}, \unit{70}{\micro\metre} and \unit{100}{\micro\metre} were introduced at the inlet, every \unit{0.5}{\milli\metre} and their lateral position was tracked for \unit{120}{\second}. The results of this simulation are displayed in Figure \ref{fig:effect} (a). \par 

It is clear that there is a clear correlation between the particle size and the radial migration in the high-shear regions of the channel, i.e.\ close to the membrane. This non-linear effect can directly be accounted to the inertial lift force  (Eq.\ \eqref{eq:lift3D}), which is quadratic with respect to \gls{partDia}. The smallest particles (\unit{25}{\micro\metre}) are almost unaffected by the radial migration. \par
As mentioned previously, the particles should theoretically reach an equilibrium position between the membrane and the center axis of the tube. This equilibrium position is not observed in Figure \ref{fig:effect}, but this is further discussed in Section \ref{sec:discValidation}.
% Lastly, it is interesting to note that the \unit{40}{\micro\metre} particles, under these conditions, are almost moving straight.

\subsubsection{Effect of the crossflow velocity}
Next, the effect of the crossflow velocity is examined. The setup was identical to the previous scenario but only one particle diameter i.e.\ \unit{70}{\micro\metre} was explored. Figure \ref{fig:effect} (b) depicts the results of this scenario. It can be seen that higher crossflow velocities bring forth an increased radial migration, just as predicted by \cite{Matas2004}. This effect demonstrates that particles are more likely to reach the membrane at high crossflow velocities. This observation is interesting as it is generally accepted that such conditions give rise to less filter cake formation by an increased detachment. This exemplifies the complexity of this process and indicates the need for an accurate calibration of the adhesion parameter $k$. In the long run, it would be meaningful to implement a more mechanistic approach like a force balance on filter cake particles (Section \ref{sec:discFilter}).

\subsubsection{Effect of the flux}
Since the effect of a radial flux is not included in the Segr\'e-Silberberg effect, it should be investigated how this affects the previous described balance of forces. To investigate this, the particle streamlines were evaluated for a flux of 0\,LMH, 36\,LMH and 72\,LMH, the last two are typical values for membrane filtration. The results of this in silico experiment are presented in Figure \ref{fig:effect} (c). It can be seen that the flux is the most important factor impacting  the radial migration of suspended solids. When applying a flux, smaller particles which are less influenced by the radial migration effects, are also transported towards the membrane surface. Under these conditions the non-existence of an equilibrium position seems to be justified.
\begin{figure}[H]
 \centerline{
 \subfigure[]{\includegraphics[width=0.9\textwidth]{figs/effectOfDiameter.png}}
 }
 \centerline{
 \subfigure[]{\includegraphics[width=0.9\textwidth]{figs/effectOfCFV.png}}
 }
 \centerline{
 \subfigure[]{\includegraphics[width=0.9\textwidth]{figs/effectOfFlux.png}}
 }
 \caption{Effect of the particle diameter (a), crossflow velocity (b) and transmembrane flux on the radial migration of particles in a microchannel.} \label{fig:effect} 
\end{figure}

\subsection{Mesh independency of the agent-based model} %onstabiel evenwichts probleem
Checking the mesh independence is good modelling practice for \gls{CFD} and since the Lagrangian modelling framework relies on the pressure and velocity field generated by \gls{CFD}, it seems only logical to perform a mesh independency check on the \gls{ABM} as well. This might provide information about the sensitivity towards variations in the flow field. The particle streamlines are simulated for the same mesh sizes as the mesh independency in Section \ref{sec:MeshIndep}.\par  %, the setup remains the same as it seems to be a straightforward and correct approach for this qualitative assessment. \par
From Figure \ref{fig:effectOfMesh} it is clear that the movement of the particles shows the same sensitivity towards the mesh size as the velocity profiles, which is expected as some forces are directly proportional to the velocity. The regions with the largest deviation of the streamlines are related to the regions with the largest deviation of the velocity profile, i.e.\ at high $x$-velocities. For particles introduced near the center axis of the tube, the radial lift force is almost zero and, hence, these are in an unstable equilibrium position. Therefore, the particle tracers at $y =$ \unit{3.9}{\milli\metre} are not significantly impacted by the mesh size. 

\begin{figure}[H]
 \centering
 \includegraphics[width=1\textwidth]{figs/effectOfMesh.png}
 \caption{Effect of the mesh size on the particle streamlines in a tubular membrane filter. \label{fig:effectOfMesh}}
\end{figure}

\subsection{Spatial segregation of the suspended particles \label{sec:spatSeg}}
The dependency of the radial migration velocity on the particle size, demonstrated in Section \ref{sec:effectOfDiameter}, should lead to the segregation of deposited particles along the longitudinal axis of the membrane. This effect will be most noticeable for an adhesion probability of 100\%, as it eliminates the stochasticity of the filter cake formation. Hence, perfect stickiness of the particles was assumed for this simulation, to demonstrate the segregation. Particles were introduced in the lower \unit{1}{\milli\metre} of the inlet and the deposition position of the particles was tracked. A normalisation of the number of deposited particles at each bin with respect to the total number of deposited particles gives rise to a histogram as depicted in Figure \ref{fig:depProb}. A flux and crossflow velocity of respectively 144\,LMH and \unit{10}{\centi\metre\,\reciprocal\second} were employed and the dispersed phase consisted of number-wise equally represented particle sizes of 40, 70 and \unit{100}{\micro\metre}. It was chosen to only consider the lower \unit{\milli\metre} in order for all entering particles to deposit within a reasonable time frame. The same reasoning can be used to motivate the choice of a higher flux and crossflow velocity as it brings forth a faster radial migration. %Moreover, considering the entire \unit{4}{mm} will only result in an elongated histogram, though the main observations would probably be very similar. 
\begin{figure}[H]
 \centering
 \includegraphics[width=1\textwidth]{figs/depProb.png}
 \caption{Relative frequency of the number of deposited particles in function of the distance to the inlet for particles of 40, 70 and \unit{100}{\micro\metre}. \label{fig:depProb}}
\end{figure}
% \todo[inline]{misschien eens kijken om dit te vervangen door 3 afzonderlijke histogrammen}
% \todo[inline]{make transparant}
Figure \ref{fig:depProb} shows that there is indeed a longitudal segregation of the suspended solids. The largest particles deposit closer to the inlet than the smaller particles. There is a clear cut-off distance for each particle size which is the deposition position of the particles that are introduced at the top of the inlet. As a consequence, all particles deposit within this distance. It is also clearly visible that the number of depositions decreases with the distance, which can be explained by the fact that, close to the membrane, a small difference in inlet position induces only a small difference in the particle trajectory due to the low velocity of the continuous phase. For higher fluid velocities closer to the middle of the tube a small change in the entry position results in a big difference in the trajectory.\par
It should be kept in mind that an adhesion probability of 100\% is not realistic but the purpose of this experiment was to characterise the movement of the suspended particles and not the simulation of a realistic filter cake build up.
%with P=1; the position of particles is \textbf{eenduidig bepaalt, voor hogere k's treedt een ``smearing'' effect op en wordt de positie minder eenduidig bepaalt waardoor grotere 6partikels ook veel verder gaan}
% mesh1000-200: \textit{particleTracerExperiment4cms1LMH-1000-200mesh.h5}
\subsection{Effect of polydispersity \label{sec:multidispresult}}
In order to assess the effect of polydispersity, a comparison was made between the filter cake of a mono- and polydisperse bulk phase. This is also the ideal opportunity to perform a full-fledged simulation with filter cake formation to identify the model imperfections. \par 
The polydisperse simulation used an experimentally determined \gls{PSD} of \gls{MBR} sludge (Figure \ref{fig:PSDplot}) and the monodisperse simulation used the number average particle size of this distribution, which was \unit{40.82}{\micro\metre}.  The filter cake formation was simulated for \unit{15}{\second} and each scenario was run four times in order to assess the stochastic nature of the model, which is depicted as a band (average, minimum and maximum) in Figure \ref{fig:monoVsMulti}. 
The average filter cake thickness in time is presented in Figure \ref{fig:monoVsMulti} (a) and shows that polydispersity gives rise to a faster filter cake build up. Section \ref{sec:spatSeg} demonstrates the fast deposition of big particles and, in a polydisperse setting, these will ``catch'' smaller particles giving rise to a fast filter cake development. This synergistic effect is not present in a monodisperse setting and the filter cake is thereby formed at a slower pace. The evolution of the filter cake porosity is shown in Figure \ref{fig:monoVsMulti} (b). Initially, there is no filter cake and the porosity is equal to one, and due to the formation of filter cake patches the porosity starts decreasing at a higher rate in the polydisperse setting. This could, however, be due to a denser filter cake or a higher surface coverage, which cannot be determined from the figure. For this, the porosity of the actual filter cake patches has to be evaluated (Figure \ref{fig:monoVsMulti} (c)). It can be seen that the filter cake in the polydisperse setting is more densily packed than in the monodisperse case. As mentioned previously, polydispersity should theoretically lead to a denser packing of the particles in the filter cake. Hence, the simulation results are in agreement with the literature \citep{Desmond2013,partPack2}. \par
However, these simulation results still have their imperfections. Figure \ref{fig:sideViewCake} shows a two-dimensional projection of the centers of the deposited particles after \unit{2}{\minute} of simulation (the monodisperse case). Note the unrealistic architecture of the filter cake, embodied by the formation of narrow filter cake piles. This observation is confirmed by the longitudal variance of the filter cake thickness, presented in Figure \ref{fig:monoVsMulti} (d). Hence, it can be concluded that the problem of particles piling up was not resolved by transitioning to a three-dimensional model.    

\begin{figure}[H]
\centerline{
    \subfigure[]{\includegraphics[width=0.44\textwidth]{figs/aveThick.png}}
    \subfigure[]{\includegraphics[width=0.44\textwidth]{figs/por.png}}
    }
\centerline{
    \subfigure[]{\includegraphics[width=0.44\textwidth]{figs/patchpor.png}}
    %\subfigure[]{\includegraphics[width=0.47\textwidth]{figs/maxThick.png}}
    \subfigure[]{\includegraphics[width=0.44\textwidth]{figs/varHeight.png}}
    }
    \caption{Evolution of the simulated filter cake characteristics in time. (a) Average filter cake thickness, (b) global filter cake porosity, (c) actual filter cake porosity, (d) longitudinal variance of the filter cake thickness.   \label{fig:monoVsMulti}}
\end{figure}

\begin{figure}[H]
\centering
\includegraphics[width=0.8\textwidth]{figs/sideViewCake.png}
\caption{Two dimensional projection of the filter cake after \unit{2}{\minute}. The dots represent particle centers and are not to scale with the particle size. \label{fig:sideViewCake}}
\end{figure}

% \begin{itemize}
%  \item k = 2
%  \item t = 15s
%  \item writeout = 0.01s
%  \item inlet velocity = 10 cm/s
%  \item Re = 800
%  \item LMH = 1
%  \item 6*e-04 breed
%  \item 1g/L
% \end{itemize}
% Mono: 1LMH10cms1gl15sk2-monodisp.h5
% Multi: 1LMH10cms1gl15sk2.h5
% number weighted average of the PSD is 40.82 µm
% \subsection{particle distance-distance}

% \subsection{comparison tubular membrane vs our MBR for the same loading rate}
% \todo[inline]{addition: misschien nog een stukje over the considerations die gemaakt zijn over de grote van de het computationele domein, te grote was is veel te zwaar maar bij kleiner domein is het wel realistisch? -> Ja, want de particles die in het bovenste binnenkomen zullen gedurende de verblijftijd toch niet neerzetten op het membraan ze hebben geen tijd genoeg.}
% 
% \subsection{2D vs 3D}
% \todo[inline]{addition: interpolation of the CFD simulations}

% For the experimental set-up a few designs for the membrane filtration unit where evaluated using openFOAM platform for CFD.
% 
% \begin{itemize}
% \item pipe-box-pipe box: 4cm 10cm 30cm 
% MBRPipe
% \item funnel-box-funnel, funnel: 0.4cm 1cm 20cm rectangle is in center, box dimensions: 4cm 10cm 30cm
% MBRHoekigeMesh
% \item funnel-box-funnel, funnel: 0.4cm 1cm 20cm inlet  rectangle is inline with bottom, box dimensions: 4cm 10cm 30cm
% MBRWedge
% \item funnel-box-funnel, funnel: 0.4cm 1cm 20cm inlet  rectangle is inline with bottom outlet is inline with top, box dimensions: 4cm 10cm 30cm
% MBRReverseWedge
% \item funnel-box-funnel, funnel: 0.3cm 1cm 20cm inlet  rectangle is inline with bottom, box dimensions: 3cm 10cm 30cm
% \item funnel-box-funnel, funnel: 1.5cm 1.5cm 20cm inlet SQUARE is inline with bottom, box dimensions: 3cm 10cm 30cm
% MBRWedgeSquare
% \item funnel-box-funnel, funnel: 2.7cm 2.7cm 20cm inlet SQUARE is inline with bottom, box dimensions: 2.7cm 10cm 30cm
% MBRWedgeBigSquare
% \item funnel-box-funnel, funnel: 2cm 2cm 20cm inlet SQUARE is inline with bottom, box dimensions: 2cm 10cm 30cm
% MBRWedgeMediumSquare
% \end{itemize}

%----------------------------------------------------------------------------------------------------------
\clearpage
\clearpage{\pagestyle{empty}\cleardoublepage} 
\hyphenation{}
\chapter[General discussion and perspectives]%
{General discussion and perspectives\label{ch:disc}}

Despite of the accomplished progress, the model still has a few shortcomings and it is important to address these. Besides, looking forward, guidelines and remarks are provided for future improvements. In the final part of this chapter, different sources of numerical instability are mapped out and a brief discussion on the suitability of different profilometric techniques for the specific task of characterising the filter cake is given.
\section{Remarks on the bulk phase force balance \label{sec:discValidation}}
In Chapter 5, a qualitative validation is performed of the force balance in the model. The results were mostly in agreement with the theoretical description of \cite{Matas2004} and the experimental observations of \cite{segre1962} and \cite{DiCarlo2007}. Still, the absence of an equilibrium position raises questions about the integrity of the force balance (Eq.\ \eqref{maxeyRiley}). The reason for this apparent shortcoming should be sought in the forces that are able to counteract the radial migration of particles. \cite{Zeng2005} indicate that the presence of a nearby wall breaks the axisymmetry of the wake vorticity distribution as well as a tortuosity change, inducing a lift force that moves the particles away from the wall. This mechanism is not explicitly included in the force balance but might be accounted for under the guise of the Fax\'en correction:
\begin{equation}
 \mathbf{U}_\mathrm{r,eff} = \mathbf{U}_\mathrm{r} - \frac{1}{24}d_\mathrm{p}^2 \nabla^2 \mathbf{U}_\mathrm{c} \, ,
 \label{eq:faxen}
\end{equation}
with \gls{Ueff} [\unit{\metre\,\reciprocal\second}] the Fax\'en corrected relative velocity and $\mathbf{U}_\mathrm{r}$ the actual relative velocity of the particles.
In two dimensions, the Laplacian of Eq.\ \eqref{eq:faxen} can be written as,
\begin{equation}
 \nabla^2 \mathbf{U}_\mathrm{c}= \cfrac{\delta^2 \mathbf{U}_\mathrm{c}}{\delta x^2}+\cfrac{\delta^2 \mathbf{U}_\mathrm{c}}{\delta y^2} \, .
\end{equation}
A fully developed Poiseuille (parabolic) flow is assumed in the tube. For this flow regime, the second derivative of $\mathbf{U}_\mathrm{c}$ with respect to $y$ is constant along the width of the tube. Hence,
\begin{equation}
 \cfrac{\delta^2 \mathbf{U}_\mathrm{c}}{\delta y^2} = \mathrm{c} \, .
\end{equation}
Furthermore, $\mathbf{U}_\mathrm{c}$ is constant with respect to $x$ under these conditions, so
\begin{equation}
 \cfrac{\delta^2 \mathbf{U}_\mathrm{c}}{\delta x^2} = 0 \, .
\end{equation}
Therefore,
\begin{equation}
\frac{1}{24}d_p^2 \nabla^2 \mathbf{U}_\mathrm{c} = \mathrm{c} \, ,
\end{equation}
and this shows that the Fax\'en correction is constant for a Poiseuille flow and is clearly not the wall repulsion effect described by \cite{Zeng2005}, which reaches a maximum magnitude near the walls. \par
In order to increase the accuracy of the force balance, it might be interesting to investigate the addition of the wall-induced lift force as it seems to have a considerable impact on the flow of particles. Furthermore, next to Saffman's lift force, two additional lift forces are described in \cite{Matas2004}; the shear gradient-induced lift force due to the curvature of the velocity profile and the Magnus effect due to the forced rotation of the particles. It is likely that the Magnus effect is negligible compared to the inertial lift force, but the shear gradient-induced lift force might be of the same order of magnitude, and should therefore be investigated. \par
Due to the uncertainty of the magnitude of these forces and the complex nature of their interaction with other operational variables, it is meaningful to first validate the model in order to assess the necessity of an extension of the force balance.
%onsider a point ($x_\mathrm{1}$, $y_\mathrm{1}$) near the wall of the tube, where the magnitude of the opposing lift force should reach its maximum value.

%\section{Particle rolling}

\section{Filter cake formation \label{sec:discFilter}}
The simulation results in Section \ref{sec:multidispresult} indicate the formation of filter cake towers, in spite of the extension of the model to three dimensions. In order to resolve this unrealistic behavior, a rolling algorithm should be included in the model. This will assure that only particles that reach a stable position will attach to the filter cake. As mentioned previously, such a framework is described in \cite{Smets} and should be assigned a high priority in the further development of the model as it is imperative for the simulation of realistic filter cake structures. \par
Next to the packing of particles, it is meaningful to reflect on the adhesion probability equation (Eq.\ \eqref{eq:stickychance}). Although this equation serves its purpose of constraining the filter cake formation to low shear regions quite well, several arguments indicate the necessity for the revision of this equation.
Firstly, due to the stochastic nature of the model, there is always a chance that particles adhere in high shear regions, and currently these particles remain attached indefinitely. Secondly, the simulations indicate a positive correlation of the radial migration magnitude and the particle size, resulting in an easier filter cake formation for bigger particles. This contradicts the experimental observations in \cite{Li1998}. %\todo{nog eens twee PSDs met elkaar vergelijken}. 
Hence, it seems that Eq.\ \eqref{eq:stickychance} is not able to accurately simulate the balance between adhesion and detachment of particles to the membrane or filter cake, which is a complex interplay between particle-particle interactions (attachment to  the filter cake), particle-interface interactions (attachment to the membrane) and the liquid flow (shear at the membrane, compression due to \gls{TMP}, etc.). A theoretical foundation for particle-particle and particle-interface interactions in dispersions is provided by the \gls{DLVO} theory, which determines the balance between the electrostatic repulsion and the van der Waals attraction \citep{Lyklema1968}. However, the \gls{DLVO} theory does not comprise all the necessary processes for a complete representation of the interactions. The agglomeration of particles is also influenced by hydration forces, steric forces and hydrophobic interactions \citep{VanOss1989,Hermansson1999}. Finally, the hydrodynamics and pressure-driven effects should also be accounted for as it is hypothesised that these effects greatly influence the formation of the filter cake. It can be concluded that Eq.\ \eqref{eq:stickychance} is too simple for an accurate representation of filter cake formation and it is clear why it does not yield satisfactory filter cake structures. Moreover, a mechanistic approach in the form of a force balance over the particles in the filter cake is imperative in order to simulate and comprehend backwashing and aeration processes.
%Next to the rolling of particles, the absence of a mechanism for filter cake detachment might also be the cause. To be able to account for processes, such as backwashing and aeration this mechanism should be in the model.\par %The increased radial migration is in agreement with the literature and based on elimination the lacking mechanisms seems to be filter cake detachment. 
% In order to maintain the mechanistic modelling mindset, it would be meaningful to take a look at mechanistic approaches to replace the adhesion probability equation. This could be in the form of a force balance, similar to the force balance in the bulk phase, taking into account \textbf{inter-particle} interactions, the drag force of the liquid, friction with the membrane and .... . %(Inspiration could be found from biofouling models, but mostly these use simple detachment rules, which are valid for the bottom up formation of filter cake as a result of microbial growth, but not for our case.)

\section{Coupling ABM and continuous model}
%The Lagrangian model of the dispersed phase and the Eulerian model of the continuous phase, the flow fields are used to model the movement of the suspended solids and the filter cake build up but 
As mentioned in Chapter \ref{spatModel}, the two model layers are coupled unidirectionally. The steady-state flow profile of the continuous phase is calculated once at the beginning of the simulation and is used throughout the whole simulation. Hence, the formation of the filter cake has no impact on the continuous phase and the permeate keeps flowing unperturbed, regardless of the fouling severity, which is not at all realistic. %In a constant pressure membrane filter, a pump employs a pressure differential over the membrane and induces a permeate flux through its pores. Particles accumulate on the membrane surface and a filter cake is gradually formed, increasing the filtration resistance and decreasing the flux. 
Two effects can be identified that demonstrate the impact of filter cake on the surrounding fluid. \par
The first is the effect of the filter cake by increasing the filtration resistance. An increase in filtration resistance due the presence of a porous filter cake in a certain region will make the fluid move towards cleaner regions, which implies the spatial variation of the flux. %This process can be incorporated by developing a \gls{RIS} model. \par
%that evaluates the Kozeny-Carman equation in order to determine the flux through each cell of the membrane face. This information needs to be transferred to OpenFOAM where a new flow pattern is established. So in order to construct a realistic filter cake formation model, the coupling of the model layers should be bidirectional. \par   %Three potential approaches can be used to achieve this and are elaborated below. \par
Additionally, the presence of a filter cake has an effect on the velocity of the surrounding fluid. Here, it is not the flux that is affected, but rather the general flow profile.    %, the presence of a thick filter cake at the walls of the \gls{MBR} will have an impact on the fluid velocity. 
Both effects are currently not included in the model but with for a realistic filter cake formation model, at least the increasing filtration resistance should be considered.%, this can be accomplished in several ways. 
\par
To model this, a \gls{RIS} approach can be applied with two resistance terms, the clean membrane resistance ($R_\mathrm{m}$) and the filter cake resistance ($R_\mathrm{c}$):
\begin{equation}
 R_\mathrm{tot}= R_\mathrm{m} + R_\mathrm{c} \, ,
 \label{RIS4}
\end{equation}
where $R_\mathrm{c}$ can be obtained from:
\begin{equation}
 R_\mathrm{c}= \gls{dl}\, \gls{Kc},
\end{equation}
and \gls{Kc} is obtained from:
\begin{equation}
 \gls{Kc}= \cfrac{\gls{K}\, 90}{\gls{dpHead}} \, \cfrac{(\gls{eps})^2}{(1-\gls{eps})^3} \, .
\end{equation}
where \gls{Kc} [\unit{\rpsquare\metre}] is the specific cake resistance, \gls{dl} [\unit{\metre}] the filter cake thickness, \gls{K} [\,-\,] the kozeny constant, \gls{dpHead} [\unit{\metre}] the mean particle diameter and \gls{eps} [\,-\,] the filter cake porosity. \par
By plugging the total filtration resistance $R_\mathrm{tot}$ into Darcy's law, an expression is obtained to calculate the flux:
\begin{equation}
\gls{J} =  \cfrac{\gls{dp}}{R_\mathrm{tot}\, \gls{fluidKin}},
 \label{darcy2}
\end{equation}
The flux decrease could be considered homogeneously over the entire membrane. However, the creation of a realistic filter cake architecture requires a spatially variable flux. So, a sectional approach similar to \cite{Li2006} should be used (Figure \ref{fig:sectional}). Hence, the membrane surface is subdivided in sections with a constant length ($\Delta x$). The flux through each section is calculated with Eqs.\ \eqref{RIS4}-\eqref{darcy2}. The newly calculated fluxes are subsequently employed as boundary conditions on the membrane face and a new flow profile has to be generated via \gls{CFD}. \par
\begin{figure}[H]
    \centering
    \def\svgwidth{0.6\columnwidth}
    \input{figs/sectional.pdf_tex}
    \caption{Schematic representation of a heterogeneously distruted flux along the membrane surface. \label{fig:sectional}}
    \label{sectional}
\end{figure}
In order to account for the variable flux, a bidirectional coupling  has to be established between \gls{CFD} and the \gls{ABM}, which is, however, not straightforward because both model layers are implemented in different software platforms. Nonetheless, two approaches are identified to establish such a coupling. \par
The first approach involves using a third program as a wrapper that distributes the flux data from MATLAB to OpenFOAM and the flow profile from OpenFOAM to MATLAB. This approach is the most straightforward and can be accomplished by means of shell scripting. The second approach involves setting up a MATLAB engine script in OpenFOAM that compiles MATLAB's m-code to C++. This approach is much more complex, but is computationally very efficient. \par
%http://www.tfd.chalmers.se/~hani/kurser/OS_CFD_2012/JohannesPalm/projectPres.pdf
%http://www.tfd.chalmers.se/~hani/kurser/OS_CFD_2012/JohannesPalm/Connecting_Matlab_with_OpenFOAM_JP_peered1.pdf
In order to be able to account for the direct effects of the filter cake on the surrounding liquid, it is necessary to explicitly model the suspended particles and the interactions with the continuous phase. This can only be achieved with discrete element methods.

% \todo[inline]{Misschien nog kort LIGGGHTS aanraden http://www.cfdem.com/liggghts-open-source-discrete-element-method-particle-simulation-code}

% the presence of suspended solids in the fluid is not taken into account by the \gls{CFD}. When the concentration of suspended solids is sufficiently low, these effects can be neglected.However, in the filter cake particles are concentrated and the impact on the surrounding fluid is not negligible. \\ \\In order to This approach is a lot more complicated and computationally demanding as it requires the explicit modelling of sludge particles and interaction with the liquid phase. The \textbf{strategy} for including the \textit{direct} and \textit{indirect} effects of filter cake formation on the liquid flow are discussed below. \\ \\
% As mentioned above, the \textit{indirect} effects of filter cake formation are governed by the flux decrease over time. To model this, a \gls{RIS} approach is applied with two resistance terms, the clean membrane resistance ($R_m$) and the filter cake resistance ($R_c$),
% 
% 
% \begin{itemize}
%  \item moeilijker
%  \item wss sneller
% \end{itemize}
% A third possibility is translating the \gls{ABM} from matlab to C++ which is the programming language used by openFOAM
% \begin{itemize}
%  \item A lot of different solvers for differential equations
%  \item is a lot of work and matlab is a higher level language so some functions might not be available in OpenFOAM?
% \end{itemize}
% 
% %http://www.tfd.chalmers.se/~hani/kurser/OS_CFD_2012/JohannesPalm/projectPres.pdf
% %http://www.tfd.chalmers.se/~hani/kurser/OS_CFD_2012/JohannesPalm/Connecting_Matlab_with_OpenFOAM_JP_peered1.pdf
% The \textit{direct} effects: DEM uitleggen en LIGGGHTS voorstellen
% 
% \todo[inline]{misschien dit nog eens uitleggen}
% \textbf{intro...} \par
% \begin{itemize}
%  \item establish the link to a third programm ing environment such as bash
%  \item translate the code to c++ en implement abm in c++ 
%  \item Compile the matlab code in openFOAM (zie bestandje)
%  \item establish the link to a third programming environment such as bash
%  \item but ideally, a discrete element package such as lightss should be used.
% \end{itemize}
% 
% \begin{itemize}
%  \item is the resolution of the CFD meshing small enough to capture the local changes in flux
%  \item something interesting to keep in mind it the EPS are not modelled here but it might be sensible to also perform viscosity measurements of the continuous phase in order to increase the accuracy of the continuous model.
% \end{itemize}

\section{Sources of numerical instability}
It is important to keep in mind that this spatio-temporal model basically consists of a system of \gls{PDE}s which are discretised and solved numerically. These methods can suffer from numerical instabilities and this is also the case for this model. As it seems important for future work, a brief overview is provided of the causes of numerical instability. \par
The first source of instability is described in \cite{Ghijs2014} and concerns the Stokes drag force which introduces a maximum solver time step limited by the smallest particle diameter, i.e.\
\begin{equation}
 \Delta t \leq \cfrac{\rho_\mathrm{p}\, \gls{partDia}^2}{18 \, \gls{fluidKin}} \, .
\end{equation}
A second source of instability are the velocity fields from \gls{CFD}. The convergence criteria of the OpenFOAM solvers are set to $10^{-7}$ which means that openFOAM will stop iterating when the difference between the fields of the current and previous iteration is smaller than $10^{-7}$. This results in velocity fields where zeros are actually numbers between $10^{-8}$ and $-10^{-8}$. Although seemingly unimportant, this causes numerical instabilities due to the sensitivity of the force balance to the sign of the fluid velocity. This issue can be easily solved by employing a filter on the velocity values setting all values below the convergence criteria to zero. \par
Lastly, an additional constraint on the solver time step was found, this time originating for large particles ($\gls{partDia} > \unit{100}{\micro\metre}$) in the system. Under high crossflow velocities and fluxes, the radial migration of particles is a lot more prominent and forms another source of numerical instability. Nevertheless, it is likely that for more extreme conditions this instability will also impact smaller particles and it should be further investigated.

\section{Profilometry}
In Chapter \ref{litRev}, an overview was given of the prominent profilometric techniques along with their lateral resolution and application restrictions. Based on this information, it is possible to make a selection of the profilometers that are most suitable for the characterisation of the filter cake surface. The filter cake should be characterised at micrometre scale, and techniques such as \gls{AFM}, \gls{SEM}, \gls{STM} are mostly used in the lower regions of the microscale and nanoscale. Although a high resolution does not impose a direct problem, this implies a low surface covering capacity and a high price tag, making these profilometers inappropriate for the task at hand. The contact-based modus operandi of stylus profilometry can disturb the surface of the soft filter cake and give rise to faulty measurements. %Hence, white light axial chromatism and interferometry seem to be the most suitable techniques for profiling filter cakes. 
Furthermore, white light axial chromatism provides direct measurements of the distance, in contrast to the interferometry-based techniques where the distance is derived from other measured quantities. Hence, it seems that white light axial chromatism is the most suitable technique for profiling filter cake.

%----------------------------------------------------------------------------------------------------------
\clearpage
\clearpage{\pagestyle{empty}\cleardoublepage}
\chapter[Conclusion]%
{Conclusion \label{ch:concl}}
\hyphenation{si-mu-la-tion}
Aiming at disclosing the mechanisms of filter cake formation, the model framework established by \cite{Ghijs2014} %to describe this process on a microscopic level 
was succesfully extended. The representation of multidisperse feed flows was implemented, enabling a realistic simulation of \gls{MBR} feed flows characterised by a \gls{PSD}. Progression was made towards the formation of realistic filter cake structures by transitioning to a three-dimensional model. The implementation of a new filter cake attachment algorithm and new sampling procedures induced an enhanced accuracy and computational efficiency. Furthermore, the model implementation was parallelised, in order to speed-up the calculations and a post-processing tool was developed for a user-friendly analysis of the simulation results.
The simulations of the extended model yielded interesting results and several conclusions can be formulated. \\ \\
First of all, an experimental microfiltration unit for the calibration and validation of the model was designed by means of \gls{CFD}. The unit presented a perfect laminar flow regime without dead volumes or recirculation streams and enables a simplified representation of the filtration unit, lowering the computational demand of the model. The \gls{CFD} simulations indicated that in order to obtain such a flow regime, a gradual transition from the inlet to the membrane compartment is imperative. \par
Next, a qualitative validation showed that the model was able to reproduce the radial migration of non-buoyant particles in a Poiseuille flow, governed by the Segr\'e-Silberberg effect. The theoretically and experimentally acknowlegded dependency of the radial migration magnitude on the Reynolds and particle Reynolds number was succesfully reproduced by the model. However, the occurence of an equilibrium position at approximately 60 \% of the distance from the center axis to the wall was not observed. This is most likely due to the absence of the wall-induced lift force in the force balance. Furthermore, the effect of the shear gradient-induced lift force should also be investigated in order to determine its relevance for filter cake formation. Through the Segr\'e-Silberberg effect, a spatial segregation of the suspended solids was demonstrated; bigger particles are more influenced by the inertial lift force and migrate faster towards the walls of the tube. \par
 Lastly, a scenario analysis was performed and the results indicated a large effect of polydispersity on the fouling rate and filter cake porosity.  Multidisperse feed flows result in a faster filter cake build up and less porous filter cakes than monodisperse feed flows. The need to implement a particle rolling algorithm was concluded due to the formation of unrealistically narrow and high filter cake piles. Additionally, the adhesion equation should be calibrated or replaced with a force balance on the particles attached to filter cake/membrane and a bidirectional coupling between the two model layers should be implemented in order to simulate the increased filtration resistance due to membrane fouling, and predict a flux decline or \gls{TMP} increase for constant pressure or constant flux membrane filtration systems, respectively. \\ \\
 To conclude, the spatio-temporal model was extended and a considerable progress was made towards realistic filter cake formation simulations. This research has yielded several new insights in the underlying mechanisms of membrane fouling and indicated the importance of the lift force induced migration of suspended solids in tubular membranes as well as the impact of polydispersity.
%  \todo[inline]{Is het een probleem dat hier deze calibratie en validatie komt na de bespreking van de resultaten maar in het werkje dit eerst was, het leek me gewoon logisch zo.}
%  \todo[inline]{Zijn er nog dingen die in de Appendix moeten?}

 
%\todo{addition: new stability issues}

% \large{Checkbox for finalisation}
% 
% \begin{itemize}
%  \item Everything in glossaries?
%  \item All equation refs to eqref
%  \item introduction of units
%  \item 
% \end{itemize}
% 
% % Addition of:
% \textbf{einde van dit stukje}
% In \cite{Ghijs2014} the flow profile is generated only once at the beginning of the filter cake simulation. This profile is used throughout the entire simulation to describe the flow of the continuous phase. As mentioned above, this approach does not allow to capture the local effects of filter cake formation on the fluid velocity. Moreover, the filter cake formation will lead to locally decreased fluxes that will have also have an impact on the local fluid velocity. Now, the flux is assumed constant throughout the simulation which is not realistic assumption. \par
% The impact of the changing flux on the flow profile in the system can be expected to be negligible due the small cake volume compared to the fluid volume in the system. However, the impact is substantial at the micro environment in close proximity of the filter cake surface. And it is here that the filter cake formation is most sensitive to variations in the flow profile. Consequently, it is necessary to elaborate a coupling between the two model layers. The flow profile will \textbf{ meermaals be aangepast gedurende de simulatie om zo the invloed van de lokale veranderingen te incorporeren.} This model extension is discussed in \textbf{sectie...}. 
% \todo[inline]{op het einde zeggen dat oorspronkelijk het maar eenmaal berekend wordt} 
% \todo[inline]{nog ergens beschrijven wat euler lagrange betekent: zie p 26 michael} 
% \todo{addition: not only important to make sure there is rolling of the particles in the direction of gravity but also in the direction parallel to flow as it can be expected that particles should be able to go around already formed pieces of filter cake.}
% 
% \todo[inline]{Appendix A technische tekening van de MBR en een fotoke}

%%%%%%%%%%%%%%%%%%possible extensions and questions for the defense %%%%%%%%%%%%%%%%%%%%%%%%%%%%%%%%%%%ù
% 
% \todo[inline]{\begin{itemize}
%   the effects of EPS are not included  Ji et al.2008 Enhancement of filterability... toont het belang van van liquid phase in fouling \\
%  no electrostatic forces are included which might have important consequences on filter cake formation (\textbf{ergens een bron vinden die dit staaft, ik denk dat broekmann of Yoon oof die forces consideren} zeker met het oog op vandermeeren \\)
%  compression of the cake layer \\
%  met een slice van de reactor te simuleren heb je geen behoud van massa want partikels kunnen via de Z-as verdwijnen maar niet bijkomen => \textbf{periodic boundary conditions nemen} \\
% \end{itemize}


% \todo[inline]{Vragen thesis: \\ What is LMH?}




%%%%%%%%%%%%%%%%%%%%%% Old TODOs and need checking if there is still time%%%%%%%%%%%%%%%%%%%%%%%%%%%%%%%%%%%%%%%%%%%%%%%%%%%%%
% \begin{itemize}
% \item diffusion \cite{Yoon1999}
% \item betere inertial lift force valid for laminar channel flows of any reynolds number \cite{Yoon1999}
% \item still a lot of forces are not taken into account in our model
% \item particle rolling after deposition \cite{Yoon1999}
% \item Looking at the coordination number and packing factors \cite{Yoon1999}
% \item \sout{In the advanced visualisation possible to click on initial particle and track it in time visualising all the forces this will lead to a better understanding of the mechanism of cake formation and why the velocity of the particle is higher then the fluid velocity}
% \item fix long time bug
% \item \sout{make ModelMainAnalysis write output in HDF5-format}
% \item \sout{acceleration of particle in information box, GUI}
% \item make a button, find diameters in the GUI
% \item \sout{plot time in GUI}
% \item find out why suddenly the forces change after settling
% \item \sout{Check if GUI actually tracks the right particle}
% \item Implement the advanced plotting into the GUI
% \item Check why the sedimentation function fails
% \item change all variable declarations to text not summation
% \item Change all equations to a readable format (bigger)
% \item Cao verder schrijven
% \item Alle vette tekst herschrijven
% \end{itemize}

 
%more chapters...

%intermezzo about making titles conform
\titleformat{\chapter}[display]{\flushright\vspace{-3cm}\pagecolor{white}}
{\usefont{OT1}{pag}{b}{n}\fontsize{30}{34}\selectfont {CHAPTER~\thechapter}}{10pt}{\usefont{OT1}{pag}{b}{n}\fontsize{30}{34}\selectfont}[{\thispagestyle{empty}\vspace{3cm}}]


%=======================================
%We want specific numbering here
% Comment this block if you want the same header/footer style as in main part for the references/appendix
% by keeping this, the numbering is simplified for bibliography/appendices
\fancyhf{}
\renewcommand{\headrulewidth}{0pt}
\fancyfoot[C]{\helv \thepage}                  % Paginanummers onderaan gecentreerd.
%=======================================

% ------------ REFERENCES ------------

\newpage
\addcontentsline{toc}{chapter}{Bibliography}
\bibliographystyle{apalike}  %apalike,phdbib.bst
\bibliography{Thesis,ThesisWebPageRef}

%\printindex                             % To print index, not done with thesis

% ------------ APPENDIX ------------
% it also possible to place your appendices in front of the references, copy then the entire app-blck
% and but the references part in the backmatter (uncomment this)
\appendix

\renewcommand{\chaptermark}[1]%
{\markboth{\MakeUppercase{APPENDIX~\thechapter \mdseries{~~~#1}}}{}}
%intermezzo about making titles conform
\titleformat{\chapter}[display]{\flushright\vspace{-3cm}\pagecolor{white}}
  {\usefont{OT1}{pag}{b}{n}\fontsize{30}{34}\selectfont {APPENDIX~\thechapter}}{10pt}{\usefont{OT1}{pag}{m}{n}\fontsize{22}{24}\selectfont}[{\thispagestyle{plain}\vspace{3cm}}]

%% APPENDIX ONE:
\hyphenation{}
\chapter[Appendix: Technical drawing microfilter]%
{Technical drawing microfilter \label{ch:tech}}

\begin{figure}[H]
\centering
  \includegraphics[width=1\textwidth]{figs/techTekExpSetup.pdf}
  \caption{Technical drawing of the microfiltration device}
\end{figure}



% 
% \section{Sensitivity base class code \label{senscode}}
% 
% \begin{lstlisting}[frame=none,columns=fixed]
% import os
% import numpy as np
% 
% from parameter import *
% import matplotlib.pyplot as plt
% from matplotlib.ticker import FixedLocator, MaxNLocator
% 
% class SensitivityAnalysis(object):
%     """
%     Base class for the Sensitivity Analysis
%        
%     Parameters
%     ----------
%     ParsIn : list
%         ModPar class instances in list or list of (min,max,'name')-tuples
%     
%     Attributes
%     -----------
%     ParsIn : list
%         a list of (min,max,'name') values, 
%         [(min,max,'name'),(min,max,'name'),...(min,max,'name')]
%     parmap : dict
%         tracks the sequence of the parameters 
%     Pars : list of ModPar instances
%         Used when working with the pyFUSE package
%     ndim :  int
%         number of uncertain input factors
%     namelist : list
%         list of the uncertain input factors used
%     
%     """
%     
%     def __init__(self,ParsIn):
%         '''
%         Check if all uniform distribution => TODO ! if all -> sobol sampling
%         is possible, else, only uniform and normal distribution are supported
%         for using the sobol sampling... Here is still work to do!!
%         '''
%         
%         if  isinstance(ParsIn, dict): #bridge with pyFUSE!
%             dictlist = []
%             for key, value in ParsIn.iteritems():
%                 dictlist.append(value)
%             ParsIn = dictlist
%             print ParsIn
%        
%         #control for other 
%         self.ParsIn = ParsIn  
%         self.parmap={} #dictionary linking ID and name, since dict instance has no intrinsic sequence
%         for i in range(len(ParsIn)):
%             if isinstance(ParsIn[i], ModPar): #or isinstance(ParsIn[i], pyFUSE.parameter.ModPar):
%                 cname = ParsIn[i].name
%                 self.Pars = ParsIn
%                 self.ParsIn[i] = (ParsIn[i].min, ParsIn[i].max, cname)
%                 self.parmap[i] = cname
%                 
%             elif isinstance(ParsIn[i],tuple):
%                 if ParsIn[i][0] > ParsIn[i][1]:
%                     raise Exception('Min value larger than max value')
%                 if not isinstance(ParsIn[i][0],float) and isinstance(ParsIn[i][1],float):
%                     raise Exception('Min and Max value need to be float')
%                 if not isinstance(ParsIn[i][2],str):
%                     raise Exception('Name of par needs to be string')                    
%                 self.parmap[i] = ParsIn[i][2]
%                 #create modpar instance of the tuple
%                 self.Pars=[]
%                 for par in ParsIn:
%                     self.Pars.append(ModPar(par[2],par[0],par[1],(par[0]+par[1])/2.,'randomUniform'))
%             else:
%                 raise Exception('The input type for sampling not correct,\
%                 choose ModPar instance or list of (min,max)-tuples')        
%         
%         self.ndim=len(ParsIn)
% 
%         self.namelist = [] 
%         for i in range(self.ndim):
%             self.namelist.append(self.parmap[i])        
% 
%     def WritePre(self,filename = 'inputparameterfile', *args, **kwargs):
%         '''
%         Parameterinputfile for external model, parameters in the columns files
%         and every line the input parameters
%         
%         Parameters
%         -----------
%         filename : str
%             name of the textfile to save
%         *args, **kwargs : args
%             arguments passed to the numpy savetext-function
%         '''
%         
%         np.savetxt(filename,self.parset2run,*args,**kwargs)
%         print 'file saved in directory %s'%os.getcwd()
%           
%     def ReadRuns(self,filename, *args, **kwargs): 
%         '''
%         Read model outputs (TODO: do sobol for multiple outputs, iterating the
%         post)
%         Format is: every output of the ithe MC on ith line
%         
%         output2evaluate can also be made on a other way
% 
%         Parameters
%         -----------
%         filename : str
%             name of the textfile to load    
%         *args, **kwargs : args
%             arguments passed to the numpy loadtext-function
%             
%         '''
%         self.output2evaluate = np.loadtxt(filename, *args, **kwargs)
% \end{lstlisting}
% 
% 
% 
% 
% \section{Model input file for PyFUSE model \label{inpufile}}
% 
% \begin{lstlisting}[frame=single,columns=fixed]
% ######################################################
% ##    Model Parameter input file
% ##    The parameter is defined by his distribution,
% ##    boundaries and extra info needed by distribution 
% ##    provide on each line one parameter with 
% ##    following information:
% ##    name : string
% ##        Name of the parameter
% ##    minval : float
% ##        Minimum value of the parameter distribution
% ##    maxval :  float
% ##        Maximum value of the parameter distribution
% ##    optguess : float
% ##        Optimal guess of the parameter, must be 
% ##        between min and max value
% ##    pardistribution : string
% ##        choose a distributionfrom: randomUniform,
% ##        randomTriangular, randomTrapezoidal, 
% ##        randomNormal, randomLogNormal
% ##    *kargs  :
% ##        Extra arguments necessary for the 
% ##        chosen distribution
% ##	Lines with ## marks are neglected
% ######################################################
% ## NAME	MIN MAX OPTGUESS DISTRIBUTION ARGS*
% S1max	50. 5000.000 400. randomTriangular 1000.
% S2max	100. 10000.000 1000. randomNormal 500. 25. 
% fitens 0.01 1.0 0.99 randomLogNormal	0.5 0.2
% firchr 0.050 0.950 0.5 randomTrapezoidal 0.4 0.6
% fibase 0.050 0.950 0.5 randomUniform
% r1 0.050 0.950 0.5 randomUniform
% ku 0.01 1000. 0.044 randomUniform
% c 0.99 20.0 1. randomUniform
% alfa 1.000 250. 150. randomUniform
% psi 1.000 5.0 2.5 randomUniform
% kappa 0.050 0.950 0.5 randomUniform
% ki 0.001 1000. 0.00833 randomUniform
% ks 0.001 10000. 0.5 randomUniform
% n 1.000 10. 3. randomUniform
% v 0.00001 0.250 0.004	randomUniform
% vA 0.001 0.250	0.0015 randomUniform
% vB 0.001 0.250 0.0015 randomUniform
% Acmax 0.050 0.950 0.5 randomUniform
% b 0.001 3.0 0.2 randomUniform
% loglambda	5.000 10.0 7.5 randomUniform
% chi 2.000 5.0 3.5 randomUniform
% mut 0.010 5.0 0.6 randomUniform
% be 0.99 4. 3.1 randomUniform
% alfah	0.01 0.99 0.5 randomUniform
% tg 0.0 0.7 0.3 randomUniform
% tif 0.0 0.7 0.26 randomUniform
% tof 0.0 0.7 0.12 randomUniform
% ko 0.01 0.99 0.15 randomUniform
% timeo 2. 48 24 randomUniform
% timei 2. 250. 20 randomUniform
% timeb 200. 10000. 2100. randomUniform
% \end{lstlisting}
%more appendices

%intermezzo
%\backmatter
\glsaddall
\end{document}