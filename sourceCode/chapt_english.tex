\chapter[Summary]{Summary}
\vspace{-2cm}
At present time, membrane filtration processes suffer from a high operational cost due to fouling abatement measures. The description of fouling mechanisms is still highly empirical and does not provide an adequate framework for the development of decision support tools to aid a cost-effective operation of these processes. The objective of this master dissertation is the development and evaluation of a spatio-temporal model to describe particle behaviour in a realistic and accurate manner in order to unravel the mechanisms of filter cake formation. This model aims to pinpoint the most influential processes that should be included in the next generation of decision support tools for filtration processes. \par In the context of this dissertation, a literature review was performed to identify the current fouling modelling approaches and to provide a overview of the available profilometric techniques for the calibration and validation of the model under development. The filter cake formation model of \cite{Ghijs2014} was further extended to a three-dimensional model, the representation of the feed flow was extended to multidisperse suspensions, a new and highly efficient collision detection algorithm was implemented and a user-friendly graphical user interface was developed for the analysis of the simulation results. \par 
A laboratory scale microfiltration device was designed with computational fluid dynamics for the calibration/validation of the model and a qualitative validation, based on the Segr\'e-Silberberg effect, was performed. The latter indicated some imperfections in the force balance on the flowing particles. A scenario analysis was carried out to assess the behaviour of the model under various operational conditions. \par
The simulations and qualitative validation led to a clear understanding of some of the modelling deficits and shortcomings, such as the formation of narrow filter cake patches and the absence of appropriate wall repulsion effects. Still, with an eye on the future development of the model, guidelines are provided to resolve these issues. \par
In the end, it can be concluded that a lot of progression was made towards a realistic representation of filter cake formation processes. Furthermore, a lot of knowledge was gained concerning their underlying mechanisms which was, after all, the main objective of this master thesis.